
Intuitivamente, un \textbf{anillo} es un conjunto donde están definidas
las siguientes operaciones:

\begin{itemize}
\item
  Suma.
\item
  Resta.
\item
  Multiplicación.
\end{itemize}

Estas operaciones deben satisfacer las propiedades habituales.

Además un anillo ha de contener elementos:

\begin{itemize}
\item
  Cero \(0\).
\item
  Uno \(1\).
\end{itemize}

Estos elementos deben satisfacer las propiedades usuales con respecto de
la suma y la multiplicación.

\hypertarget{ejemplos}{%
\subsection{Ejemplos}\label{ejemplos}}

\Begin{example}\textrm{\normalfont (Clásicos)} Los números enteros
\(\mathbb Z\), racionales \(\mathbb Q\), reales \(\mathbb R\) y
complejos \(\mathbb C\) son anillos, pero los naturales \(\mathbb N\)
no. \End{example}

\Begin{example}\textrm{\normalfont (Polinomios)} Dado un anillo \(R\),
podemos considerar su anillo de \textbf{polinomios} \(R[x]\) en una
variable \(x\), cuyos elementos \(p(x)\in R[x]\) son de la forma
\[p(x)=a_nx^n+\cdots+a_1x+a_0\] con \emph{coeficientes} \(a_i\in R\),
\(1\leq i\leq n\). En ocasiones los denotaremos como si fueran series
\[p(x)=\sum_{n\geq 0}a_nx^n\] dando por supuesto que \emph{casi todos}
los coeficientes son \(0\). Esto facilita la definición de la suma y la
multiplicación
\[\sum_{n\geq 0}a_nx^n+\sum_{n\geq 0}b_nx^n=\sum_{n\geq 0}(a_n+b_n)x^n,\]
\[\left(\sum_{i\geq 0}a_ix^i\right)\left(\sum_{j\geq 0}b_jx^j\right)=\sum_{n\geq 0}\left(\sum_{i+j=n}a_ib_j\right)x^n.\]
Los anillos de polinomios en varias variables se definen inductivamente
\[R[x_1,\dots, x_{n-1},x_n]=(R[x_1,\dots, x_{n-1}])[x_n].\]
\End{example}

\Begin{example}

También podemos considerar el anillo de \textbf{series formales}
\(R[[x]]\) en una variable \(x\) con coeficientes en un anillo \(R\).
Sus elementos son de la forma \[\sum_{i\geq 0}a_ix^i\in R[[x]],\] sin
restricción sobre los coeficientes \(a_i\in R\). La suma y el producto
se defininen como antes. Los anillos de series formales en varias
variables también se definen inductivamente
\[R[[x_1,\dots, x_{n-1},x_n]]=(R[[x_1,\dots, x_{n-1}]])[[x_n]].\]
\End{example}

\hypertarget{definiciuxf3n}{%
\subsection{Definición}\label{definiciuxf3n}}

\Begin{definition}

Un \textbf{anillo} es un conjunto \(R\) equipado con dos aplicaciones,
llamadas \emph{suma} y \emph{multiplicación} o \emph{producto}, \[
\begin{array}{ccc}
R\times R\rightarrow R, &\qquad& R\times R\rightarrow R,\cr
(a,b)\mapsto a+b;&&(a,b) \mapsto ab.
\end{array}
\] que satisfacen las siguientes propiedades:

\begin{itemize}
\item
  Asociativa: \[
  \begin{array}{rcl}
  a+(b+c)&=&(a+b)+c,\cr 
  a(bc)&=&(ab)c.
  \end{array}
  \]
\item
  Conmutativa: \[
  \begin{array}{rcl}
  a+b&=&b+a,\cr 
  ab&=&ba.
  \end{array}
  \]
\item
  Distributiva: \[a(b+c)=ab+ac.\]
\item
  Existencia de elementos neutros \(0,1\in R\) para la suma y el
  producto: \[
  \begin{array}{rcl}
  0+a&=&a,\cr 
  1a&=&a.
  \end{array}
  \]
\item
  Existencia de un elemento opuesto para la suma \(-a\in R\) para todo
  \(a\in R\) de modo que \[a+(-a)=0.\]
\end{itemize}

\End{definition}

\Begin{remark}

La suma de un anillo lo dota de estructura de grupo abeliano. Los
elementos neutros son únicos, no puede haber dos distintos que
satisfagan la misma propiedad. Los opuestos para la suma también. Restar
es sumar el elemento opuesto \(a-b=a+(-b)\). Multiplicar por cero
siempre da cero, \(0a=0\), y además \(a(-b)=-ab\). La conmutatividad de
la multiplicación no suele exigirse en la definición de anillo, pero
nosotros la hemos incluido porque todos los anillos que veremos la
satisfacen. Otros, como el anillo \(M_{2\times 2}(\mathbb R)\) de
matrices \(2\times 2\) sobre los números reales, no la cumplen.
\End{remark}

\Begin{example}\textrm{\normalfont (El anillo trivial)} El conjunto
unitario \(R=\{0\}\), dotado de las únicas operaciones suma y
multiplicación posibles, es un anillo. Aquí obviamente \(1=0\).
\End{example}

\Begin{proposition}

En un anillo \(R\), \(1=0\) si y solo si \(R=\{0\}\).
\End{proposition}

\Begin{proof}

\(\Leftarrow\) Obvio.

\(\Rightarrow\) Dado \(a\in R\), \(a=1a=0a=0\).

\End{proof}

\Begin{example}\textrm{\normalfont (Anillos de Boole)} Dado un conjunto
\(X\), el conjunto \(\mathcal P(X)=\{A|A \subset X\}\) formado por los
subconjuntos de \(X\) es un anillo, denominado \textbf{anillo de Boole},
donde la suma es la \emph{diferencia simétrica},
\[A+B=(A\cup B)\setminus (A\cap B)\]

\begin{figure}
\centering
\includegraphics{static/images/symmetric_difference.png}
\caption{Diferencia simétrica}
\end{figure}

y el producto es la intersección, \[AB=A\cap B.\] ¿Cuál es el \(0\)? ¿Y
el \(1\)? ¿Y \(-A\)? ¿Y \(A^2\)? Dibuja \(A+B+C\) para tres conjuntos en
posición general. \End{example}

\Begin{example}\textrm{\normalfont (Anillo producto)} Dados dos anillos
\(R\) y \(S\), el producto cartesiano \(R\times S\) es un anillo con las
operaciones siguientes: \[
\begin{array}{rcl}
(r,s)+(r',s')&=&(r+r',s+s'),\cr
(r,s)(r',s')&=&(rr',ss').
\end{array}
\] El cero es \((0,0)\) y el uno es \((1,1)\). \End{example}

\Begin{definition}

Un subconjunto \(S\subset R\) de un anillo \(R\) es un
\textbf{subanillo} si:

\begin{itemize}
\item
  \(1\in S\).
\item
  \(a+b\in S\) para todo \(a,b\in S\).
\item
  \(-a\in S\) para todo \(a\in S\).
\item
  \(ab\in S\) para todo \(a,b\in S\).
\end{itemize}

\End{definition}

\Begin{remark}

Un subanillo \(S\subset R\) es un anillo por derecho propio con la suma
y la multiplicación heredadas de \(S\). Observa que \(0\in S\). También
es un subgrupo para la suma. Ejemplos de subanillos son
\(\mathbb Z\subset \mathbb Q\subset \mathbb R\subset \mathbb C\) y
\(R\subset R[x]\subset R[[x]]\). \End{remark}

\Begin{example}\textrm{\normalfont (Series convergentes)} Si
\(R=\mathbb R\) o \(\mathbb C\), podemos considerar el subanillo de
\textbf{series convergentes} \(R\{x\}\subset R[[x]]\). De hecho
\(R[x]\subset R\{x\}\subset R[[x]]\). \End{example}

\Begin{example}\textrm{\normalfont (Anillos de funciones)} Si \(R\) es
un anillo y \(X\) es un conjunto, podemos considerar el anillo \(R^X\)
cuyos elementos son las aplicaciones \(X\rightarrow R\). La suma y el
producto de aplicaciones \(f,g\colon X\rightarrow R\) se define punto a
punto, \(x\in X\),
\[\begin{array}{rcl}(f+g)(x)&=&f(x)+g(x),\cr (f\cdot g)(x)&=&f(x)g(x).\end{array}\]
Si \(R=\mathbb R\) o \(\mathbb C\) y \(X\) es un espacio topológico,
tenemos el subanillo \(\mathcal C(X)\subset R^X\) de funciones
continuas. \End{example}

\Begin{exercise}

Dados dos anillos \(R\) y \(S\), ¿es \(R\times\{0\}\subset R\times S\)
un subanillo? ¿Y \(\{0\}\times S\subset R\times S\)? \End{exercise}

\Begin{definition}

Una \textbf{unidad} \(u\in R\) es un elemento de un anillo tal que
existe \(u^{-1}\in R\), denominado \textbf{inverso} de \(u\), de modo
que \(uu^{-1}=1\). Un \textbf{cuerpo} es un anillo no trivial donde
todos los elementos no nulos son unidades. \End{definition}

\Begin{remark}

El elemento inverso \(u^{-1}\) de una unidad \(u\) es único. Si \(u\) es
una unidad entonces \(u^{-1}\) también y \((u^{-1})^{-1}=u\). Dividir
por una unidad es multiplicar por el elemento inverso
\(\frac{a}{u}=au^{-1}\). Los elementos \(1\) y \(-1\) son siempre
unidades (no necesariamente distintas) cuyos inversos son ellos mismos.
El subconjunto \(R^{\times}\subset R\) formado por las unidades de un
anillo \(R\) es un grupo con la multiplicación. \End{remark}

\Begin{exercise}

Intenta calcular las unidades de los ejemplos de anillos vistos hasta
ahora. ¿Puede el cero ser una unidad? \End{exercise}

\hypertarget{homomorfismos}{%
\subsection{Homomorfismos}\label{homomorfismos}}

Los homomorfismos de anillos son aplicaciones entre anillos que
preservan la estructura, es decir, la suma, la multiplicación, el \(0\)
y el \(1\).

\Begin{definition}

Dados dos anillos \(R\) y \(S\), un \textbf{homomorfismo}
\(f\colon R\rightarrow S\) es una aplicación tal que, para todo
\(a,b\in R\),
\[\begin{array}{rcl} f(a+b)&=&f(a)+f(b),\cr f(ab)&=&f(a)f(b),\cr f(0)&=&0,\cr f(1)&=&1.\end{array}\]
Un \textbf{isomorfismo} es un homomorfismo biyectivo. Un
\textbf{automorfismo} es un isomorfismo de un anillo en sí mismo.
\End{definition}

\Begin{remark}

Los homomorfismos satisfacen \(f(-a)=-f(a)\). Es más, si \(u\) es una
unidad entonces \(f(u)\) también y \(f(u^{-1})=f(u)^{-1}\). La identidad
\(\operatorname{id}_R\colon R\rightarrow R\) es un isomorfismo.
Comprueba que si
\[R\stackrel{f}\longrightarrow S\stackrel{g}\longrightarrow T\] son
homomorfismos entonces la composición \(g\circ f\colon R\rightarrow T\)
también. Lo mismo es cierto para isomorfismos. Es más, demuestra que si
\(f\colon R\rightarrow S\) es un isomorfismo entonces su aplicación
inversa \(f^{-1}\colon S\rightarrow R\) también. El símbolo \(\cong\) se
usará para denotar la relación de ser isomorfos \(R\cong S\). Prueba que
esta relación es de equivalencia. \End{remark}

\Begin{example}\textrm{\normalfont (La inclusión)} Si \(R\) es un anillo
y \(S\subset R\) es un subanillo, la \textbf{inclusión}
\(i\colon S\hookrightarrow R\), \(i(a)=a\), es un homomorfismo. ¿Qué
diferencia a la inclusión de la identidad? \End{example}

\Begin{example}\textrm{\normalfont (Las proyecciones)} Dados dos anillos
\(R\) y \(S\), la proyección sobre la primera coordenada
\(p_1\colon R\times S\rightarrow R\), \(p_1(r,s)=r\), es un
homomorfismo. También lo es la proyección sobre la segunda coordenada
\(p_2\colon R\times S\rightarrow S\), \(p_2(r,s)=s\). \End{example}

\Begin{proposition}

Dado un homomorfismo \(f\colon R\rightarrow S\), su imagen
\(\operatorname{im} f\subset S\) es un subanillo. \End{proposition}

\Begin{proof}

\begin{itemize}
\item
  \(1=f(1)\in \operatorname{im} f\).
\item
  \(f(a)+f(b)=f(a+b)\in \operatorname{im} f\) para \(a,b\in R\).
\item
  \(-f(a)=f(-a)\in \operatorname{im} f\) para todo \(a\in R\).
\item
  \(f(a)f(b)=f(ab)\in \operatorname{im} f\) para todo \(a,b\in R\).
\end{itemize}

\End{proof}

\Begin{proposition}\label{factorimage} Dado un homomorfismo
\(f\colon R\rightarrow S\) y un subanillo \(U\subset S\), si
\(\operatorname{im} f\subset U\) entonces \(f\) factoriza de manera
única a través de la inclusión, es decir, existe un único homomorfismo
\(g\colon R\rightarrow U\) tal que \(f=i\circ g\),
\[f\colon R\stackrel{g}\rightarrow U\stackrel{i}\hookrightarrow S.\]
\End{proposition}

\Begin{proof}

Si \(f=i\circ g\) entonces tendríamos
\[f(a)=(i\circ g)(a)=i(g(a))=g(a).\] La unicidad de \(g\) sería
consecuencia de esta fórmula ya que fuerza su definición. Definimos pues
la aplicación \(g\colon R\rightarrow U\) como \(g(a)=f(a)\). Tiene
sentido porque \(\operatorname{im}f\subset U\). La aplicación \(g\) es
un homomorfismo pues está definida por la misma fórmula que el
homomorfismo \(f\).\\
\End{proof}

\Begin{remark}

En la proposición anterior podemos siempre tomar
\(U=\operatorname{im} f\). \End{remark}

\Begin{example}\textrm{\normalfont (La evaluación)} Dado un anillo \(R\)
y \(a\in R\) está definido el homomorfismo de \textbf{evaluación}
\(ev_a\colon R[x]\rightarrow R\) como \(ev_a(p(x))=p(a)\). Define
análogamente homomorfismos de evaluación en los anillos de polinomios en
varias variables \(R[x_1,\dots, x_n]\), de series convergentes
\(R\{x\}\) y de funciones \(R^X\) y \(\mathcal C(X)\) definidos
arriba. \End{example}

Los anillos de polinomios satisfacen una propiedad universal relacionada
con los homomorfismos de evaluación.

\Begin{theorem}\textrm{\normalfont (Principio de sustitución)} Dado un
homomorfismo de anillos \(f\colon R\rightarrow S\) y un elemento
\(c\in S\) existe un único homomorfismo \(g\colon R[x]\rightarrow S\)
tal que la restricción de \(g\) a \(R\) es \(f\) y \(g(x)=c\).
\End{theorem}

\Begin{proof}

Dado \(p(x)=a_nx^n+\cdots+ a_1x+ a_0\in R[x]\), si tal
\(g\colon R[x]\rightarrow S\) existiera, \[
\begin{array}{rl}
g(p(x))&=g(a_nx^n+\cdots+a_1x+ a_0)\cr
&=g(a_n)g(x)^n+\cdots+ g(a_1)g(x)+g(a_0)\cr 
&=f(a_n)c^n+\cdots+ f(a_1)c+f(a_0).
\end{array}
\] Definimos pues \[
g(p(x))=f(a_n)c^n+\cdots+ f(a_0).
\] Es tedioso pero trivial comprobar \(g\) así definido es un
homomorfismo. El cálculo anterior demuestra la unicidad. \End{proof}

\Begin{corollary}

Dado un homomorfismo de anillos \(f\colon R\rightarrow S\) y elementos
\(c_1,\dots, c_n\in S\) existe un único homomorfismo
\(g\colon R[x_1,\dots,x_n]\rightarrow S\) tal que la restricción de
\(g\) a \(R\) es \(f\) y \(g(x_i)=c_i\), \(1\leq i\leq n\).
\End{corollary}

\Begin{proof}

Por inducción en \(n\).

Para \(n=1\), la existencia y unicidad de
\(g\colon R[x_1]\rightarrow S\) es consecuencia del teorema anterior.

Veamos que \(n-1\Rightarrow n\). Suponiendo que hay un único
homomorfismo \(h\colon R[x_1,\dots,x_{n-1}]\rightarrow S\) que se
restringe a \(f\colon R\rightarrow S\) sobre las constantes y satisface
\(g(x_i)=c_i\), \(1\leq i\leq n-1\), aplicamos el teorema anterior a
\[R[x_1,\dots,x_n]=(R[x_1,\dots,x_{n-1}])[x_n]\] y obtenemos el
homomorfismo buscado.\\
\End{proof}

El anillo de los enteros cumple la siguiente curiosa propiedad, que en
términos categóricos se denomina ser \emph{inicial} en la categoría de
los anillos.

\Begin{proposition}

Para todo anillo \(R\) existe un único homorfismo
\(f\colon \mathbb Z\rightarrow R\). \End{proposition}

\Begin{proof}

Cualquier homomorfismo \(f\colon \mathbb Z\rightarrow R\) satisface
\(f(0)=0\) y \(f(1)=1\). Por tanto, si \(n>0\) en \(\mathbb Z\),
\[\begin{array}{rcl}
f(n)&=&f(1+\stackrel{n}{\cdots}+1)\cr&=&f(1)+\stackrel{n}{\cdots}+f(1)\cr&=&1+\stackrel{n}{\cdots}+1,\cr
f(-n)&=&-f(n).
\end{array}\] Es fácil comprobar que estas fórmulas definien un
homomorfismo, que ha de ser único.\\
\End{proof}

\hypertarget{ideales}{%
\subsection{Ideales}\label{ideales}}

Existe otro tipo destacado de subconjunto de un anillo que juega un
papel más importante que los subanillos.

\Begin{definition}

Dado un anillo \(R\), un \textbf{ideal} \(I\subset R\) es un subconjunto
tal que:

\begin{itemize}
\item
  \(0\in I\).
\item
  \(a+b\in I\) para todo \(a,b\in I\).
\item
  \(-a\in I\) para todo \(a\in I\).
\item
  \(ra\in I\) para todo \(r\in R\) y \(a\in I\).
\end{itemize}

\End{definition}

En \(\mathbb Z\) los números pares forman un ideal.

\Begin{remark}

Un ideal \(I\subset R\) es un subgrupo para la suma. Si
\(a_1,\dots,a_n\in I\) y \(r_1,\dots, r_n\in R\) entonces la
\textbf{combinación lineal} \(r_1a_1+\cdots+r_na_n\in I\). Todo anillo
posee al menos el ideal \textbf{total} \(R\subset R\) y el
\textbf{trivial} \(\{0\}\subset R\). Además, si \(R\) no es el anillo
trivial, el ideal total es distinto del trivial. \End{remark}

\Begin{proposition}

El \textbf{núcleo} de un homomorfismo \(f\colon R\rightarrow S\),
\[\ker f=\{a\in R\;|\;f(a)=0\},\] es un ideal \(\ker f\subset R\).
\End{proposition}

\Begin{proof}

\begin{itemize}
\item
  \(0\in\ker f\) pues \(f(0)=0\).
\item
  Si \(a,b\in\ker f\) entonces \(a+b\in \ker f\) puesto que
  \(f(a+b)=f(a)+f(b)=0+0=0\).
\item
  Si \(a\in\ker f\) entonces \(-a\in \ker f\) puesto que
  \(f(-a)=-f(a)=0\).
\item
  Si \(a\in\ker f\) y \(r\in R\) entonces \(ra\in \ker f\) pues
  \(f(ra)=f( r )f(a)=f( r )0=0\).
\end{itemize}

\End{proof}

Por tanto, en \(R[x]\), los polinomios \(f(x)\) tales que \(f(1)=0\)
forman un ideal pues constituyen el núcleo del homomorfismo de
evaluación en \(1\in R\). De hecho podríamos evaluar en cualquier otro
elemento de \(R\). También podríamos reemplazar \(R[x]\) por otro de los
anillos de funciones antes vistos.

\Begin{remark}

Como ocurre con los grupos, un homomorfismo de anillos
\(f\colon R\rightarrow S\) es inyectivo si y solo si \(\ker f=\{0\}\).
De otro modo, la inyectividad de \(f\) equivale a que si \(a\in R\) es
tal que \(f(a)=0\) entonces \(a=0\). \End{remark}

\Begin{exercise}

Dados dos anillos \(R\) y \(S\), ¿es \(R\times\{0\}\subset R\times S\)
un ideal? ¿Y \(\{0\}\times S\subset R\times S\)? \End{exercise}

Definimos ahora el ideal generado por un conjunto de elementos de un
anillo, que es el menor ideal que los contiene.

\Begin{definition}\label{generators} El \textbf{ideal generado por} un
conjunto finto de elementos \(a_1,\dots,a_n\in R\) está formado por
todas las combinaciones lineales de los generadores con coeficientes en
el anillo:
\[(a_1,\dots,a_n)=\{r_1a_1+\dots+r_na_n\;|\; r_1,\dots,r_n\in R\}\subset R.\]
Un \textbf{ideal principal} es uno que está generado por un único
elemento \((a)=\{ra\,|\, r\in R\}\) y que por tanto está formado por
sus múltiplos. \End{definition}

En \(\mathbb Z\), el ideal de los números pares es \((2)\).

\Begin{exercise}

Comprueba que \((a_1,\dots,a_n)\) es en efecto un ideal. Observa que
\(a_1,\dots,a_n\in (a_1,\dots, a_n)\). Es más, demuestra que si
\(I\subset R\) es un ideal y \(a_1,\dots,a_n\in I\) entonces
\((a_1,\dots,a_n)\subset I\). Intenta dar una definición razonable de
ideal generado por un conjunto infinito de elementos que satisfaga las
propiedades análogas al caso finito. \End{exercise}

\Begin{proposition}

Todos los ideales de \(\mathbb Z\) son principales. Es más, todo ideal
no nulo de \(\mathbb Z\) está generado por cualquiera de sus elementos
de valor absoluto mínimo. \End{proposition}

\Begin{proof}

Sea \(I\subset \mathbb Z\) un ideal. Si \(I=\{0\}=(0)\) ya es
principal. Si no, tomamos \(a\in I\), \(a\neq 0\), de valor absoluto
mínimo. Veamos que \(I=(a)\).

Por un lado \((a)\subset I\) pues \(a\in I\).

Por otro, dado \(b\in I\) realizamos la división euclídea de \(b\) por
\(a\), \[b=ca+r.\] El resto satisface \(|r|<|a|\). Además
\(r=b-ca\in I\), por tanto \(r=0\) y \(b=ca\in (a)\).\\
\End{proof}

La demostración de la proposición anterior solo usa la noción de
división euclídea, por tanto es válida no solo para \(\mathbb Z\) sino
para cualquier \emph{dominio euclídeo} (noción conocida que repasaremos
más adelante). La siguiente proposición destaca otro caso particular de
interés.

\Begin{proposition}

Dado un cuerpo \(k\), todos los ideales de \(k[x]\) son principales. Es
más, todo ideal no nulo de \(k[x]\) está generado por cualquiera de sus
elementos de grado mínimo. \End{proposition}

Veamos ahora la relación entre ideales y unidades.

\Begin{proposition}

Un ideal \(I\subset R\) contiene una unidad \(\Leftrightarrow\) \(I=R\).
\End{proposition}

\Begin{proof}

\(\Leftarrow\) Obvio porque \(1\in R=I\) es una unidad.

\(\Rightarrow\) Si \(u\in I\subset R\) es una unidad, \(u^{-1}\in R\) y
por ser \(I\) un ideal \(1=uu^{-1}\in I.\)

Si \(1\in I\) y \(a\in R\) entonces \(1a=a\in I\), por tanto
\(R\subset I\), así que \(I=R\).\\
\End{proof}

\Begin{corollary}

Un anillo es un cuerpo \(\Leftrightarrow\) tiene solo dos ideales
(necesariamente el total y el trivial). \End{corollary}

\Begin{proof}

\(\Rightarrow\) Sea \(k\) un cuerpo. Los cuerpos, en tanto que anillos
no triviales, tienen al menos dos ideales: el trivial y el total. Si
\(I\subset k\) es un ideal no trivial entonces existe un elemento
\(a\in I\subset k\) no nulo. Como \(k\) es un cuerpo este elemento ha de
ser una unidad, así que \(I=k\).

\(\Leftarrow\) Sea \(R\) un anillo cuyos dos únicos ideales son
\(\{0\}\) y \(R\). En particular \(R\) no es trivial. Sea \(a\in R\)
un elemento no trivial. Como \(a\in (a)\), este ideal no puede ser el
trivial, así que ha de ser el total \((a)=R\). Al ser \(1\in R=(a)\) ha
de existir un elemento \(r\in R\) tal que \(ra=1\), así que \(a\) es una
unidad.\\
\End{proof}

\Begin{corollary}

Si \(f\colon k\rightarrow R\) es un homomorfismo de anillos donde \(k\)
es un cuerpo y \(R\) no es trivial entonces \(f\) es inyectivo.
\End{corollary}

\Begin{proof}

El ser \(f\colon k\rightarrow R\) un homomorfismo, \(f(1)=1\). Como
\(R\) no es trivial, \(1\neq 0\) luego \(1\notin\ker f\subset k\) no
puede ser el total. Por ser \(k\) es un cuerpo la única opción posible
es \(\ker f=\{0\}\), luego \(f\) es inyectivo.\\
\End{proof}

\hypertarget{cocientes}{%
\subsection{Cocientes}\label{cocientes}}

\Begin{definition}

Dado un anillo \(R\) y un ideal \(I\subset R\), el \textbf{anillo
cociente} \(R/I\) es el cociente de los grupos abelianos subyacentes
dotado de la multiplicación \[(a+I)(b+I)=(ab)+I.\] \End{definition}

\Begin{remark}

Recordemos que \(R/I=\{a+I\,|\, a\in R\}\) de modo que \(a+I=b+I\) si
y solo si \(a-b\in I\). En particular \(a+I=0+I\) si y solo si
\(a\in I\). El elemento \(a+I\) del cociente se denomina \textbf{clase}
de \(a\) \textbf{módulo} \(I\). Cuando el ideal \(I\) se sobreentiende
se escribe simplemente \[a+I=\bar a=[a].\] La suma en el cociente se
define como \((a+I)+(b+I)=(a+b)+I\). El cero y el uno en el cociente son
\(0+I\) y \(1+I\). Comprueba que \(R/R\) es el anillo trivial y
\(R/(0)\cong R\). Los cocientes \(\mathbb Z/(n)\) son bien conocidos,
\(\bar a\in\mathbb Z/(n)\) es una unidad si y solo si
\(\operatorname{mcd}(a,n)=1\), luego \(\mathbb Z/(n)\) es un cuerpo si y
solo si \(n\) es primo. \End{remark}

\Begin{theorem}

El anillo cociente \(R/I\) está bien definido. Su estructura es la única
que hace que la \textbf{proyección natural}
\(p\colon R\twoheadrightarrow R/I\), \(p(a)=a+I\), sea un homomorfismo.
El núcleo de esta proyección es \(\ker p=I\). \End{theorem}

\Begin{proof}

Para ver que la multiplicación está bien definida hay que comprobar que
\[\left.\begin{array}{r}a+I=a'+I\cr b+I=b'+I\end{array}\right\}\Rightarrow(ab)+I=(a'b')+I.\]
Esto equivale a
\[\left.\begin{array}{r}a-a'\in I\cr b-b'\in I\end{array}\right\}\Rightarrow ab-a'b'= (a-a')b+a'(b-b')\in I.\]
Las propiedades que el producto y la suma deben satisfacer se cumplen
obviamente pues se derivan de las correspondientes propiedades de \(R\).

Probemos la unicidad. Si \(p\colon R\rightarrow R/I\) es un homomorfismo
entonces \[\begin{array}{rcl}
p(a+b)&=&p(a)+p(b),\cr p(ab)&=&p(a)p(b),
\end{array}\] lo cual equivale a \[\begin{array}{rcl}
(a+b)+I&=&(a+I)+(b+I),\cr (ab)+I&=&(a+I)(b+I).
\end{array}\]

El núcleo de la proyección natural es
\[\ker p =\{a\in R\;|\; p(a)=0\},\] pero \(p(a)=a+I\) y \(a+I=0+I\) si
y solo si \(a\in I\), luego \(\ker p=I\).\\
\End{proof}

\Begin{proposition}\label{factorquotient} Dado un ideal \(I\subset R\) y
un homomorfismo \(f\colon R\rightarrow S\) tal que \(I\subset \ker f\),
\(f\) factoriza de manera única a través de la proyección natural, es
decir existe un único homomorfismo \(g\colon R/I\rightarrow S\) tal que
\(f=g\circ p\),
\[f\colon R\stackrel{p}\twoheadrightarrow R/I\stackrel{g}\rightarrow S.\]
\End{proposition}

\Begin{proof}

Si \(f=g\circ p\) entonces tendríamos
\[f(a)=(g\circ p)(a)=g(p(a))=g(a+I).\] Definimos la aplicación
\(g\colon R/I\rightarrow S\) como \[g(a+I)=f(a).\] Veamos que en efecto
está bien definida. La unicidad se seguirá de la primera fórmula.

Si \(a+I=a'+I\) entonces \(a-a'\in I\subset\ker f\) luego
\[0=f(a-a')=f(a)-f(a').\] Por tanto \[g(a+I)=f(a)=f(a')=g(a'+I).\]
Claramente \(g\) es un homomorfismo pues se definie como el homomorfismo
\(f\) en los representantes.\\
\End{proof}

\Begin{remark}

En la proposición anterior podemos tomar siempre \(I=\ker f\).
\End{remark}

\Begin{theorem}\textrm{\normalfont (Primer Teorema de Isomorfía)}\label{primer}
Dado un homomorfismo \(f\colon R\rightarrow S\), existe un único
homomorfismo \(\bar f\colon R/\ker f\rightarrow \operatorname{im}f\) tal
que \(f\) factoriza como \(f=i\circ\bar f\circ p\), es decir, \(f\)
encaja en el siguente \textbf{diagrama conmutativo},

\begin{figure}
\centering
\includegraphics{static/images/isomorfianillos.png}
\caption{Primer teorema de isomorfía}
\end{figure}

Aquí \(p\) es la proyección natural e \(i\) es la inclusión. Además
\(\bar f\) es un isomorfismo. \End{theorem}

\Begin{proof}

\protect\hyperlink{factorimage}{Por un lado} podemos factorizar
\(f\colon R\rightarrow S\) de manera única como \(f=i\circ g\),
\[f\colon R\stackrel{g}\rightarrow \operatorname{im} f\stackrel{i}\hookrightarrow S,\]
donde \(g(a)=f(a)\). En particular \[\ker g = \ker f.\]

\protect\hyperlink{factorquotient}{Por otro lado} podemos factorizar
\(g\colon R\rightarrow \operatorname{im} f\) de manera única como
\(g=\bar f\circ p\),
\[g\colon R\stackrel{p}\twoheadrightarrow R/\ker f\stackrel{\overline{f}}\rightarrow \operatorname{im} f,\]
donde \(\bar f(\bar{a})=g(a)=f(a)\).

Por tanto \(f=i\circ g= i\circ(\overline{f}\circ i)\), como queríamos.
La unicidad de \(\bar f\) se deduce de esta fórmula, ya que fuerza su
definición: \[
\begin{array}{rcl}
f(a)&=&(i\circ\bar f\circ p)(a)\cr
&=&i(\bar{f}(p(a)))\cr
&=&i(\bar{f}(\bar{a}))\cr
&=&\bar{f}(\bar{a}).
\end{array}
\]

Veamos que \(\bar f\colon R/\ker f\rightarrow \operatorname{im} f\) es
un isomorfismo. Comenzamos probando que es inyectivo. Sea
\(\bar{a}\in R/\ker f\) tal que \(\bar f(\bar{a})=0\). Como
\(\bar f(\bar{a})=f(a)\), deducimos que \(a\in \ker f\), por lo que
\(\bar{a}=\bar{0}\).

Veamos que \(\bar f\colon R/\ker f\rightarrow \operatorname{im} f\) es
sobreyectiva. Dado \(b\in\operatorname{im} f\) existe \(a\in R\) tal que
\(f(a)=b\). Por tanto \(\bar f(\bar{a})=f(a)=b\).\\
\End{proof}

\Begin{corollary}

\(\mathbb R[x]/(x^2+1)\cong\mathbb C\). \End{corollary}

\Begin{proof}

Consideremos el homomorfismo
\(f\colon \mathbb R[x]\rightarrow\mathbb C\) definido por la inclusión
\(\mathbb R\subset\mathbb C\) y tal que \(f(x)=i\). Este homomorfismo es
sobreyectivo ya que dado \(a+ib\in\mathbb C\), \(f(bx+a)=a+ib\) por
tanto \(\operatorname{im} f =\mathbb C\). Basta ahora ver que
\(\ker f=(x^2+1)\). Como \(\mathbb R\) es un cuerpo, es suficiente
comprobar que \(x^2+1\in\ker f\) pero \(\ker f\) no posee ningún
polinomio no trivial de grado \(<2\). Claramente \(f(x^2+1)=i^2+1=0\).
Si \(bx+a\in\mathbb{R}[x]\) es un polinomio no trivial entonces
\(f(bx+a)=a+ib\) es un número complejo no trivial, con lo que queda
demostrado. \End{proof}

Concluimos con un estudio de los ideales de un anillo cociente.

\Begin{proposition}

Sea \(f\colon R\rightarrow S\) un homomorfismo.

\begin{itemize}
\item
  Si \(J\subset S\) es un ideal entonces \(f^{-1}(J)\subset R\) también
  y además \(\ker f\subset f^{-1}(J)\).
\item
  Si \(I\subset R\) es un ideal y \(f\) es sobreyectivo entonces
  \(f(I)\subset S\) también es un ideal.
\end{itemize}

\End{proposition}

\Begin{proof}

Comenzamos por el primer apartado:

\begin{itemize}
\item
  \(0\in f^{-1}(J)\) porque \(f(0)=0\in J\).
\item
  Si \(a,b\in f^{-1}(J)\) es porque \(f(a),f(b)\in J\), luego
  \(f(a+b)=f(a)+f(b)\in J\) y por tanto \(a+b\in f^{-1}(J)\).
\item
  En el caso anterior también \(f(-a)=-f(a)\in J\), así que
  \(-a\in f^{-1}(J)\).
\item
  Es más, dado \(r\in R\), \(f(ra)=f( r )f(a)\in J\) luego
  \(ra\in f^{-1}(J)\).
\end{itemize}

Además, como \(\{0\}\subset J\),
\(\ker f=f^{-1}(\{0\})\subset f^{-1}(J)\).

En el segundo caso:

\begin{itemize}
\item
  \(0=f(0)\in f(I)\) pues \(0\in I\).
\item
  Si \(a,b\in I\) entonces \(a+b\in I\) luego
  \(f(a)+f(b)=f(a+b)\in f(I)\).
\item
  En el caso anterior también \(-a\in I\) luego \(-f(a)=f(-a)\in f(I)\).
\item
  Es más, dado \(s\in S\), por ser \(f\) sobreyectiva \(s=f( r )\) para
  cierto \(r\in R\), y como \(ra\in I\) entonces
  \(sf(a)=f( r )f(a)=f(ra)\in f(I)\).
\end{itemize}

\End{proof}

\Begin{theorem}\textrm{\normalfont (de correspondencia)} Dado un anillo
\(R\) y un ideal \(I\), si \(p\colon R\twoheadrightarrow R/I\) denota la
proyección natural tenemos la siguiente biyección
\[\begin{array}{rcl}\left\{\text{ideales de }R\text{ que contienen a }I\right\}&\longleftrightarrow& \left\{\text{ideales de }R/I\right\},\cr I'&\mapsto&p(I'),\cr p^{-1}(J)&\leftarrow&J.\end{array}\]
\End{theorem}

\Begin{proof}

La proyección natural es un homomorfismo sobreyectivo con núcleo \(I\),
por tanto las aplicaciones del enunciado están bien definidas por la
proposición anterior. Veamos que una es inversa de la otra. La igualdad
\(p(p^{-1}(J))=J\) es cierta por ser \(p\) sobreyectiva. En general
\(I'\subset p^{-1}(p(I'))\). Veamos que la otra inclusión es también
cierta si \(I\subset I'\). Dado \(a\in p^{-1}(p(I'))\),
\(p(a)\in p(I')\) por tanto existe \(b\in I'\) tal que \(p(b)=p(a)\).
Esto implica que \(p(a-b)=p(a)-p(b)=0\), es decir,
\(a-b\in I\subset I'\), por tanto \(a=b+(a-b)\in I'\). \End{proof}

\hypertarget{auxf1adir-elementos-a-un-anillo}{%
\subsection{Añadir elementos a un
anillo}\label{auxf1adir-elementos-a-un-anillo}}

La siguiente definición nos da una receta para añadir nuevos elementos a
un anillo contenido en otro mayor.

\Begin{definition}

Dado un anillo \(S\), un subanillo \(R\subset S\) y \(s\in S\), el menor
subanillo \(R[s]\subset S\) que contiene a \(R\) y a \(s\) es la imagen
del homomorfismo \(g\colon R[x]\rightarrow S\) definido como la
inclusión \(i\colon R\hookrightarrow S\) sobre \(R\) tal que \(g(x)=s\),
\(R[s]=\operatorname{im} g\). \End{definition}

\Begin{remark}

La propiedad de ser el menor viene dada porque todo elemento de \(R[s]\)
se puede expresar (aunque no de manera única) como
\(a_ns^n+\cdots+a_1s+a_0\) para ciertos \(a_i\in R\). Por tanto, si
\(U\subset S\) es un subanillo tal que \(R\subset U\) y \(s\in U\)
entonces \(R[s]\subset U\). En particular \(\mathbb R[i]=\mathbb C\) y
\(\mathbb Z[i]\subset\mathbb C\) son los enteros de Gauss. \End{remark}

\Begin{exercise}

Da una definición directa del menor subanillo
\(R[s_1,\dots,s_n]\subset S\) que contiene a varios elementos
\(s_i\in S\). \End{exercise}

También podemos añadir nuevos elementos a un anillo \(R\) de manera
abstracta, es decir, sin tener previamente otro anillo mayor. El propio
anillo de polinomios \(R[x]\) consiste en añadirle un nuevo elemento
\(x\) a \(R\) de manera libre. Veamos cómo añadir elementos que
satisfagan ecuaciones.

Dado un polinomio \(p(x)=a_nx^n+\cdots + a_1x+ a_0\in R[x]\),
consideramos el anillo \(S=R[x]/(p(x))\). Por abuso de notación, la
clase de una constante \(a\in R\) en \(S\) se denotará igual,
\(a\in S\), no \(\bar a\). En este nuevo anillo \(\bar x\in S\) es una
raíz de \(p(x)\) puesto que

\[ p(\bar{x})=a_n\bar x^n+\cdots + a_1 \bar x+ a_0=\overline{p(x)}=\bar 0\in S.\]

Este anillo posee en ciertos casos una descripción similar a la de los
números complejos.

Para demostrarlo usaremos el siguiente resultado que asegura que en
\(R[x]\) siempre podemos dividir por un polinomio mónico del modo
habitual.

\Begin{lemma}

Dado un polinomio \textbf{mónico}
\(p(x)=x^n+\cdots + a_1x+ a_0\in R[x]\) y otro polinomio cualquiera
\(f(x)\in R[x]\), existen dos únicos polinomios\\
\(c(x), r(x)\in R[x]\) tales que \(r(x)\) tiene grado \(<n\) y
\(f=c\cdot p+r\). \End{lemma}

\Begin{proof}

Sea \(f_0=f\). Si grado \(f_0<n\) entonces podemos tomar \(c=0\) y
\(r=f_0\). Veamos ahora cómo proceder si grado \(f_0\geq n\). En este
caso existen polinomios \(c_1,f_1\in R[x]\) tales que grado \(f_1<\)
grado \(f_0\) y \(f_0=c_1\cdot p + f_1\). En efecto, si
\(f_0=b_mx^m+\cdots\) tiene grado \(m\geq n\) podemos tomar
\(c_1(x)=b_mx^{m-n}\), que tiene sentido pues estamos suponiendo que
\(m\geq n\). Si el grado de \(f_1\) sigue siendo \(\geq n\), podemos
aplicar el mismo razonamiento a \(f_1\) obteniendo así polinomios
\(c_2,f_2\in R[x]\) tales que grado \(f_2<\) grado \(f_1\) y
\(f_1=c_2\cdot p + f_2\). Podemos continuar \[
\begin{array}{rcl}
f_0&=&c_1\cdot p + f_1,\cr
&\vdots&\cr
f_{i-1}&=&c_i\cdot p + f_i,
\end{array}
\] hasta que grado \(f_i<n\). De este modo
\[f=(c_1+\cdots+c_i)\cdot p+f_i\] y podemos tomar
\(c=c_1+\cdots+c_i\) y \(r=f_i\). Hemos probado la existencia.

Veamos la unicidad de \(r\). Si \(f=c\cdot p+r=c'\cdot p+r'\) en las
condiciones del enunciado, tenemos entonces que \(r-r'=(c'-c)\cdot p\).
Por un lado \(r-r'\) tiene grado \(<n\). Supongamos por reducción al
absurdo que \(c'\neq c\). Entonces \(c'-c=e_kx^k+\cdots\) para cierto
\(k\geq 0\) y \(e_k\in R\) no nulo. Esto implica que
\((c'-c)\cdot p=e_kx^{k+n}+\cdots\) y por tanto tendría grado
\(\geq n\). Hemos llegado a una contradicción, así que \(c=c'\), luego
también \(r=r'\). \End{proof}

\Begin{corollary}\label{uniquerep} Dado un polinomio mónico
\(p(x)=x^n+\cdots + a_1x+ a_0\in R[x]\) de grado \(n\), todo elemento
de \(R[x]/(p)\) posee un único representante de grado \(<n\).
\End{corollary}

\Begin{proof}

En efecto, dado \([f]\in R[x]/(p)\), \(r\in R[x]\) es un representante
de \([f]\) si y solo si \(f-r\in (p)\), lo que equivale a la existencia
de \(c\in R[x]\) tal que \(f-r=c\cdot p\), es decir, \(f=c\cdot p+r\).
Este resultado se deduce por tanto del lema anterior. \End{proof}

\Begin{remark}

El corolario anterior nos dice que, bajo sus condiciones, todo elemento
de \(R[x]/(p)\) se puede escribir de manera única como
\[b_{n-1}\bar{x}^{n-1}+\cdots+ b_1\bar{x}+b_0,\] donde
\(b_0,\dots, b_{n-1}\in R\).

En particular, si \(n\geq 1\), el homomorfismo
\(R\hookrightarrow R[x]/(p)\colon r\mapsto\bar{r}\) que envía cada
constante a la clase del correspondiente polinomio constante es
inyectivo. Por ello, en adelante eliminaremos la barra de las clases de
los polinomios constantes y las denotaremos simplemente \(r\). De este
modo podemos ver \(R\) como un subanillo \(R\subset R[x]/(p)\). Esto
refuerza la idea de que este cociente \emph{añade} elementos a \(R\).
\End{remark}

En adelante, cuando hablemos de añadirle a un anillo \(R\) una raíz
\(\alpha\) de un polinomio \(p(x)\in R[x]\) de manera abstracta nos
estaremos refiriendo al cociente \(R[x]/(p)\) y a \(\alpha=\bar{x}\),
que como hemos visto es una raíz de \(p(x)\) en este anillo. Si \(p\) es
mónico de grado \(n\), todo elemento de \(R[x]/(p)\) se escribe de
manera única como \(b_{n-1}\alpha^{n-1}+\cdots+ b_1\alpha+b_0\), con
\(b_0,\dots, b_{n-1}\in R\).

Es posible añadir a un anillo de manera abstracta no solo uno sino
varios elementos que satisfagan determinadas ecuaciones. Se puede hacer
tanto de manera directa como inductiva. Prueba a hacerlo como ejericio.

\Begin{example}\textrm{\normalfont ($\mathbb Z[x]/(x^3+3x+1)$)} Todo
elemento de este anillo se puede expresar de manera única como
\(a_2 \bar x^2+ a_1 \bar x+ a_0\) para ciertos coeficientes
\(a_0,a_1,a_2\in\mathbb Z\). La suma se calcula sumando coeficientes. El
producto es más complejo porque suele ser necesario reducir el grado del
representante obtenido. Esto se hace usando que \(\bar x\) es una raíz
del denominador. Concretamente en este caso \(\bar x^3+3\bar x+1=0\),
luego
\[\begin{array}{rcl}\bar x^3&=& -3\bar x-1,\cr \bar x^4&=& -3\bar x^2-\bar x,\cr\bar x^5&=& -3\bar x^3-\bar x^2\cr&=& -3(-3\bar x-1)-\bar x^2\cr&=&-\bar x^2+9\bar x+3,\cr\bar x^6&=&\dots\end{array}\]Usamos
esto en el siguiente ejemplo de
cálculo,\[\begin{array}{rcl}(- \bar x^2+ \bar x+ 2)(\bar x+ 1)&=& -\bar x^3+3\bar x+2\cr&=& -(-3\bar x-1)+3\bar x+2\cr&=&6\bar x+3.\end{array}\]
\End{example}

\Begin{example}\textrm{\normalfont ($(\mathbb Z/(4))[x]/(2x^2-1)$)} En
este anillo la posible generalización del corolario anterior es
totalmente falsa. En efecto, aquí \(2=0\) ya que
\(0=2(2\bar x^2-1)=4\bar x^2-2=2\) pues \(4=0\) en \(\mathbb Z/4\).
Además \(\bar x^2\) no se puede expresar como la clase de un polinomio
de grado \({<}2\) porque, si se pudiera, entonces el ideal
\((2x^2-1)\subset (\mathbb Z/(4))[x]\) tendría polinomios mónicos de
grado \(2\), pero no tiene. \End{example}

\hypertarget{dominios-y-fracciones}{%
\subsection{Dominios y fracciones}\label{dominios-y-fracciones}}

\Begin{definition}

Dado un anillo \(R\), un \textbf{divisor de cero} es un elemento
\(a\in R\) no nulo, \(a\neq 0\), tal que existe otro \(b\in R\),
\(b\neq 0\), de modo que \(ab=0\). Un anillo no trivial \(R\) es un
\textbf{dominio (de integridad)} si no posee divisores de cero.
\End{definition}

\Begin{remark}

Dicho de otro modo, \(R\) es un dominio cuando dados \(a,b\in R\) tales
que \(ab=0\) entonces \(a=0\) o \(b=0\). Los dominios poseen la
\textbf{propiedad cancelativa}, es decir, si \(ab=ac\) y \(a\neq 0\)
entonces \(b=c\) ya que esto equivale a \(a(b-c)=0\). Los cuerpos \(k\)
y los enteros \(\mathbb Z\) son dominios. Los subanillos de un dominio
también son dominios. El anillo \(\mathbb Z/(6)\) no es un dominio
porque aquí \(2\cdot 3=6=0\) pero \(2\neq 0\neq 3\). \End{remark}

\Begin{proposition}

Si \(R\) es un dominio entonces \(R[x]\) también. \End{proposition}

\Begin{proof}

Dados dos polinomios no nulos \(p(x)=a_nx^n+\cdots\) y
\(q(x)=b_mx^m+\cdots\) de grados respectivos \(n\) y \(m\)
(\(a_n\neq 0\neq b_m\)), su producto \(p(x)q(x)=a_nb_mx^{n+m}+\cdots\)
es no nulo de grado \(n+m\) ya que \(a_nb_m\neq 0\) por ser \(R\) un
dominio. \End{proof}

\Begin{corollary}

Si \(R\) es un dominio entonces \(R[x_1,\dots, x_n]\) también.
\End{corollary}

Cualquier subanillo de un cuerpo es un dominio. Veamos que,
recíprocamente, todo dominio se puede incluir en un cuerpo.

\Begin{definition}

Dado un dominio \(R\), su \textbf{cuerpo de fracciones} \(Q( R )\) es el
cociente del conjunto
\[\left\{\frac{a}{b}\;\bigg|\; a,b\in R,\,b\neq 0\right\}\] por la
relación de equivalencia
\[\frac{a}{b}\sim\frac{a'}{b'}\Leftrightarrow ab'=a'b\] dotado de las
operaciones
\[\begin{array}{rcl}\displaystyle \frac{a}{b}+\frac{c}{d}&\displaystyle =&\displaystyle  \frac{ad+bc}{bd},\cr\displaystyle \frac{a}{b}\cdot\frac{c}{d}&\displaystyle =&\displaystyle \frac{a c}{b d}.\end{array}\]
\End{definition}

El ejemplo principal es \(Q(\mathbb Z)=\mathbb Q\).

\Begin{proposition}

El cuerpo de fracciones \(Q( R )\) de un dominio \(R\) está bien
definido. La aplicación \(i\colon R\rightarrow Q( R )\),
\(i(a)=\frac{a}{1}\), es un homomorfismo inyectivo. Todo homomorfismo
inyectivo \(f\colon R\rightarrow k\) donde \(k\) es un cuerpo factoriza
de manera única a través de \(i\), es decir, existe un único
homomorfismo \(g\colon Q( R )\rightarrow k\) tal que \(f=g\circ i\),
\[f\colon R\stackrel{i}\rightarrow Q( R )\stackrel{g}\rightarrow k.\]
\End{proposition}

\Begin{proof}

La relación es simétrica y reflexiva porque el producto en \(R\) es
conmutativo. Veamos la transitividad. Si
\[\displaystyle \frac{a}{b}\sim \frac{a'}{b'}\sim \frac{a''}{b''}\]
entonces \[\begin{array}{rcl}
ab'&=&a'b,\cr 
a'b''&=&a'' b'.
\end{array}\] En particular, \[\begin{array}{rcl}
(a b'') b'&=&(ab')b''\cr
&=&(a'b)b''\cr
&=&(a'b'')b\cr
&=&(a'' b')b\cr
&=&(a'' b)b'.
\end{array}\] Por la propiedad cancelativa de los dominios,
\(ab''=a'' b\), es decir \(\frac{a}{b}\sim \frac{a''}{b''}\). Por tanto
el conjunto cociente \(Q( R )\) está bien definido. Demostrar que las
definiciones de la suma y la multiplicación en \(Q(R)\) no dependen de
la elección de fracciones representantes es laborioso pero trivial, no
es distinto de la construcción clásica de los números racionales.
También es fácil probar que los axiomas que definen los anillos se
verifican. Obviamente el cero y el uno de \(Q( R )\) son \(\frac{0}{1}\)
y \(\frac{1}{1}\), respectivamente. Por tanto una fracción
\(\frac{a}{b}\) es nula \(\Leftrightarrow\) \(a=0\). Si \(\frac{a}{b}\)
es no nula entonces es claramente una unidad y
\((\frac{a}{b})^{-1}=\frac{b}{a}\), por lo que \(Q( R )\) es un cuerpo.

Es inmediato ver que \(i\) es un homomorfismo. Es inyectivo porque
\(a\in\ker f\) si y solo si \(\frac{a}{1}=\frac{0}{1}\), lo cual
equivale a \(a=0\).

Dado \(f\colon R\rightarrow k\) como en el enunciado, si existiera una
descomposición \(f=g\circ i\) entonces tendríamos que
\[f(a)=(g\circ i)(a)=g(i(a))=g\left(\frac{a}{1}\right).\] Toda fracción
de \(Q( R )\) se puede descomponer como
\[\frac{a}{b}=\frac{a}{1}\cdot\frac{1}{b}=\frac{a}{1}\left(\frac{b}{1}\right)^{-1},\]
por tanto tendríamos que
\[g\left(\frac{a}{b}\right)=g\left(\frac{a}{1}\left(\frac{b}{1}\right)^{-1}\right)=g\left(\frac{a}{1}\right)g\left(\frac{b}{1}\right)^{-1}=f(a)f(b)^{-1}.\]
Esto demuestra que, caso de existir, \(g\) sería único.

Ahora basta definir \(g\colon Q( R )\rightarrow k\) como
\(g\left(\frac{a}{b}\right)=f(a)f(b)^{-1}\). Esta definición tiene
sentido porque, como \(b\neq 0\) y \(f\) es inyectivo, \(f(b)\neq 0\), y
al ser \(k\) un cuerpo todo elemento no nulo tiene inverso, luego
\(f(b)^{-1}\) existe. Con esta definición es un ejercicio comprobar que
\(g\) es un homomorfismo.\\
\End{proof}

\Begin{corollary}

Dado un homomorfismo inyectivo entre dominios
\(f\colon R\rightarrow S\), existe un único homomorfismo entre sus
cuerpos de fracciones \(g\colon Q( R )\rightarrow Q(S)\) que extiende
\(f\), en el sentido de que el siguiente cuadrado es conmutativo

\begin{figure}
\centering
\includegraphics{static/images/fractionfield.png}
\caption{Cuerpos de fracciones}
\end{figure}

es decir, \(g\circ i_R=i_S\circ f\), donde \(i_R\) e \(i_S\) son las
inclusiones de \(R\) y \(S\) en sus cuerpos de fracciones.
\End{corollary}

\Begin{proof}

Basta aplicar la proposición anterior a
\(i_S\circ f\colon R\rightarrow Q(S)\), que es inyectivo por ser
composición de inyectivos. El homomorfismo \(g\) estará por tanto
definido como \(g\left(\frac{a}{b}\right)=\frac{g(a)}{g(b)}\).
\End{proof}

\Begin{definition}

Dado un cuerpo \(k\), el \textbf{cuerpo de funciones racionales} en una
variable se define como \(k(x)=Q(k[x])\). \End{definition}

\Begin{exercise}

Da dos definiciones del cuerpo de funciones racionales en varias
variables \(k(x_1,\dots,x_n)\), una inductiva y otra directa, que sean
aparentemente distintas pero isomorfas. \End{exercise}

\Begin{definition}

Los ideales distintos del total se denominan \textbf{propios}. Un ideal
\(I\subsetneq R\) es \textbf{primo} si dados \(a,b\in R\) tales que
\(ab\in I\) entonces \(a\in I\) o \(b\in I\). \End{definition}

\Begin{remark}

Un ideal \(I\subset R\) es propio si y solo si \(R/I\) no es trivial. Si
\(p\in\mathbb Z\) es un primo entonces el ideal \((p)\subset \mathbb Z\)
es primo ya que si \(ab\in (p)\) es porque \(p\) divide a \(ab\), luego
\(p\) ha de dividir a \(a\) o a \(b\), es decir \(a\in(p)\) o
\(b\in(p)\). En general \((0)\subset R\) es primo si y solo si \(R\) es
un dominio. \End{remark}

\Begin{proposition}

Un ideal \(I\subset R\) es primo \(\Leftrightarrow\) \(R/I\) es un
dominio. \End{proposition}

\Begin{proof}

Ser un ideal propio se corresponde con tener un cociente no trivial.
Veamos la equivalencia del resto de propiedades.

\(\Rightarrow\) Dados \(\bar a,\bar b\in R/I\), si
\(\bar a\bar b =\overline{ab}=\bar 0\) entonces \(ab\in I\), luego
\(a\in I\) o \(b\in I\), es decir \(\bar a=\bar 0\) o \(\bar b=\bar 0\).

\(\Leftarrow\) Dados \(a,b\in R\), si \(ab\in I\) entonces
\(\bar a\bar b=\overline{ab}=\bar 0\), luego \(\bar a=\bar 0\) o
\(\bar b=\bar 0\), es decir \(a\in I\) o \(b\in I\).

\End{proof}

\Begin{definition}

Un ideal \(I\subsetneq R\) es \textbf{maximal} si los únicos ideales que
lo contienen son el total \(R\) y el propio \(I\). \End{definition}

\Begin{remark}

De otro modo, no puede existir ningún ideal \(J\) tal que
\(I\subsetneq J\subsetneq R\). Todo anillo no trivial posee al menos un
ideal maximal. ¿Cuál es en el caso de los cuerpos? \End{remark}

\Begin{proposition}

Un ideal \(I\subset R\) es maximal \(\Leftrightarrow\) \(R/I\) es un
cuerpo. \End{proposition}

\Begin{proof}

Recordemos que un cuerpo es un anillo no trivial con dos ideales. El
anillo \(R/I\) es no trivial si y solo si \(I\subsetneq R\), que es la
primera condición de maximalidad. Es más \(R/I\) tiene solo dos ideales
si y solo si solo hay dos ideales de \(R\) que contengan a \(I\)
(necesariamente el total y el propio \(I\)). Esta es la segunda
condición de maximalidad.\\
\End{proof}

\Begin{corollary}

Todo ideal maximal es primo. \End{corollary}

\Begin{definition}

Un \textbf{dominio de ideales principales} (también \textbf{DIP} o
\textbf{PID}) es un dominio donde todos los ideales son principales.
\End{definition}

Son dominios de ideales principales \(\mathbb Z\) y \(k[x]\) si \(k\) es
un cuerpo.

\Begin{proposition}

En un dominio de ideales principales \(R\) todos los ideales primos no
nulos son maximales. \End{proposition}

\Begin{proof}

Supongamos que \((a)\subset (b)\subset R\), con \((a)\) primo y
\(a\neq0\). Entonces \(a=cb\) para cierto \(c\in R\). En particular
\(cb\in (a)\), que es primo, luego \(c\in (a)\) o \(b\in (a)\).

Si \(b\in (a)\) entonces \((b)\subset (a)\), luego \((a)=(b)\).

Si \(c\in (a)\) entonces \(c=da\) para cierto \(d\in R\), por tanto
\(a=dab=dba\). Por la propiedad cancelativa \(db=1\), así que \(b\) es
una unidad, luego \((b)=R\). \End{proof}

\Begin{example}\textrm{\normalfont (Ideales maximales y geometría)} Dado
un cuerpo \(k\), todo punto \(\mathbf{a}=(a_1,\dots,a_n)\in k^n\) del
espacio afín \(n\)-dimensional define un ideal maximal de
\(k[x_1,\dots,x_n]\), \[I_{\mathbf{a}}=(x_1-a_1,\dots,x_n-a_n).\]
Es en efecto maximal porque es el núcleo del homomorfismo sobreyectivo
de evaluación
\[\begin{array}{rcl} k[x_1,\dots,x_n]&\longrightarrow& k,\cr p(x_1,\dots,x_n)&\mapsto&p(a_1,\dots,a_n). \end{array}\]
Por tanto \(k[x_1,\dots,x_n]/I_{\mathbf{a}}\cong k\) es un cuerpo por
el primer teorema de isomorfía. El \textbf{Teorema de los Ceros de
Hilbert} dice que si \(k=\mathbb C\) o cualquier otro cuerpo
algebraicamente cerrado, entonces estos son los únicos ideales maximales
de \(k[x_1,\dots,x_n]\), con lo que tendríamos una biyección,
\[\{\text{Ideales maximales de }k[x_1,\dots,x_n]\}\longleftrightarrow k^n.\]
Como consecuencia de esto y de la caracterización de ideales de un
cociente deducimos que si \(X\subset k^n\) es el conjunto de soluciones
de unas ecuaciones polinómicas,
\(p_i(x_1,\dots,x_n)\in k[x_1,\dots,x_n]\), \(1\leq i\leq m\),
\[X\colon\left\{ \begin{array}{c} p_1(x_1,\dots,x_n)=0,\cr \vdots\quad\cr p_m(x_1,\dots,x_n)=0, \end{array} \right.\]
entonces tenemos una biyección
\[\{\text{Ideales maximales de }k[x_1,\dots,x_n]/(p_1,\dots,p_m)\}\longleftrightarrow X.\]
El álgebra del anillo cociente \(k[x_1,\dots,x_n]/(p_1,\dots,p_m)\) no
solo captura el conjunto de puntos de \(X\) sino toda su geometría. Por
desgracia, otros resultados más precisos al respecto se escapan del
alcance de la asignatura. \End{example}

\Begin{example}\textrm{\normalfont (Ideales maximales, análisis y topología)}
Dado un espacio topológico compacto de Hausdorff \(X\), denotamos
\(\mathcal C(X)\) al anillo de funciones continuas
\(X\rightarrow \mathbb C\). Cualquier punto \(x\in X\) define un
homomorfismo sobreyectivo
\[\begin{array}{rcl} ev_x\colon\mathcal C(X)&\longrightarrow& \mathbb C,\cr f&\mapsto&f(x), \end{array}\]
cuyo núcleo \(\ker ev_x\subset\mathcal C(X)\) es un ideal maximal por el
primer teorema de isomorfía. La \textbf{Teoría de Representación de
Gelfand} dice que todos los ideales maximales de \(\mathcal C(X)\) son
de esta forma, con lo que tenemos una biyección
\[\{\text{Ideales maximales de }\mathcal C(X)\}\longleftrightarrow X.\]
Esta correspondencia da lugar a una equivalencia de categorías entre los
espacios compactos de Hausdorff y las \(C^{\ast}\)-álgebras conmutativas
unitarias, que es una clase de anillos a la que \(\mathcal C(X)\)
pertenece. Esto permite estudiar la topología desde el punto de vista
del álgebra y del análisis funcional. \End{example}
