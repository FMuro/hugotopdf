
La noción clásica de divisibilidad se comporta de manera inesperada en
presencia de divisores de cero. Aquí evitaremos estos problemas y
estudiaremos la divisibilidad en dominios. Pondremos especial énfasis en
el estudio de anillos donde el sea posible generalizar el teorema
fundamental de la aritmética. Este teorema, ya conocido para los números
enteros y más generalmente para los dominios euclídeos, nos dice que
todo número no nulo que no sea una unidad factoriza como producto de
primos de manera esencialmente única. Veremos además cómo estos anillos
son de utilidad a la hora de resolver ecuaciones diofánticas.

\hypertarget{divisores}{%
\subsection{Divisores}\label{divisores}}

\Begin{definition}

En un dominio \(R\), decimos que \(a\) \textbf{divide} a \(b\), o que
\(b\) es un \textbf{múltiplo} de \(a\), y escribimos \(a|b\), si
\(b=aq\) para cierto \(q\in R\), que llamaremos \textbf{cociente}.
Decimos que \(a\) es un \textbf{divisor propio} de \(b\) si además ni
\(a\) ni \(q\) son unidades. Un elemento no trivial de \(R\) es
\textbf{irreducible} si no tiene divisores propios ni es una unidad. Dos
elementos \(a\) y \(a'\) son \textbf{asociados} si tanto \(a| a'\) como
\(a'|a\). \End{definition}

\Begin{watch}

El cero solo divide al cero, por tanto el cero es el único asociado del
cero. Es más, si \(a\neq 0\) y \(a|b\) el cociente es único, es decir,
solo hay un \(q\in R\) tal que \(b=qa\) pues si \(b=q'a\) entonces
\(qa=q'a\) y por la propiedad cancelativa \(q=q'\). \End{watch}

\Begin{proposition}

En un dominio \(R\):

\begin{itemize}
\item
  \(a|b \Leftrightarrow (a)\supset (b)\).
\item
  \((a)=(a')\Leftrightarrow\) \(a\) y \(a'\) son asociados
  \(\Leftrightarrow a'=ua\) para cierta unidad \(u\).
\item
  \(u\) es una unidad \(\Leftrightarrow (u)=(1)\).
\item
  \(a\) es un divisor propio de
  \(b \Leftrightarrow (1)\supsetneq (a)\supsetneq (b)\).
\end{itemize}

\End{proposition}

\Begin{proof}

\begin{itemize}
\item
  \((a)\supset (b)\Leftrightarrow (a)\ni b\Leftrightarrow a|b\).
\item
  La primera equivalencia se sigue del apartado anterior. Para la
  segunda, \(\Leftarrow\) es obvio pues también \(a=u^{-1}a'\). Veamos
  \(\Rightarrow\). Si \(a\) y \(a'\) son asociados entonces \(a'=qa\) y
  \(a=q'a'\), luego \(a=q'qa\). Si \(a=0\) entonces \(a'=0\) y podemos
  tomar \(u=1\). En caso contrario, por la propiedad cancelativa
  \(1=q'q\), luego \(q'\) y \(q\) son unidades y podemos tomar \(u=q\).
\item
  \(\Rightarrow\) se ha visto antes. \(\Leftarrow\) si \((u)=(1)\)
  entonces existe \(r\in R\) tal que \(ru=1\), con lo que \(u\) es una
  unidad.
\item
  Es consecuencia de los tres apartados anteriores.
\end{itemize}

\End{proof}

\Begin{remark}

Si \(a\in R\) no es nulo ni una unidad, un divisor de \(a\) es bien
propio, bien asociado, o bien una unidad, pero no puede ser dos de estas
cosas a la vez. En particular, si \(a\) y \(a'\) son irreducibles y
\(a|a'\) entonces \(a\) y \(a'\) son asociados. Los divisores de una
unidad son las unidades. La primera caracterización de los asociados es
especialmente interesante porque demuestra que cualquier propiedad de
elementos de \(R\) que pueda enunciarse en términos de sus
correspondientes ideales principales es también válida para los
elementos asociados. \End{remark}

\hypertarget{factorizaciones}{%
\subsection{Factorizaciones}\label{factorizaciones}}

En un dominio \(R\), si \(a\in R\) es un elemento no trivial que no es
una unidad ni tampoco irreducible, entonces podemos descomponerlo como
producto de dos divisores propios \(a=bc\). Lo mismo podemos hacer con
\(b\) y con \(c\), y así sucesivamente. Este procedimiento puede acabar
en una descomposición de \(a=b_1\cdots b_n\) como producto de
elementos irreducibles \(b_i\in R\), \(1\leq i\leq n\), pero también
podría no acabar nunca.

\Begin{definition}

Decimos que \textbf{existen factorizaciones} en un dominio \(R\) si el
anterior proceso de factorización acaba para todo elemento no nulo que
no sea una unidad. \End{definition}

\Begin{proposition}

En un dominio \(R\) existen factorizaciones \(\Leftrightarrow\) no
existe ninguna sucesión estrictamente creciente de ideales principales
\((a_1)\subsetneq(a_2)\subsetneq(a_3)\subsetneq\cdots\).
\End{proposition}

\Begin{proof}

\(\Rightarrow\) En lugar de A \(\Rightarrow\) B demostraremos (no A)
\(\Leftarrow\) (no B). Tomamos pues una sucesión
\((a_1)\subsetneq(a_2)\subsetneq(a_3)\subsetneq\cdots\). Podemos suponer
que \(a_1\neq 0\) ya que en caso contrario la inclusión estricta
\((a_1)\subsetneq(a_2)\) implicaría que \(a_2\neq 0\) y bastaría
reindexar. Observa que \((a_n)\subsetneq (1)\) para todo \(n\geq 1\),
ya que la sucesión es estrictamente creciente y \((1)=R\) es el ideal
total. Las inclusiones \((a_n)\subsetneq (a_{n+1})\subsetneq(1)\) nos
dicen entonces que \(a_n\) se puede descomponer como producto de
divisores propios \(a_n=q_{n+1}a_{n+1}\), por tanto el proceso de
factorización no termina para \[\begin{array}{rcl}
a_1&=& q_2a_2\cr
&=& q_2q_3a_3\cr
&=&q_2q_3q_4a_4\cr
&=&\cdots,
\end{array}\] lo cual es una contradicción.

\(\Leftarrow\) Como antes, en lugar de A \(\Leftarrow\) B demostraremos
(no A) \(\Rightarrow\) (no B). Probaremos pues que si no existen
factorizaciones entonces podemos encontrar una sucesión
\((a_1)\subsetneq(a_2)\subsetneq(a_3)\subsetneq\cdots\). Sea
\(a_1\in R\) un elemento no nulo que no sea una unidad para el cual el
proceso de factorización no termina. Entonces podemos descomponerlo como
producto de divisores propios \(a_1=q_2a_2\) de modo que alguno de
los dos no es irreducible. Como el producto es conmutativo podemos
suponer que es \(a_2\) el que no es irreducible. Por tanto este también
se puede descomponer como producto de dos divisores propios
\(a_2=q_3a_3\) alguno de los cuales no es irreducible. Una vez más
podemos suponer que es \(a_3\) el no irreducible y continuar
indefinidamente con el proceso. Por construcción, esto nos da una
sucesión creciente
\((a_1)\subsetneq(a_2)\subsetneq(a_3)\subsetneq\cdots\) que es estricta
porque todas las descomposiciones anteriores son como producto de dos
divisores propios. \End{proof}

La condición de la derecha del enunciado de la proposición anterior se
suele denominar \textbf{condición de cadena ascendente} para ideales
principales. Las condiciones de cadena juegan un papel muy importante en
el álgebra moderna. Los anillos que satisfacen la condición de cadena
ascendente para ideales cualesquiera se denominan \textbf{noetherianos},
por \href{https://es.wikipedia.org/wiki/Emmy_Noether}{Emmy Noether}. Los
que cumplen la \textbf{condición de cadena descendente} para ideales
arbitrarios se llaman \textbf{artinianos}, por
\href{https://es.wikipedia.org/wiki/Emil_Artin}{Emil Artin}, padre de
Michael, el autor del libro que estamos siguiendo.

\Begin{example}\textrm{\normalfont (Un dominio sin factorizaciones)} No
es fácil construir dominios donde no existan factorizaciones. El ejemplo
más sencillo es el cociente \(R/I\), donde \[R=k[x_1,x_2,x_3,\dots]\]
es un anillo de polinomios sobre un cuerpo \(k\) en una sucesión
infinita de variables \(\{x_n\}_{n\geq 1}\) e
\[I=(x_1-x_2^2, x_2-x_3^2, x_3-x_4^2,\dots)\] es un ideal
infinitamente generado que fuerza las relaciones
\(\bar x_n=\bar x_{n+1}^2\in R/I\), \(n\geq 1\). En este cociente
\[(\bar x_1)\subsetneq(\bar x_2)\subsetneq(\bar x_3)\subsetneq\cdots.\]
Más concretamente, el proceso
\(\bar x_1=\bar x_2\bar x_2=\bar x_2\bar x_3\bar x_3=\bar x_2\bar x_3\bar x_4\bar x_4=\cdots\)
no termina.

En rigor, es necesario probar que ningún \(\bar{x}_i\) es una unidad.
Esto es cierto porque de lo contrario existiría algún polinomio
\(f\in R\) tal que \(\bar{x}_i\bar{f}=1\), o equivalentemente
\(x_if-1\in I\). Esto es imposible porque los elementos de \(I\), al ser
combinaciones lineales de sus generadores, no tienen término
independiente. \End{example}

\Begin{definition}

Un \textbf{dominio de factorización única} (también \textbf{DFU} o
\textbf{UFD}) es un dominio donde existen factorizaciones y tal que dos
factorizaciones de un mismo elemento coinciden salvo orden y asociados,
es decir si \[b_1\cdots b_s=c_1\cdots c_t\] son productos de
irreducibles entonces \(s=t\) y existe una permutación
\(\sigma\in S_s\) de \(s\) elementos tal que \(b_i\) y
\(c_{\sigma(i)}\) son asociados, \(1\leq i\leq s\). \End{definition}

\Begin{example}\textrm{\normalfont (Un dominio con factorizaciones que no son únicas)}
En \(\mathbb Z[\sqrt{-5}]\subset\mathbb C\),
\[9=3^2=(2+\sqrt{-5})(2-\sqrt{-5}).\] Todo elemento de
\(\mathbb Z[\sqrt{-5}]\) es de la forma \[z=a+b\sqrt{-5}=a+ib\sqrt{5}\]
donde \(a,b\in\mathbb Z\). El cuadrado del módulo de tal elemento es
\[|z|^2=|a+b\sqrt{-5}|=a^2+5b^2\in\mathbb Z.\] Como
\[|z_1z_2|^2=|z_1|^2\cdot|z_2|^2,\] si \(z\in \mathbb Z[\sqrt{-5}]\) es
una unidad entonces \(1=|z|^2\cdot|z^{-1}|^2\), así que
\(|z|^2\in\mathbb Z\) es invertible (y positivo), luego \(|z|^2=1\). La
única posibilidad de que esto ocurra es que \(z=\pm 1\), por tanto las
unidades de nuestro anillo son \(\{\pm1\}\). Es más, las no unidades
tienen módulo al cuadrado \(>1\), así que si \(z_1\) es un divisor
propio de \(z_2\) entonces \(|z_1|^2<|z_2|^2\).

En \(\mathbb Z[\sqrt{-5}]\) existen factorizaciones. En efecto, si
hubiera una cadena
\((z_1)\subsetneq(z_2)\subsetneq(z_3)\subsetneq\cdots\) entonces
tendríamos que \(|z_1|^2>|z_2|^2>|z_3|^2>\cdots\geq 0\), pero esto es
imposible, no hay sucesiones infinitas estrictamente decrecientes de
enteros no negativos.

Veamos que \(3\) es irreducible. Si \(3=z_1z_2\) es una factorización
como producto de divisores propios entonces \(3^2=|z_1|^2|z_2|^2\). No
puede ser que \(|z_i|^2=1\) ya que \(z_i\) no es una unidad para ningún
\(i\in\{1,2\}\), luego, por el teorema fundamental de la aritmética
para números enteros, \(|z_1|^2=|z_2|^2=3\), pero no hay ningún elemento
en \(\mathbb Z[\sqrt{-5}]\) que tenga este cuadrado de módulo.
Análogamente, \(2\pm\sqrt{-5}\) es irreducible pues si
\(2\pm\sqrt{-5}=z_1z_2\) entonces de nuevo
\(|2\pm\sqrt{-5}|^2=3^2=|z_1|^2|z_2|^2\).

Por último, el \(3\) no es un asociado de \(2\pm\sqrt{-5}\) porque las
únicas unidades de \(\mathbb Z[\sqrt{-5}]\) son \(\pm1\), luego las dos
factorizaciones anteriores de \(9\) como producto de irreducibles son
esencialmente distintas. \End{example}

Con el objetivo de caracterizar los DFU, introducimos un nuevo tipo de
elemento.

\Begin{definition}

Un elemento de un dominio \(p\in R\) es \textbf{primo} si no es nulo ni
una unidad y además si \(p|ab\) entonces bien \(p|a\) o bien \(p|b\).
\End{definition}

\Begin{remark}

En términos de ideales, un elemento \(p\in R\) no nulo es primo si y
solo si \((p)\subset R\) es un ideal primo. \End{remark}

\Begin{proposition}

En un dominio, todo elemento primo \(p\in R\) es irreducible.
\End{proposition}

\Begin{proof}

El primo no es una unidad porque los ideales primos son distintos del
total, \((p)\neq (1)\). Veamos que no tiene divisores propios. Para ello
supongamos que \(p=ab\) y demostremos que \(a\) o \(b\) es una unidad.
Como \(p|ab\), entonces \(p|a\) o \(p|b\). Si \(p|a\) entonces \(a=pq\),
luego \(p=ab=pqb\). Como los primos no son nulos y estamos en un dominio
esto implica que \(1=qb\), por tanto \(b\) es una unidad. Si \(p|b\)
deducimos del mismo modo que \(a\) tendría que ser una unidad.
\End{proof}

\Begin{example}\textrm{\normalfont (Un irreducible que no es primo)}
Continuando con el ejemplo anterior, el irreducible
\(3\in \mathbb Z[\sqrt{-5}]\) no es primo porque
\(3|(2+\sqrt{-5})(2-\sqrt{-5})\) pero \(3\nmid 2\pm \sqrt{-5}\) porque
ambos son irreducibles pero no asociados. \End{example}

\Begin{theorem}

Un dominio \(R\) es de factorización única \(\Leftrightarrow\) existen
factorizaciones y todo irreducible es primo. \End{theorem}

\Begin{proof}

\(\Rightarrow\) Basta ver que los irreducibles son primos. Sea
\(c\in R\) irreducible. Para ver que es primo, supongamos que \(c|ab\).
Hay que probar que \(c|a\) o \(c|b\). Como \(c|ab\), entonces \(ab=cd\)
para cierto \(d\in R\). Si \(a=0\) entonces \(c|a\) y si \(b=0\),
\(c|b\). Si \(a\) es una unidad, entonces despejando vemos que \(c|b\),
y si \(b\) es una unidad \(c|a\). Supongamos en adelante que \(a\) y
\(b\) no son nulos ni unidades. Entonces \(d\) no puede ser una unidad,
ya que de lo contrario podriamos despejarla y \(c=(d^{-1}a)b\) no sería
irreducible, pues estaría descompuesto como producto de dos divisores
propios. Por tanto \(a\), \(b\) y \(d\) factorizan como productos de
irreducibles, \[\begin{array}{rcl}
a&=&a_1\cdots a_r,\cr
b&=&b_1\cdots b_s,\cr
d&=&d_1\cdots d_t.
\end{array}\] Tenemos pues que
\[a_1\cdots a_rb_1\cdots b_s=cd_1\cdots d_t\] son dos
descomposiciones de un mismo elemento como producto de irreducibles. Por
la unicidad, \(c\) ha de ser asociado de algún \(a_i\) o \(b_j\), por
tanto \(c\) divide a \(a\) o a \(b\).

\(\Leftarrow\) Veamos primero que si un primo \(p\) divide a un producto
de irreducibles \(a_1\cdots a_r\) entonces \(p\) es asociado de algún
\(a_i\). Procedemos por inducción en \(r\). Para \(r=1\), como \(p|a_1\)
y ambos elementos son irreducibles, han de ser asociados, según hemos
visto anteriormente. Para \(r>1\), como \(p|a_1(a_2\cdots a_r)\), bien
\(p|a_1\) o bien \(p|a_2\cdots a_r\). En el primer caso, ya tratado,
\(p\) es asociado de \(a_1\), y en el segundo, por hipótesis de
inducción, \(p\) es asociado de algún otro \(a_i\).

Consideramos ahora dos productos de irreducibles (primos) que son
iguales \[a_1\cdots a_r=b_1\cdots b_s.\] Supongamos sin pérdida de
generalidad que \(1\leq r\leq s\). Procedemos por inducción en \(r\). Si
\(r=1\) entonces \(s=1\) puesto que un irreducible no tiene divisores
propios. En este caso no hay nada que probar pues \(a_1=b_1\). Sea
\(r>1\). Como \(a_1|b_1\cdots b_s\) y \(a_1\) es primo, \(a_1\) es
asociado de algún \(b_i\), es decir \(b_i=ua_1\) para cierta unidad
\(u\in R\). Por la propiedad cancelativa
\[a_2\cdots a_r=ub_1\cdots\widehat{b}_i\cdots b_s.\] Observa que
\(b_1\) y \(ub_1\) son asociados. Por hipótesis de inducción, las
últimas factorizaciones coinciden salvo orden y asociados, por tanto las
anteriores también. \End{proof}

\Begin{proposition}

En un DFU, un producto de irreducibles \(a_1\cdots a_r\) divide a otro
\(b_1\cdots b_s\) si y solo si \(r\leq s\) y, salvo reordenamiento,
\(a_i\) y \(b_i\) son asociados \(1\leq i\leq r\). \End{proposition}

\Begin{proof}

Como el primer producto divide al segundo, existe \(c\in R\) tal que
\(a_1\cdots a_rc=b_1\cdots b_s\). Si \(c\) no es una unidad, lo
factorizamos como producto de irreducibles \(c=c_1\cdots c_t\),
\[a_1\cdots a_rc_1\cdots c_t=b_1\cdots b_s.\] Por la unicidad de las
factorizaciones, cada \(a_i\) es asociado a un \(b_j\) distinto, y salvo
reordenamiento podemos suponer que son los \(r\) primeros. Si \(c\)
fuera una unidad, la unicidad de las factorizaciones demostraría
directamente que \(r=s\) y que cada \(a_i\) es asociado de un \(b_j\)
distinto. \End{proof}

\Begin{corollary}

Dados dos elementos no nulos de un DFU, \(a,b\in R\), existe un
\textbf{divisor común máximo} \(d\in R\), que es un elemento que
satisface:

\begin{itemize}
\item
  \(d|a\) y \(d|b\).
\item
  si \(d'|a\) y \(d'|b\) entonces \(d'|d\).
\end{itemize}

El divisor común máximo es único salvo asociados y se denota
\(\operatorname{mcd}(a,b)\) o \(\gcd(a,b)\). \End{corollary}

\Begin{proof}

Si \(a\) o \(b\) fuera una unidad, cualquier unidad sería un divisor
común máximo puesto los divisores de una unidad son las unidades. En
caso contrario, basta factorizar ambos elementos \(a=a_1\cdots a_r\) y
\(b=b_1\cdots b_s\) como producto de primos y tomar \(d\) como el
producto del mayor subconjunto de los factores de \(a\) que tienen
asociados distintos entre los factores de \(b\). La unicidad salvo
asociados se deduce de que si \(d\) y \(d'\) son divisores comunes
máximos de \(a\) y \(b\) entonces por definición \(d|d'\) y \(d'|d\).
\End{proof}

\Begin{remark}

Si \(a\) o \(b\) es una unidad \(\operatorname{mcd}(a,b)=1\). A
diferencia de los dominios euclídeos, en un DFU un divisor común máximo
no tiene por qué satisfacer una identidad de Bézout, es decir,
\(\operatorname{mcd}(a,b)\) no tiene por qué estar en el ideal
\((a,b)\). Veremos ejemplos más adelante. \End{remark}

\Begin{corollary}

Dados dos elementos no nulos de un DFU, \(a,b\in R\), existe un
\textbf{múltiplo común mínimo} \(m\in R\), que es un elemento que
satisface:

\begin{itemize}
\item
  \(a|m\) y \(b|m\).
\item
  si \(a|m'\) y \(b|m'\) entonces \(m|m'\).
\end{itemize}

El múltiplo común mínimo es único salvo asociados y se denota
\(\operatorname{mcm}(a,b)\) o \(\operatorname{lcm}(a,b)\).
\End{corollary}

\Begin{proof}

Si \(d\) es un divisor común máximo entonces
\(m=\frac{ab}{d}=a\frac{b}{d}=\frac{a}{d}b\) es un múltiplo común
mínimo. En efecto, la primera propiedad es obvia. Comprobemos la
segunda. Si \(a|m'\) y \(b|m'\) entonces \(d|m'\). Es más,
\(\frac{a}{d}|\frac{m'}{d}\) y \(\frac{b}{d}|\frac{m'}{d}\). Como
\(\frac{a}{d}\) y \(\frac{b}{d}\) no tienen factores primos asociados,
su múltiplo común mínimo es el producto \(\frac{ab}{d^2}\), así que
\(\frac{ab}{d^2}|\frac{m'}{d}\). Multiplicando por \(d\) deducimos que
\(m=\frac{ab}{d}|m'\). La unicidad salvo asociados se deduce como en el
caso del divisor común máximo. \End{proof}

\Begin{lemma}

Dada una sucesión creciente de ideales
\(I_1\subset I_2\subset I_3\subset\cdots\) en un anillo \(R\), su
unión \(I_\infty=\cup_{n\geq 1}I_n\) es un ideal. \End{lemma}

\Begin{proof}

Por un lado, \(0\in I_1\subset I_{\infty}\). Por otro, dados
\(a,b\in I_\infty\) es obvio que \(a,b\in I_n\) para cierto
\(n\geq 1\), por tanto \(a+b\), \(-a\) y \(ra\), \(r\in R\), pertenecen
a \(I_n\subset I_\infty\). \End{proof}

\Begin{proposition}

Todo DIP es un DFU. \End{proposition}

\Begin{proof}

Sea \(R\) un DIP. Veamos por reducción al absurdo que no puede existir
una sucesión estrictamente creciente de ideales principales
\[(a_1)\subsetneq(a_2)\subsetneq(a_3)\subsetneq\cdots.\] Al estar en un
DIP, \(\cup_{n\geq 1}(a_n)=(b)\) para cierto \(b\in R\). Como
\(b\in \cup_{n\geq 1}(a_n)\) este elemento ha de estar en cierto
término de la sucesión, \(b\in (a_n)\). Si esto es así
\((b)\subset (a_n)\subsetneq (a_{n+1})\subset (b)\), lo cual es una
contradicción.

Veamos ahora que todo irreducible \(a\in R\) es primo. Probaremos por
reducción al absurdo que el ideal \((a)\) es maximal y por tanto primo.
Si \((a)\), que es un ideal propio y no trivial, no fuera maximal
podríamos encontrar un ideal \(I\) tal que
\((a)\subsetneq I\subsetneq R\). Como \(R\) es un DIP, \(I=(b)\) para
cierto \(b\in I\). Entonces tenemos que
\((a)\subsetneq(b)\subsetneq(1)\). Esto quiere decir que \(b\) es un
divisor propio de \(a\), lo cual es una contradicción. \End{proof}

Más adelante veremos ejemplos de DFU que no son DIP.

\Begin{proposition}

En un DIP, dados \(a,b\in R\), cualquier \(d\in R\) tal que
\((a,b)=(d)\) es un \(\operatorname{mcd}(a,b)\). \End{proposition}

\Begin{proof}

Como \(a,b\in (a,b)=(d)\), \(d\) es un divisor común. Si \(d'|a\) y
\(d'|b\) entonces \(a,b\in (d')\), luego \((d)=(a,b)\subset (d')\) y por
tanto \(d'|d\). \End{proof}

Acabamos de demostrar que en un DIP todo divisor común máximo satisface
una \textbf{identidad de Bézout}.

\Begin{definition}

Un \textbf{dominio euclídeo} es un dominio \(R\) equipado con una
función
\[\delta\colon R\setminus\{0\}\longrightarrow\{0,1,2\dots\},\]
llamada \textbf{función de tamaño} o \textbf{euclídea}, tal que dados
\(D,d\in R\) con \(d\neq 0\), denominados \textbf{dividendo} y
\textbf{divisor}, respectivamente, existen \(c,r\in R\),
\textbf{cociente} y \textbf{resto}, de modo que \[D=dc+r\] y bien
\(r=0\) o bien \(\delta( r )<\delta(d)\). \End{definition}

\Begin{remark}

Sabemos que \(\mathbb Z\) y \(k[x]\), con \(k\) un cuerpo, son dominios
euclídeos con función de tamaño el valor absoluto y el grado,
respectivamente. En general, el cociente y el resto no están
determinados de manera única por el dividendo y el divisor, por ejemplo
\[13=4\cdot 3+1=5\cdot 3+(-2).\] Los dominios euclídeos son dominios de
ideales principales. Todo ideal no trivial está generado por cualquiera
de sus elementos no nulos de menor tamaño. \End{remark}

\Begin{example}\textrm{\normalfont (Los enteros de Gauss)} Vamos a ver
que \(\mathbb{Z}[i]\) con el cuadrado del módulo como función euclídea
es un dominio euclídeo. Tomamos \(D,d\in\mathbb{Z}[i]\), este último no
nulo,\[\begin{array}{rcl}D&=&a+ib,\cr d&=& x+iy.\cr \end{array}\]
Encontrar un cociente euclídeo se reduce a hallar un múltiplo de \(d\)
en el interior del círculo de centro \(D\) y radio \(|d|\). Vamos a ver
cómo hacerlo de manera analítica. Consideramos el número complejo
\[\frac{D}{d}=u+iv.\] Aquí \(u\) y \(v\) son números reales, de hecho
racionales, pero no necesariamente enteros. Aproximamos el anterior
número complejo por un entero de Gauss \[c=u_0+iv_0\in\mathbb Z[i]\] de
modo que sus partes real e imaginaria estén lo más cerca posible de las
del complejo \(\frac{D}{d}\),
\[\begin{array}{rcl} |u-u_0|&\leq &\frac{1}{2},\cr |v-v_0|&\leq &\frac{1}{2}. \end{array}\]
De este modo
\[\left|\frac{D}{d}-c\right|=\sqrt{(u-u_0)^2+(v-v_0)^2}\leq \frac{1}{\sqrt{2}}.\]
Veamos que \(c\) es el cociente de una división euclídea. El resto sería
\(r=D-dc\) y su módulo es
\[|r|=|D-dc|=|d|\cdot \left|\frac{D}{d}-c\right|\leq \frac{|d|}{\sqrt{2}}<|d|.\]
\End{example}

\Begin{example}\textrm{\normalfont (Enteros cuadráticos)} Un entero
\(n\in\mathbb Z\) es \textbf{libre de cuadrados} no es divisible por el
cuadrado de ningún primo, es decir, si entre sus factores primos no
podemos encontrar dos asociados. Por ejemplo, \(-4=2(-2)\) no es libre
de cuadrados pero \(6=2\cdot 3\) y \(-1\) sí. Los \textbf{cuerpos de
números cuadráticos} son \(\mathbb Q[\sqrt{n}]\subset\mathbb C\) donde
\(n\) es un entero libre de cuadrados. Su \textbf{anillo de enteros}
\(R\subset\mathbb Q[\sqrt{n}]\) está formado por los elementos que son
raíces de un polinomio mónico en \(\mathbb Z[x]\). Se puede comprobar
que \(R=\mathbb Z[\sqrt{n}]\) si \(n\equiv 2,3\) mod \(4\) y
\(R=\mathbb Z[\frac{1+\sqrt{n}}{2}]\) si \(n\equiv 1\) mod \(4\).
Decimos que \(R\) es un \textbf{anillo de enteros cuadráticos
imaginarios} si \(n{<}0\). Los anillos de enteros cuadráticos
imaginarios que son DIPs se obtienen para
\[n=-1, -2, -3, -7, -11, -19, -43, -67, -163.\] De estos, son dominios
euclídeos para \[n=-1, -2, -3, -7, -11.\] En todos estos casos podemos
además tomar el cuadrado del módulo como función de tamaño. El resto de
anillos de enteros cuadráticos imaginarios no son ni siquiera DFUs. Para
\(n{>}0\), obtenemos dominios euclídeos con la `norma'
\(N(a+b\sqrt{n})=a^2-b^2n\) para
\[n=2, 3, 5, 6, 7, 11, 13, 17, 19, 21, 29, 33, 37, 41, 57, 73.\] Para
\(n=69\), \(R=\mathbb Z[\frac{1+\sqrt{69}}{2}]\) es también un dominio
euclídeo pero no con una función de tamaño distinta de \(N\).
\End{example}

La siguiente aplicación interactiva te permite explorar la posibilidad
de realizar divisiones euclídeas respecto del cuadrado del módulo
complejo en el anillo de enteros cuadráticos imaginarios
\(R\subset\mathbb Q[\sqrt{-n}]\) para ciertos valores positivos de
\(n\). Para \(n=1\) tenemos los enteros de Gauss. Puedes seleccionar los
coeficientes del dividendo \(D=a+b\sqrt{-n}\) y del divisor
\(d=x+y\sqrt{-n}\), si \(-n\not\equiv 1\) mod \(4\). Si \(-n\equiv 1\)
mod \(4\), el dividendo y el divisor son de la forma
\(D=a+b\frac{1+\sqrt{-n}}{2}\) y \(d=x+y\frac{1+\sqrt{-n}}{2}\),
respectivamente. Los coeficientes del dividendo pueden estar en
\([-10,10]\) y los del divisor en \([-5,5]\). Aparece un círculo
amarillo centrado en \(D\) de radio \(|d|\). Los puntos verdes son
elementos del anillo y los azules son además múltiplos del divisor. Cada
punto azul en el \emph{interior} del círculo representa una división
euclídea. La aplicación da la lista de todos los pares \((c,r)\) que
producen divisiones euclídeas \(D=d\cdot c+r\).

\hypertarget{polinomios}{%
\subsection{Polinomios}\label{polinomios}}

En este epígrafe demostraremos que los anillos de polinomios con
coeficients en un DFU son también DFUs. En adelante \(R\) denotará un
DFU y \(k=Q( R )\) su cuerpo de fracciones.

\Begin{definition}

Un polinomio no nulo \(f=f(x)=a_nx^n+\cdots+a_1x+a_0\in R[x]\) es
\textbf{primitivo} si el divisor común máximo de sus coeficientes es
\(1\), es decir, si no existe ningún primo \(p\in R\) tal que \(p|a_i\)
para todo \(1\leq i\leq n\). \End{definition}

Los únicos polinomios constantes primitivos son las unidades de \(R\).

\Begin{lemma}

Dado \(f=f(x)=a_nx^n+\cdots+a_1x+a_0\in k[x]\) no nulo existe una
constante \(c\in k\), llamada \textbf{contenido}, y un polinomio
primitivo \(f_0(x)\in R[x]\) tal que \[f(x)=c\cdot f_0(x).\] Además
\(c\) y \(f_0(x)\) son únicos salvo producto por unidades de \(R\).
Denotaremos \(c=\operatorname{cont}(f)\). \End{lemma}

\Begin{proof}

Veamos la existencia. Podemos quitar denominadores de los coeficientes
de \(f(x)\) multiplicando por una constante \(d\in R\) no nula,
\[d\cdot f(x)\in R[x].\] Si \(e\) es el divisor común máximo de los
coeficientes de \(d\cdot f(x)\) vemos que podemos tomar
\[\begin{array}{rcl}
f_0&=&\frac{d}{e}\cdot f(x),\cr
c&=&\frac{e}{d}.
\end{array}\]

Probemos ahora la unicidad. Supongamos que
\(c\cdot f_0(x)=c'\cdot f'_0(x)\) siendo \(f_0(x),f'_0(x)\in R[x]\)
primitivos. Podemos además suponer sin pérdida de generalidad que
\(c,c'\in R\), multiplicando por un denominador común si fuera
necesario. Como el divisor común máximo de los coeficientesde \(f_0(x)\)
es \(1\), el divisor común máximo de los coeficientes de
\(c\cdot f_0(x)\) es \(c\). Análogamente el divisor común máximo de los
coeficientes de \(c'\cdot f'_0(x)\) es \(c'\). Por la unicidad del
divisor común máximo, \(c\) y \(c'\) son asociados, es decir
\(c'=u\cdot c\) donde \(u\in R\) es una unidad. Por tanto, por la
propiedad cancelativa, \(f_0(x)=u\cdot f_0'(x)\). \End{proof}

\Begin{remark}

Si el contenido de un polinomio \(f(x)\in k[x]\) está en \(R\) entonces
\(f(x)\in R[x]\). Recíprocamente, el contenido de un polinomio
\(f(x)\in R[x]\) es el divisor común máximo de sus coeficientes, en
particular \(\operatorname{cont}(f)\in R\). Es más, dada una constante
\(a\in R\) tenemos que \(a|f(x)\) si y solo si
\(a|\operatorname{cont}(f)\). Un polinomio \(f(x)\in R[x]\) es primitivo
si y solo si \(\operatorname{cont}(f)=1\). \End{remark}

\Begin{theorem}\textrm{\normalfont (Lema de Gauss)} El producto de
polinomios primitivos en \(R[x]\) es primitivo. \End{theorem}

\Begin{proof}

Dado un primo \(p\in R\), consideramos el homomorfismo de
\textbf{reducción módulo \(p\)}
\[\phi_p\colon R[x]\longrightarrow (R/(p))[x]\] definido en las
constantes como \(\phi_p(a)=\bar a\), \(a\in R\), tal que
\(\phi_p(x)=x\). Es decir,
\[\phi_p(a_nx^n+\cdots+a_1x+a_0)=\bar a_nx^n+\cdots+\bar a_1x+\bar a_0.\]
El homomorfismo \(\phi_p\) consiste simplemente en reducir los
coeficientes módulo \((p)\). En particular \(f\in \ker \phi_p\) si y
solo si \(p\) divide a todos los coeficientes de \(f\). Por tanto
\(f\in R[x]\) es primitivo si y solo si \(\phi_p(f)\neq 0\) para todo
\(p\in R\) primo. Si \(f,g\in R[x]\) son primitivos entonces
\[\phi_p(f\cdot g)=\phi_p(f)\cdot \phi_p(g)\neq 0\] para todo \(p\in R\)
primo ya que \((R/(p))[x]\) es un dominio. Es decir, \(f\cdot g\)
también es primitivo. \End{proof}

\Begin{corollary}

Dados \(f,g\in k[x]\) tenemos que
\(\operatorname{cont}(f\cdot g)=\operatorname{cont}(f)\cdot \operatorname{cont}(g)\).
\End{corollary}

\Begin{proof}

Tomamos \(f,g\in k[x]\) y los descomponemos
\[\begin{array}{rcl} f&=&c\cdot f_0,\cr g&=&d\cdot g_0, \end{array}\]
con \(c,d\in k\) y \(f_0,g_0\in R[x]\) primitivos. Entonces
\[f\cdot g=(c\cdot d)\cdot (f_0\cdot g_0).\] Como \(f_0\cdot g_0\) es
primitivo por el Lema de Gauss, esta es una descomposición válida del
producto \(f\cdot g\), así que \(c\cdot d\) es su contenido. \End{proof}

\Begin{proposition}

Dados \(f,g\in R[x]\), si \(g|f\) en \(k[x]\) y \(g\) es primitivo
entonces \(g|f\) en \(R[x]\). \End{proposition}

\Begin{proof}

Supongamos que \(f=g\cdot q\) en \(k[x]\). Como \(g\) es primitivo,
\[\operatorname{cont}(f)=\operatorname{cont}(g)\operatorname{cont}(q)=\operatorname{cont}(q).\]
Como \(f\in R[x]\) su contenido está en \(R\), y como este coindice con
el de \(q\), entonces \(q\in R[x]\), por lo que \(g|f\) en \(R[x]\).
\End{proof}

\Begin{proposition}

Un polinomio \(f\in R[x]\) no constante es irreducible en \(R[x]\)
\(\Leftrightarrow\) \(f\) es primitivo e irreducible en \(k[x]\).
\End{proposition}

\Begin{proof}

\(\Leftarrow\) Supongamos que por reducción al absurdo que \(f\) no es
irreducible en \(R[x]\). Lo descomponemos como producto de divisores
propios \(f=gh\) en \(R[x]\). Si \(g\) fuera constante entonces
dividiría al contenido de \(f\), que es \(1\), por tanto \(g\) sería una
unidad, lo cual entra en contradicción con que sea un divisor propio. Lo
mismo ocurriría si \(h\) fuera constante. Si \(g\) y \(h\) no son
constantes entonces también son divisores propios de \(f\) en \(k[x]\),
pues no podrían ser unidades, luego \(f\) no sería irreducible.

\(\Rightarrow\) Si \(f\) no fuera primitivo tampoco sería irreducible en
\(R[x]\) pues su contenido sería un divisor propio. Supongamos por
reducción al absurdo que \(f\) tiene un divisor propio \(g\) en
\(k[x]\). Aquí ser un divisor propio significa que \(0<\) grado de
\(g<\) grado de \(f\). Multiplicando por una constante no nula de \(k\)
si fuera necesario (por el inverso del contenido), podemos suponer que
\(g\in R[x]\) y es primitivo. Por la proposción anterior \(g\) también
divide a \(f\) en \(R[x]\) y por tanto es un divisor propio por cuestión
de grados. \End{proof}

\Begin{remark}

Una constante \(a\in R\) es irreducible en \(R[x]\) si y solo si lo es
en \(R\). \End{remark}

\Begin{theorem}

\(R[x]\) es un DFU. \End{theorem}

\Begin{proof}

Primero probamos que existen factorizaciones en \(R[x]\). Supongamos por
reducción al absurdo que tenemos una sucesión estrictamente creciente de
ideales principales (que podemos suponer no nulos) en este anillo,
\[(f_1)\subsetneq (f_2)\subsetneq (f_3)\subsetneq\cdots.\] Ningún
\((f_n)\) puede ser el ideal total porque la sucesión estabilizaría
necesariamente a partir de este punto. Por tanto, cada \(f_{n+1}\) es
un divisor propio de \(f_{n}\), \(n\geq 1\). En particular grado de
\(0\leq\) grado de \(f_{n+1}\leq\) grado de \(f_{n}\), es decir, los
grados de los generadores forman una sucesion decreciente de enteros no
negativos. Esta sucesión de enteros no nulos ha de estabilizar a partir
de cierto punto, es decir, ha de existir cierto \(n_0\geq 1\) tal que
grado de \(f_{n+1}=\) grado de \(f_{n}\) para todo \(n\geq n_0\), o
equivalentemente \(f_n=c_{n+1}f_{n+1}\) para cierto \(c_{n+1}\in R\)
que no puede ser una unidad ni tampoco nulo. Si llamamos
\(d_n=\operatorname{cont}(f_n)\) tenemos que
\(d_n=c_{n+1}d_{n+1}\). Ningún contenido \(d_n\) puede ser una
unidad porque es divisible por \(c_{n+1}\) así que por tanto
\(d_n=c_{n+1}d_{n+1}\) es una factorización como producto de
divisores propios. Sustituyendo reiteradamente vemos que \[
\begin{array}{rcl}
d_{n_0}&=& c_{n_0+1}d_{n_0+1}\cr
&=& c_{n_0+1}c_{n_0+2}d_{n_0+2}\cr
&=& c_{n_0+1}c_{n_0+2}c_{n_0+3}d_{n_0+3}\cr
&=&\cdots
\end{array}
\] Por tanto el proceso no termina para \(d_{n_0}\), lo que contradice
la existencia de factorizaciones en \(R\).

Veamos que todo elemento irreducible de \(R[x]\) es primo.
Consideraremos primero el caso en el que nuestro irreducible es un
polinomio \(f\in R[x]\) no constante. Supongamos que \(f|gh\) en
\(R[x]\). Como \(f\) también es irreducible en \(k[x]\), que es un DFU,
entonces \(f\) es primo en \(k[x]\) y por tanto \(f|g\) o \(f|h\) en
\(k[x]\). Los tres polinomios están en \(R[x]\) y al ser \(f\)
irreducible en este anillo ha de ser primitivo, así que entonces \(f|g\)
o \(f|h\) en \(R[x]\).

Supongamos ahora que \(a\in R\subset R[x]\) es una constante irreducible
y que \(a|gh\) en \(R[x]\). Esto último equivale adecir que
\(a|\operatorname{cont}(gh)=\operatorname{cont}(g)\operatorname{cont}(h)\).
Como \(R\) es un DFU, el irreducible \(a\) es primo en \(R\), así que
\(a|\operatorname{cont}(f)\) o \(a|\operatorname{cont}(g)\), es decir,
\(a|g\) o \(a|h\). \End{proof}

\Begin{corollary}

\(R[x_1,\dots, x_n]\) es un DFU para todo \(n\geq 0\). \End{corollary}

\Begin{example}\textrm{\normalfont (El anillo $\mathbb Z[x]$)} Este
anillo es un DFU pero no es un DIP. Para comprobarlo basta ver que la
identidad de Bézout para el divisor común máximo no siempre se da. Tanto
\(2\) como \(x\) son primos en \(\mathbb Z[x]\) según criterios vistos
anteriormente. Como no son asociados, \(\operatorname{mcd}(2,x)=1\),
pero \(1\notin (2,x)\) ya que todo elemento de este ideal es de la forma
\(2g+xh\) para ciertos \(g,h\in \mathbb Z[x]\), así que su término
independiente ha de ser par. Por tanto no hay una identidad de Bézout en
este caso. El ideal \((2,x)\subset \mathbb Z[x]\) es de hecho un ejemplo
de ideal que no es principal. \End{example}

Tenemos que \(R[x]\subset k[x]\). El siguiente resultado nos permite
calcular cómo se ven los ideales del segundo dentro del primero.

\Begin{proposition}

Si \((f)\subset k[x]\) es un ideal no nulo entonces
\((f)\cap R[x]=(f_0)\), donde \(f=c\cdot f_0\) con \(c\in k\) el
contenido y \(f_0 \in R[x]\) primitivo. \End{proposition}

\Begin{proof}

La intersección \((f)\cap R[x]\) es un ideal ya que es la imagen inversa
de \((f)\subset k[x]\) a través de la inclusión
\(R[x]\hookrightarrow k[x]\). Veamos la igualdad de ideales por doble
inclusión.

\(\supset\) Como \(f_0 \in R[x]\) y
\(f_0 =c^{-1}f \in (f)\subset k[x]\), tenemos que
\(f_0 \in (f)\cap R[x]\), lo cual demuestra esta inclusión.

\(\subset\) Todo elemento \(p \in (f)\) es de la forma
\(p =g\cdot f =(g \cdot c)\cdot f_0\). Si \(p \in R[x]\), como
\(f_0 |p\) en \(k[x]\) y \(f_0\) es primitivo, \(f_0 |p\) también en
\(R[x]\), así que \(g \cdot c\in R[x]\) y por tanto
\(p \in (f_0)\subset R[x]\). \End{proof}

El siguiente resultado nos demuestra con rigor que las dos posibles
maneras de añadirle a \(R\) raíces de polinomios irreducibles dan
resultados isomorfos.

\Begin{theorem}

Dado un polinomio irreducible \(f\in R[x]\), un cuerpo \(K\) que
contiene al cuerpo de fracciones, \(k\subset K\), y una raíz
\(\alpha\in K\) de \(f\), hay un único isomorfismo
\(R[x]/(f)\cong R[\alpha]\subset K\) que se comporta sobre \(R\) como la
identidad y que envía \(\bar{x}\) a \(\alpha\). \End{theorem}

\Begin{proof}

Por el principio de sustitución, hay un único homomorfismo
\(g\colon R[x]\rightarrow K\) que se restringe a la inclusón
\(R\subset k\subset K\) sobre el dominio de coeficientes y que satisface
\(g(x)=\alpha\). La imagen de \(g\) es \(R[\alpha]\) por definición. Por
el primer teorema de isomorfía, basta probar que
\(\ker g=(f)\subset R[x]\). Para ello, consideramos la extensión
\(\bar g\colon k[x]\rightarrow K\) de \(g\) que se define como la
inclusión \(k\subset K\) sobre el cuerpo de coeficientes y que cumple
\(\bar{g}(x)=\alpha\). Veamos que \(\ker\bar{g}=(f)\subset k[x]\).

El ideal \(\ker\bar{g}\subset k[x]\) está formado por todos los
polinomios que tienen a \(\alpha\) como raíz. Al ser \(k[x]\) un dominio
euclídeo, \(\ker\bar{g}=(\tilde f)\) donde \(\tilde f\in k[x]\) es
cualquier polinomio no nulo de grado mínimo en este ideal. Realizamos la
división euclídea en \(k[x]\), \(f(x)=c(x)\tilde{f}(x)+r(x)\). Si \(r\)
no fuera trivial, su grado sería menor que el de \(\tilde{f}\), pero
\(r(x)=f(x)-c(x)\tilde{f}(x)\), por tanto
\(r(\alpha)=f(\alpha)-c(\alpha)\tilde{f}(\alpha)=0-c(\alpha)0=0\). Esto
es imposible porque \(\tilde{f}\) es de grado mínimo. Por tanto \(r=0\)
y \(f(x)=c(x)\tilde{f}(x)\). El polinomio \(c(x)\) ha de ser constante
porque \(f\) es también irreducible en \(k[x]\), así que \(f\) y
\(\tilde{f}\) son asociados, luego generan el mismo ideal,
\(\ker\bar{g}=(f)\subset k[x]\).

Como \(g=\bar{g}|_{R[x]}\),
\(\ker g=\ker\bar{g}\cap R[x]=(f)\subset R[x]\) en virtud de la
proposición anterior, pues \(f\in R[x]\), al ser irreducible, es
primitivo. Esto concluye la demostración. \End{proof}

\Begin{remark}

Recuerda que, dado un cuerpo \(k\), un polinomio \(f\in k[x]\) de grado
\(\leq 3\) es irreducible si y solo si no tiene raíces en \(k\).
\End{remark}

\Begin{example}\textrm{\normalfont (Añadiendo elementos)} El polinomio
\(x^2+1\in\mathbb{Z}[x]\) es irreducible ya que es primitivo e
irreducible en \(\mathbb{Q}[x]\) pues su grado es \(\leq 3\) y no tiene
raíces racionales. Por tanto el resultado anterior se aplica a la
inclusión \(\mathbb Q\subset\mathbb C\) y a la raíz compleja
\(i\in\mathbb C\) de \(x^2+1\) y obtenemos un isomorfismo con los
enteros de Gauss, \[
\begin{array}{rcl}
\mathbb Z[x]/(x^2+1)&\cong&\mathbb Z[i],\cr
 \bar{x}&\mapsto& i.
\end{array}
\]

Análogamente obtenemos por ejemplo \[
\begin{array}{rcl}
\mathbb Z[x]/(x^2-2)&\cong&\mathbb Z[\sqrt{2}],\cr
 \bar{x}&\mapsto& \sqrt{2}.
\end{array}
\] Vimos que todo elemento de \(\mathbb Z[x]/(x^2-2)\) se podía expresar
de manera única como \(a_1\bar x+a_0\), \(a_0,a_1\in\mathbb Z\), así
que todo elemento de \(\mathbb Z[\sqrt{2}]\) se puede escribir de manera
única como \(a_1 \sqrt{2}+a_0\), \(a_0,a_1\in\mathbb Z\).
\End{example}

Finalmente veremos un par de condiciones suficientes más avanzadas para
la irreducibilidad de un polinomio.

\Begin{proposition}

Si \(f=a_nx^n+\cdots+a_1x+a_0\in R[x]\) es un polinomio primitivo de
grado \(n>0\), \(p\in R\) es un primo que no divide \(a_n\) y la
reducción de \(f\) módulo \(p\) es irreducible en \((R/(p))[x]\),
entonces \(f\) es irreducible en \(R[x]\). \End{proposition}

\Begin{proof}

Usaremos el homomorfismo \(\phi_p\colon R[x]\rightarrow (R/(p))[x]\) de
reducción módulo \(p\) introducido en la demostración del Lema de Gauss.
En general,
\[\operatorname{grado}(\phi_p(f))\leq \operatorname{grado}(f).\] La
condición sobre \(a_n\) equivale a decir que concretamente para el
polinomio \(f\) del enunciado
\[\operatorname{grado}(\phi_p(f))= \operatorname{grado}(f).\] Reduzcamos
al absurdo. Si \(f\) fuera reducible se descompondría como producto de
dos divisores propios \(f=gh\). Como \(f\) es primitivo, ni \(g\) ni
\(h\) puede ser constante, es decir
\[\operatorname{grado}(g),\operatorname{grado}(h)>0.\] Al ser \(\phi_p\)
un homomorfismo, \[\phi_p(f)=\phi_p(g)\phi_p(h).\] Ninguna de las
desigualdades
\[\begin{array}{rcl} \operatorname{grado}(\phi_p(g))&\leq &\operatorname{grado}(g),\cr \operatorname{grado}(\phi_p(h))&\leq &\operatorname{grado}(h), \end{array}\]
puede ser estricta ya que de ser así
\[\operatorname{grado}(\phi_p(f))=\operatorname{grado}(\phi_p(g))+\operatorname{grado}(\phi_p(h))<\operatorname{grado}(g)+\operatorname{grado}(h)=\operatorname{grado}(f),\]
pero \(\operatorname{grado}(\phi_p(f))=\operatorname{grado}(f)\). Las
dos igualdades de la ecuación anterior son ciertas porque tanto \(R\)
como \(R/(p)\) son dominios, el segundo por ser \(p\) primo. Por tanto,
\[\operatorname{grado}(\phi_p(g)),\operatorname{grado}(\phi_p(h))>0\] y
tanto \(\phi_p(g)\) como \(\phi_p(h)\) serían divisores propios de
\(\phi_p(f)\), que no sería irreducible. \End{proof}

\Begin{theorem}\textrm{\normalfont (Criterio de Eisenstein)}\label{eisenstein}
Si \(f=a_nx^n+\cdots+a_1x+a_0\in R[x]\) es un polinomio primitivo de
grado \(n>0\) y \(p\in R\) es un primo tal que:

\begin{itemize}
\item
  \(p\) no divide \(a_n\),
\item
  \(p\) divide a \(a_{n-1},\dots,a_0\),
\item
  \(p^2\) no divide a \(a_0\),
\end{itemize}

entonces \(f\) es irreducible en \(R[x]\). \End{theorem}

\Begin{proof}

Esta demostración transcurre de manera exactamente igual que la anterior
hasta la última frase, que no es válida en este caso. A partir de ahí
continuamos del siguiente modo. Si \(b_0, c_0\in R\) son los términos
independientes de \(g\) y \(h\) entonces \(a_0=b_0c_0\). Como \(p|a_0\)
y \(p\) es primo, \(p|b_0\) o \(p|c_0\), pero no puede dividir a ambos a
la vez ya que \(p^2\) no divide a \(a_0\). Esto prueba que bien
\(\phi_p(g)\) o bien \(\phi_p(h)\) tiene término independiente no nulo.
Por las condiciones del enunciado, \(\phi_p(f)=\bar a_nx^n\) con
\(\bar a_n\neq 0\). Al ser \(\phi_p(f)=\phi_p(g)\phi_p(h)\) un monomio y
\(R/(p)\) es un dominio, también \(\phi_p(g)\) y \(\phi_p(h)\) han de
ser monomios. Como uno de ellos tiene término independiente no nulo,
entonces ha de tener grado \(0\), lo que contradice el cálculo al que se
llega en la última ecuación de la demostración anterior.\\
\End{proof}

\hypertarget{enteros-de-gauss}{%
\subsection{Enteros de Gauss}\label{enteros-de-gauss}}

Vamos a estudiar los primos y las factorizaciones en el anillo
\(\mathbb Z[i]\), que es un DFU y un DIP por ser un DE. En nuestros
argumentos haremos uso de la conjugación compleja, del módulo y de su
cuadrado. Recordemos que el cero es el único elemento de módulo cero y
las unidades \(\{\pm1,\pm i\}\) son los elementos de módulo \(1\).

\Begin{proposition}

Si \(\pi\in\mathbb Z[i]\) es primo entonces su conjugado \(\bar\pi\)
también. \End{proposition}

\Begin{proof}

Como la conjugación preserva productos, si \(\bar\pi|ab\) entonces
\(\pi|\bar a\bar b\) luego \(\pi|\bar a\) o \(\pi|\bar b\), es decir
\(\bar\pi|a\) o \(\bar\pi|b\). \End{proof}

Necesitaremos la siguiente observación sobre enteros primos.

\Begin{lemma}

Todo entero primo \(p\in\mathbb Z\) satisface una y solo una de las
siguientes ecuaciones:

\begin{itemize}
\item
  \(p\equiv 1\) mod \(4\).
\item
  \(p\equiv 3\) mod \(4\).
\item
  \(p=\pm2\).
\end{itemize}

\End{lemma}

\Begin{proof}

Si \(p\equiv 0\) mod \(4\) entonces \(p\) sería un múltiplo de \(4\),
con lo cual no sería primo. Si \(p\equiv 2\) mod \(4\) entonces
\(p=2+4n=2(1+2n)\) para cierto \(n\in\mathbb Z\), que solo es primo si
\(1+2n\) es invertible en \(\mathbb Z\), es decir si y solo si
\(p=\pm2\). \End{proof}

Los primos 5, 13, 17, 29, 37, 41, 53, 61\ldots{} son 1 mod 4, y 3, 7,
11, 19, 23, 31, 43, 47\ldots{} son 3 mod 4.

\Begin{exercise}

Demuestra que hay una cantidad infinita de primos que satisfacen tanto
la primera como la segunda ecuación. \End{exercise}

\Begin{proposition}

Si \(\pi\in\mathbb Z[i]\) es tal que \(|\pi|^2=p\in\mathbb Z\) es un
entero primo entonces \(\pi\) es primo en los enteros de Gauss y además
bien \(p=2\) o bien \(p\equiv 1\) mod \(4\). \End{proposition}

\Begin{proof}

Supongamos que \(\pi\) se descompone como \(\pi=z_1z_2\) en
\(\mathbb Z[i]\). Entonces tenemos la ecuación
\(|z_1|^2|z_2|^2=|\pi|^2=p\) de números enteros. Como \(p\) es primo en
los enteros, necesariamente \(|z_i|^2=1\) para algún \(i\in\{1,2\}\),
es decir, algún \(z_i\) tendría que ser una unidad. Esto prueba que el
entero de Gauss \(\pi\) es irreducible, luego primo.

Veamos la ecuación en congruencias. Si \(\pi=a+ib\) entonces
\(p=|\pi|^2=a^2+b^2\). En \(\mathbb Z/4\) los únicos elementos que son
cuadrados de otros son \(0,1\in\mathbb Z\), por tanto \(p=a^2+b^2\)
puede ser \(0\), \(1\) o \(2\) módulo \(4\). La primera posibilidad
queda descartada por el lema anterior y la tercera solo se da cuando
\(p=2\). \End{proof}

De este modo vemos que \(1+i\), \(2+i\), \(3+2i\), \(4+i\), \(5+2i\),
\(6+i\), \(5+4i\), \(7+2i\), \(6+5i\)\ldots{} son primos en los enteros
de Gauss, así como sus conjugados y asociados. En particular
\(5=(2+i)(2-i)\) es una factorización como producto de primos en
\(\mathbb Z[i]\).

\Begin{proposition}

Si \(p\in\mathbb Z\) es un entero primo tal que \(p\equiv 3\) mod \(4\)
entonces \(p\) también es primo en los enteros de Gauss.
\End{proposition}

\Begin{proof}

Supongamos por reducción al absurdo que \(p\) se descompone como
producto de divisores propios \(p=z_1z_2\), entonces tenemos la ecuación
\(p^2=|z_1|^2|z_2|^2\) en los enteros. Como ningún \(z_i\) es una
unidad, necesariamente \(|z_1|^2=|z_2|^2=p\). Como \(p\equiv 3\) mod
\(4\), esto es imposible por la proposición anterior. \End{proof}

\Begin{proposition}

Salvo asociados, \(1+i\in\mathbb Z[i]\) es el único primo cuyo módulo al
cuadrado es \(2\). \End{proposition}

\Begin{proof}

Si \(\pi=a+ib\) y \(2=|\pi|^2=a^2+b^2\) es fácil observar que
\(a^2=b^2=1\), es decir \(a=\pm1=b\). Esto nos da \[\begin{array}{rcl}
1+i,&&\cr
1-i&=&(-i)(1+i),\cr
-1+i&=&i(1+i),\cr
-1-i&=&(-1)(1+i).
\end{array}\] \End{proof}

\Begin{remark}

La factorización del \(2\) como producto de primos en \(\mathbb Z[i]\)
es \(2=(1+i)(1-i)\). Los dos primos que aparecen en esta factorización
son asociados. \End{remark}

Veamos que para el resto de enteros primos \(p\equiv 1\) mod \(4\)
también hay primos en los enteros de Gauss que lo tienen como móldulo al
cuadrado y que son de hecho los factores primos de \(p\) en
\(\mathbb Z[i]\). Para ello necesitamos resultados técnicos sobre
enteros primos.

\Begin{lemma}

Todo entero primo \(p\in\mathbb Z\) no negativo satisface la ecuación
\((p-1)!\equiv -1\) mod \(p\). \End{lemma}

\Begin{proof}

El resultado es obvio para \(p=2\). Supongamos que \(p>2\). Observemos
la definición del factorial \[(p-1)!=1\underbrace{2\cdots (p-2)}(p-1).\]
Como \(p\) es primo, \(\mathbb Z/(p)\) es un cuerpo y todo elemento no
nulo es una unidad. Ningún factor de la definición de \((p-1)!\) es
divisible por \(p\), porque es menor. Por tanto todos son unidades
módulo \(p\). En \(\mathbb Z/(p)\) las únicas unidades que son inversas
de sí mismas son \(\pm 1\) ya que estas son las únicas raíces del
polinomio \(x^2-1=(x-1)(x+1)\). El primer factor de \((p-1)!\) es \(1\)
y el último es \(p-1\equiv -1\) mod \(p\), por tanto, todos los factores
de en medio tienen una inversa diferente, que es otro elemento del mismo
producto. Dicho de otro modo, el producto de los \(p-3\) factores
centrales se puede dividir en \((p-3)/2\) pares de elementos mutuamente
inversos módulo \(p\), con lo que este producto es congruente con \(1\)
módulo \(p\), así que \((p-1)!\equiv 1(p-1)\equiv -1\) mod \(p\).
\End{proof}

\Begin{lemma}

Si \(p\in\mathbb Z\) es un entero primo tal que \(p\equiv 1\) mod \(4\)
entonces \(p|(m^2+1)\) para cierto \(m\in\mathbb Z\). \End{lemma}

\Begin{proof}

Podemos suponer sin pérdida de genralidad que \(p\geq 0\). Por el lema
anterior, basta ver que \((p-1)!\) es un cuadrado módulo \(p\). Como
\(p=4n+1\) entonces
\[\begin{array}{rcl}(p-1)!&=&1\cdot 2\cdots (4n-1)\cdot (4n)\cr &=& \underbrace{1\cdot 2\cdots (2n-1)\cdot (2n)}\cdot \underbrace{(2n+1)\cdot (2n+2)\cdots (4n-1)\cdot (4n)}.\end{array}\]
Para todo \(1\leq i\leq 2n\), en \(\mathbb Z/(p)\) el \(i\)-ésimo factor
de la primera mitad es el opuesto por el signo del \(i\)-ésimo factor de
la segunda mitad empezando por el final ya que ambos suman
\(4n+1=p\equiv 0\) mod \(p\). Por tanto, módulo \(p\),
\[\begin{array}{rcl}(p-1)!&\equiv& \underbrace{1\cdot 2\cdots (2n-1)\cdot (2n)}\cdot \underbrace{(-2n)\cdot (-2n-1)\cdots (-2)\cdot (-1)}\cr&\equiv&(-1)^{2n}\cdot 1^2\cdot 2^2\cdots (2n-1)^2 (2n)^2\cr &=&m^2\end{array}\]
para \(m=(2n)!\). \End{proof}

\Begin{remark}

En la demostración hemos visto que si \(p=4n+1\geq 0\) podemos tomar
\(m=(2n)!\), pero en general se pueden usar números más pequeños,
concretamente siempre hay un \(0{<}m{<}p\) adecuado ya que simplemente
se trata de resolver la ecuación \(x^2+1\equiv 0\) mod \(p\). Este \(m\)
es el resto de dividir \((2n)!\) por \(p\). Por ejemplo, para
\(p=13=4\cdot 3+1\), \((2\cdot 3)!=720\) pero podemos tomar \(m=5\).
\End{remark}

\Begin{proposition}

Si \(p\in\mathbb Z\) es un entero primo tal que \(p\equiv 1\) mod \(4\)
entonces \(p\) no es primo en los enteros de Gauss. \End{proposition}

\Begin{proof}

Sabemos que \(p|(m^2+1)\) para cierto \(m\in\mathbb Z\), es decir
\(p|(m+i)(m-i)\) pero \(p\) no divide a \(m\pm i\) ya que no divide a su
parte imaginaria. Por tanto \(p\) no es primo en \(\mathbb Z[i]\).
\End{proof}

\Begin{theorem}

Si \(p\in\mathbb Z\) es un entero primo tal que \(p\equiv 1\) mod \(4\)
entonces, salvo asociados, hay exactamente dos primos en los enteros de
Gauss cuyo módulo al cuadrado es \(p\). Además estos dos primos son
conjugados \(\pi,\bar\pi\in\mathbb Z[i]\). \End{theorem}

\Begin{proof}

Como \(p\) no es primo en \(\mathbb Z[i]\) podemos descomponerlo como
producto de dos divisores propios \(p=z_1z_2\). Tenemos la ecuación
\(p^2=|z_1|^2|z_2|^2\) en los enteros. Como ningún \(z_i\) es una
unidad, necesariamente \(|z_1|^2=|z_2|^2=p\). Según hemos visto antes
estos \(z_i\) son primos en \(\mathbb Z[i]\). Es más
\(z_1\bar z_1=|z_1|^2=p=z_1z_2\), por tanto \(z_2=\bar z_1\).

Llamemos \(\pi=z_1=a+ib\). Veamos que \(\pi\) y \(\bar\pi\) no son
asociados. Los asociados de \(\pi\) son \[\begin{array}{rcl}
a+ib,&&\cr
(-1)(a+ib)&=&-a-ib,\cr
i(a+ib)&=&-b+ia,\cr
(-i)(a+ib)&=&b-ia.
\end{array}\] Veamos que ninguno de estos enteros de Gauss puede ser
\(\bar\pi=a-ib\). Si fuera el primero tendríamos que \(b=0\), pero
entonces \(p=|\pi|^2=a^2\), lo cual es una contradicción. Si fuera el
segundo tendríamos que \(a=0\) y llegaríamos a la contradicción
\(p=|\pi|^2=b^2\). Si fuera el tercero tendríamos que \(a=-b\), con lo
que \(\pi=a(1-i)\), que solo es primo si \(a\) es una unidad, pero en
este caso \(p=|\pi|^2=2\not\equiv 1\) mod \(4\). Análogamiente si fuera
el último tendríamos que \(a=b\) y \(\pi=a(1+i)\), incurriendo en la
misma contradicción que en el caso anterior.

Finalmente, comprobemos no puede haber más que estos primos de Gauss y
sus asociados con módulo al cuadrado \(p\). En efecto, si
\(\pi'\in\mathbb Z[i]\) satisficiera \(p=|\pi'|^2=\pi'\bar\pi'\)
entonces como \(\pi'|p=\pi\bar\pi\) tendríamos que bien \(\pi'|\pi\) o
bien \(\pi'|\bar\pi\), es decir, como estos tres elementos son primos,
\(\pi'\) es asociado a \(\pi\) o a \(\bar\pi\). \End{proof}

\Begin{remark}

En las condiciones del enunciado anterior, la factorización de \(p\) en
\(\mathbb Z[i]\) es \(p=\pi\bar\pi\). Para \(p=5\) los dos primos de
Gauss son \(\pi=2+i\) y \(\bar\pi=2-i\). Los asociados de \(\pi\) son
\(2+i\), \(-2-i\), \(-1+2i\) y \(1-2i\). Los asociados de \(\bar\pi\)
son los conjugados de los anteriores, \(2-i\), \(-2+i\), \(-1-2i\) y
\(1+2i\). \End{remark}

\Begin{example}\textrm{\normalfont (Factores de $p\equiv 1$ mod $4$)}\label{exm:prime1mod4}
Dado un entero primo \(p\in\mathbb Z\) tal que \(p\equiv 1\) mod \(4\),
podemos hallar su factorización como producto de primos \(p=\pi\bar\pi\)
en \(\mathbb Z[i]\) del siguiente modo. Primero encontramos un
\(m\in\mathbb Z\) tal que \(p|(m^2+1)\). Hemos visto en una demostración
anterior que \(p\) no divide a \(m\pm i\), pero \(\pi|p\) y
\(p|(m^2+1)=(m+i)(m-i)\), por tanto el primo de Gauss \(\pi\) divide a
\(m+i\) o a su conjugado, y análogamente \(\bar\pi\). Deducimos que
\(\pi\) y \(\bar \pi\) son \(\operatorname{mcd}(p,m+i)\) y
\(\operatorname{mcd}(p,m-i)\).

Por ejemplo, para \(p=13\) hemos visto que podemos tomar \(m=5\).
Calculamos \(\operatorname{mcd}(13, 5+i)\), mediante el algoritmo de
Euclides. Como el módulo de \(13\) es mayor que el de \(5+i\),
comenzamos realizando la división euclídea del primero por el segundo,
\[13=(5+i)\cdot 3+(-2-3i).\] Ahora dividimos \(5+i\) por el resto de la
anterior división, \[(5+i)=(-2-3i)(-1+i)+0.\] El resto de esta división
es \(0\). El divisor común máximo es el último resto no nulo,
\[\pi=-2-3i.\] \End{example}

Hasta el momento hemos conseguido factorizar los primos enteros en
\(\mathbb{Z}[i]\) y por tanto calcular aquellos primos de Gauss que son
factores de un primo entero. Veamos que estos son todos los primos de
Gauss posibles y que por tanto hemos dado ya una descripción completa de
todos los primos en \(\mathbb Z[i]\).

\Begin{proposition}

Todo primo en \(\mathbb Z[i]\) divide a un primo en \(\mathbb Z\).
\End{proposition}

\Begin{proof}

Sea \(\pi\in\mathbb Z[i]\) un primo. Factorizamos
\(|\pi|^2\in\mathbb Z\) como producto de primos enteros
\(|\pi|^2=p_1\cdots p_n\). Como \(|\pi|^2=\pi\bar\pi\) entonces
\(\pi|p_1\cdots p_n\) así que \(\pi|p_i\) para cierto \(1\leq i\leq n\).
\End{proof}

\Begin{remark}

Recapitulando, los primos de Gauss son los siguientes, salvo asociados:

\begin{itemize}
\item
  \(1+i\).
\item
  Los primos enteros \(p\in \mathbb{Z}\), \(p > 0\), tales que
  \(p\equiv 3\) mod \(4\).
\item
  Para cada primo entero \(p\in\mathbb{Z}\), \(p > 0\), tal que
  \(p\equiv 1\) mod \(4\), dos primos de Gauss conjugados \(\pi\) y
  \(\bar{\pi}\) tales que \(p=\pi\bar{\pi}\), cuyo cálculo se ha
  explicado en un \protect\hyperlink{exm:prime1mod4}{ejemplo anterior}.
\end{itemize}

Hay una cantidad infinita de primos de Gauss tanto del segundo tipo como
del tercer tipo. En \(\mathbb{Z}[i]\), los asociados de un elemento se
obtienen multiplicándolo por las unidades \(\{\pm1,\pm i\}\).
\End{remark}

\Begin{example}\textrm{\normalfont (Factorizar un entero en $\mathbb Z[i]$)}\label{exm:integer}
Para factorizar \(n\in\mathbb{Z}\), \(n\neq 0,\pm1\), como producto de
primos en \(\mathbb{Z}[i]\), primero lo factorizamos como producto de
primos en \(\mathbb{Z}\), \(n=p_1\cdots p_r\), y luego factorizamos cada
\(p_i\in\mathbb{Z}\) como producto de primos en \(\mathbb{Z}[i]\).
Recuerda que si \(p_i\equiv 3\) mod \(4\) entonces ya es primo de Gauss,
la factorización del \(2\) como producto de primos de Gauss es
\(2=(1+i)(1-i)\), y el caso \(p_i\equiv 1\) mod \(4\) se ha tratado
\protect\hyperlink{exm:prime1mod4}{más arriba.} Por ejemplo, \[
\begin{array}{rcl}
n&=&1350\cr
&=&2\cdot 3^3\cdot 5^2\cr
&=&(1+i)\cdot(1-i)\cdot 3^3\cdot(2+i)^2\cdot(2-i)^2.
\end{array}
\] \End{example}

\Begin{definition}

Diremos que un entero de Gauss \(z=a+ib\) no tiene \emph{parte entera}
si \(a\neq 0\neq b\) y \(\operatorname{mcd}(a,b)=1\). \End{definition}

\Begin{remark}

Un entero de Gauss no tiene parte entera si y solo no es divisible por
ningún entero distinto de \(\pm1\). \End{remark}

\Begin{lemma}

Sea \(z\) un entero de Gauss sin parte entera y \(\pi\) un primo de
Gauss tal que \(|\pi|^2=p\) es un entero primo \(p\equiv 1\) mod \(4\).
Si \(\pi|z\) entonces \(\bar{\pi}\nmid z\). \End{lemma}

\Begin{proof}

Por reducción al absurdo, si también \(\bar{\pi}\mid p\) entonces
\(\operatorname{mcm}(\pi,\bar{\pi})|z\). Como \(\pi\) y \(\bar{\pi}\)
son primos no asociados,
\(\operatorname{mcm}(\pi,\bar{\pi})=\pi\bar{\pi}=p\), por tanto \(p|z\)
y \(z\) tendría parte entera. \End{proof}

\Begin{example}\textrm{\normalfont (Factorización de enteros de Gauss sin parte entera)}\label{exm:nointeger}
Sea \(z\in\mathbb{Z}[i]\) sin parte entera. Supongamos que
\(z=\pi_1\cdots\pi_r\) es su factorización como producto de primos de
Gauss. Como \(z\) no tiene parte entera, ningún \(\pi_i\) es un primo
entero \(p\equiv 3\) mod \(4\), así que \(|\pi_i|^2=2\), y en dicho caso
\(\pi_i\) es asociado de \(1+i\), o bien \(|\pi_i|^2=p\) es un primo
entero \(p\equiv 1\) mod \(4\). Es más, en este último caso ni
\(\bar{\pi}_i\) ni ninguno de sus asociados puede aparecer en la
factorización.

Por tanto, para factorizar \(z\) en \(\mathbb{Z}[i]\) podemos proceder
del siguiente modo. Primero, factorizamos \(|z|^2\) como producto de
potencias de primos enteros positivos,
\[|z|^2 = p_1^{n_1}\cdots p_s^{n_s}.\] Entonces
\[z=u\pi_1^{n_1}\cdots \pi_s^{n_s}\] donde:

\begin{itemize}
\item
  Si \(p_i=2\) entonces \(\pi_i=1+i\).
\item
  Si \(p_i\equiv 1\) mod \(4\), entonces \(\pi_i|p\). Para calcularlo,
  factorizamos \(p_i\) como producto de primos de Gauss,
  \(p_i=\pi\bar{\pi}\), según el
  \protect\hyperlink{exm:prime1mod4}{ejemplo anterior} y dividimos
  \(\frac{z}{\pi}\) en \(\mathbb{C}\). Si \(\frac{z}{\pi}\) resulta ser
  un entero de Gauss entonces \(\pi_i=\pi\), y si no
  \(\pi_i=\bar{\pi}\).
\item
  \(u\) es una unidad, \(u\in\{\pm1,\pm i\}\), que se determina a
  posteriori.
\end{itemize}

Veámoslo en el caso particular \(z=201-43i\). En este caso
\[|z|^2=201^2+43^2=42250=2\cdot 5^3 \cdot 13^2.\] Las factorizaciones de
\(5\) y de \(13\) en \(\mathbb{Z}[i]\) son \(3=(2+i)(2-i)\) y
\(13=(3+2i)(3-2i)\), por tanto \[z=u(1+i)(2\pm i)^3(3\pm 2i)^2.\] Para
determinar qué factor del \(5\) aparece realmente, dividimos \(z\) por
uno de ellos en \(\mathbb{C}\), por ejemplo \[
\begin{array}{rcl}
\frac{z}{2+i}&=&\frac{(201-43i)(2-i)}{(2+i)(2-i)}\cr
&=&\frac{359}{5}-\frac{287}{5}i.
\end{array}
\] Como no es un entero de Gauss, \(2+i\nmid z\), así que \(2-i\mid z\),
luego \[z=u(1+i)(2-i)^3(3\pm 2i)^2.\] Ahora, para hallar qué factor del
\(13\) aparece realmente, dividimos \(z\) por uno de ellos en
\(\mathbb{C}\), \[
\begin{array}{rcl}
\frac{z}{3-2i}&=&\frac{(201-43i)(3+2i)}{(3-2i)(3+2i)}\cr
&=&23+21i.
\end{array}
\] Este sí es un entero de Gauss, por tanto \(3-2i\mid z\) y
\[z=u(1+i)(2-i)^3(3- 2i)^2.\] Para hallar la unidad, calculamos el
producto de la derecha \[
(1+i)(2-i)^3(3- 2i)^2=-43-201i,
\] así que \(u=i\), \[z=i(1+i)(2+i)^3(3\pm 2i)^2.\] La unidad \(i\) se
puede incorporar a cualquier factor primo, por ejemplo al primero,
\(i(1+i)=-1+i\), y en conclusión \[z=(-1+i)(2+i)^3(3\pm 2i)^2\] es una
factorización de \(z\) como producto de primos de Gauss. \End{example}

\Begin{example}\textrm{\normalfont (Factorización en $\mathbb Z[i]$)} En
general, todo entero de Gauss \(z=a+ib\in\mathbb{Z}[i]\) se puede
descomponer como \(z=n\cdot z'\), con \(n=\operatorname{mcd}(a,b)\) y
\(z'\in\mathbb{Z}[i]\) sin parte entera. La factorización de \(z\) como
producto de primos de Gauss se obtiene multiplicando las
correspondientes factorizaciones de \(n\) y \(z'\), que se realizan
según indicamos \protect\hyperlink{exm:integer}{aquí} y
\protect\hyperlink{exm:nointeger}{aquí}.

Por ejemplo, \(z=15+45i=15(1+3i)\). Por un lado
\(n=15=3\cdot 5=3\cdot(2+i)\cdot (2-i)\). Por otro lado \(z'=1+3i\),
\(|z'|^2=1^2+3^2=10=2\cdot 5\). Por tanto \[z'=u(1+i)(2\pm i).\] Para
saber qué factor de \(5=(2+i)\cdot (2-i)\) divide a \(z'\) realizamos la
siguiente operación en \(\mathbb{C}\), \[
\begin{array}{rcl}
\frac{1+3i}{2+i}&=&\frac{(1+3i)(2-i)}{(2+i)(2-i)}\cr
&=&1+i.
\end{array}
\] Por tanto \[z'=u(1+i)(2+ i).\] De hecho, el cálculo anterior nos
demuestra que la unidad es \(u=1\), así que \[z'=(1+i)(2+ i),\] luego \[
\begin{array}{rcl}
z&=&n\cdot z'\cr
&=&3\cdot(2+i)\cdot (2-i)\cdot(1+i)\cdot (2+ i)\cr
&=&3\cdot(2+i)^2\cdot (2-i)\cdot(1+i).
\end{array}
\] \End{example}

El siguiente gráfico nos muestra la distribución de los primos cercanos
al origen en los enteros de Gauss.

\begin{figure}
\centering
\includegraphics{static/images/gaussian_primes.png}
\caption{Primos de Gauss}
\end{figure}

Puedes también usar la siguiente aplicación interactiva para explorar la
distribución de los primos de Gauss en cuadrados de diferente tamaño
centrados en el origen. Los lados del cuadrado tienen tamaño \(2n\). Los
puntos rojos son los primos de Gauss de módulo al cuadrado 2. En azul
están los que son enteros. El resto, en verde.

\hypertarget{ecuaciones-diofuxe1nticas}{%
\subsection{Ecuaciones diofánticas}\label{ecuaciones-diofuxe1nticas}}

A modo de ejemplo, vamos a estudiar aquí un par de ecuaciones
diofánticas cuyas soluciones pasan por el estudio de los enteros de
Gauss realizado anteriormente.

Al comienzo del tema de anillos nos habíamos planteado como motivación
el solucionar la ecuación diofántica \[x^2+y^2=5.\] Ahora reemplazaremos
el término independiente con un entero positivo \(>1\) cualquiera.

\Begin{theorem}

Dado \(n\geq 2\), la ecuación diofántica \[x^2+y^2=n\] tiene solución si
y solo si cualquier primo \(p\equiv 3\) mod \(4\) tiene exponente par en
la factorización de \(n\). Además, en dicho caso el número de soluciones
es finito. \End{theorem}

\Begin{proof}

La ecuación planteada equivale a encontrar los enteros de Gauss \(x+iy\)
tales que \(|x+iy|^2=n\). Si \(x+iy=\pi_1\cdots\pi_n\) es una
factorización en \(\mathbb Z[i]\) entonces
\(|x+iy|^2=|\pi_1|^2\cdots|\pi_n|^2\). Sabemos además, por la
clasificación de los primos de Gauss, que \(|\pi_i|^2\) puede ser \(2\),
un entero primo \(p\equiv 1\) mod \(4\) o \(p^2\) donde \(p\equiv 3\)
mod \(4\). Esto demuestra la necesidad de la condición del enunciado.
También la suficiencia porque, si se cumple, basta tomar \(x+iy\) como
el producto de:

\begin{itemize}
\item
  Un factor \(1+i\) por cada \(2\) que aparezca en la factorización de
  \(n\).
\item
  Si \(p\equiv 1\) mod \(4\), un factor \(\pi\) con \(|\pi|^2=p\) por
  cada factor \(p\) de \(n\).
\item
  Si \(p\equiv 3\) mod \(4\), un factor \(p\) por \emph{cada dos}
  factores \(p\) de \(n\).
\end{itemize}

El conjunto de todas las soluciones se obtiene permitiendo reemplazar
los \(\pi\) del segundo apartado por sus conjugados y tomando los
asociados de todas las soluciones particulares así obtenidas. En
particular hay un número finito de soluciones. \End{proof}

\Begin{example}\textrm{\normalfont ($x^2+y^2=1170$)} En este caso
concreto \(1170=2\cdot 3^2\cdot 5\cdot 13\). El único primo que vale
\(3\) módulo \(4\) y que aparece en esta factorización es el propio
\(3\), con exponente par, por lo que la ecuación tiene solución. Una
solución se corresponde con el entero de Gauss \(x+iy\) obtenido al
multiplicar los siguientes factores:

\begin{itemize}
\item
  \(1+i\) por aparecer un \(2\).
\item
  \(2+i\) por haber un \(5\) y \(2+3i\) por haber un \(13\).
\item
  \(3\) por el \(3^2\) que aparece.
\end{itemize}

Es decir, \[(1+i)(2+i)(2+3i)3=-21+27i.\] Otras soluciones concretas se
obtienen permitiendo reemplazar los factores del segundo apartado por
sus conjugados, \[\begin{array}{rcl}
(1+i)(2-i)(2+3i)3&=&9+33i,\cr
(1+i)(2+i)(2-3i)3&=&33+9i,\cr
(1+i)(2-i)(2-3i)3&=&27-21i.
\end{array}\] El conjunto de todas las soluciones \(x+iy\) son las
cuatro anteriores y sus asociados, 16 en total: \[
\begin{array}{rrrr}
-21+27i,&9+33i,&33+9i,&27-21i,\cr
-27-21i,&-33+9i,&-9+33i,&21+27i,\cr
21-27i,&-9-33i,&-33-9i,&-27+21i,\cr
27+21i,&33-9i,&9-33i,&-21-27i.
\end{array}
\]

\End{example}

La otra ecuación diofántica que vamos a considerar en este epígrafe es
la \textbf{ecuación de Pitágoras} \[x^2+y^2=z^2.\] Sus soluciones
positivas \(x,y,z>0\) se denominan \textbf{ternas pitagóricas} y
parametrizan los triángulos rectángulos con lados de medida entera.

\begin{figure}
\centering
\includegraphics{static/images/pythagorean.png}
\caption{Teorema de Pitágoras}
\end{figure}

Los papeles de \(x\) e \(y\) en la ecuación de Pitágoras son
intercambiables, por lo que \((x,y,z)\) es una solución si y solo si lo
es \[(y,x,z).\] Los signos de las soluciones son irrelevantes, es decir
si \((x,y,z)\) es una solución entonces también lo son
\[(\pm x,\pm y,\pm z).\] Las soluciones triviales son las de la forma
\((x,0,\pm x)\) o \((0,y,\pm y)\). Por tanto basta estudiar las ternas
pitagóricas.

No hay ternas pitagóricas con \(x\) e \(y\) impares porque en ese caso
\(x\equiv\pm 1\) e \(y\equiv \pm1\) mod \(4\), así que
\(z^2=x^2+y^2\equiv 1+1=2\) mod \(4\). Esto es imposible porque los
únicos cuadrados en \(\mathbb Z/(4)\) son \(0\) y \(1\).

Si \((x,y,z)\) es una terna pitagórica y \(d=\operatorname{mcd}(x,y)\)
entonces \(d^2|x^2\) y \(d^2|y^2\) por lo que \(d^2|z^2\). Por tanto
\(d|z\) y \[\left(\frac{x}{d},\frac{y}{d},\frac{z}{d}\right)\] es otra
terna pitagórica con \(\operatorname{mcd}(\frac{x}{d},\frac{y}{d})=1\).
En definitiva, podemos centrarnos en buscar las ternas pitagóricas
\((x,y,z)\) tales \(\operatorname{mcd}(x,y)=1\). Estas se denominan
\textbf{ternas pitagóricas primitivas}. Las que no son primitivas se
obtienen a partir de las primitivas multiplicando por enteros positivos.
En una terna pitagórica primitiva \(x\) e \(y\) no pueden ser ambos
pares. A la luz del párrafo anterior, \(x\) ha de ser par e \(y\) impar,
o viceversa, es decir, \[x\not\equiv y \mod 2.\] Podemos pues suponer
que \(x\) es impar e \(y\) es par, el resto de ternas pitagóricas
primitivas se obtendrán intercambiando la \(x\) y la \(y\).

La conexión de la ecuación de Pitágoras con los enteros de Gauss
proviene de que esta ecuación equivale a \[(x+iy)(x-iy)=z^2.\]

\Begin{lemma}

Dados \(x,y\in\mathbb Z\) no nulos, tenemos que \(x\equiv y\) mod \(2\)
\(\Leftrightarrow\) \((1+i)|(x+iy)\). \End{lemma}

\Begin{proof}

\(\Rightarrow\) Si \(x\equiv y\) mod \(2\) entonces \(x\) e \(y\) son
ambos pares o ambos impares. Si son ambos pares entonces \(2|(x+iy)\) y
ya sabemos que \((1+i)|2\) con lo que \((1+i)|(x+iy)\). Si son ambos
impares entonces \(y\pm x\) es par y tenemos la siguiente ecuación en
\(\mathbb Z[i]\),
\[x+iy=(1+i)\left(\frac{y+x}{2}+i\frac{y-x}{2}\right).\]

\(\Leftarrow\) Si \((1+i)|(x+iy)\) entonces
\[x+iy=(1+i)(x'+iy')=(x'-y')+(x'+y')i\] y por tanto
\[x=x'-y'\equiv x'+y'=y \mod 2.\]\\
\End{proof}

En el siguiente lema caracterizamos en términos de los enteros de Gauss
la condición sobre \(x\) e \(y\) que caracteriza las ternas pitagóricas
que son primitivas.

\Begin{lemma}\label{lem:mcdconj} Dados \(x,y\in\mathbb Z\) no nulos,
tenemos que \(\operatorname{mcd}(x,y)=1\) y \(x\not\equiv y\) mod \(2\)
\(\Leftrightarrow\) \(\operatorname{mcd}(x+iy,x-iy)=1\). \End{lemma}

\Begin{proof}

\(\Rightarrow\) Por reducción al absurdo. Si \(\pi\in\mathbb Z[i]\) es
un primo de Gauss tal que \(\pi|(x+iy)\) y \(\pi|(x-iy)\) entonces
\[\begin{array}{rcl}\pi\;|\;[(x+iy)+(x-iy)]&=&2x,\cr \pi\;|\;[(x+iy)-(x-iy)]&=&2yi.\end{array}\]
El primo \(\pi\) no puede dividir simultáneamente a \(x\) y a \(y\) ya
que \(\operatorname{mcd}(x,y)=1\), y el divisor común máximo de dos
enteros es el mismo calculado en \(\mathbb{Z}\) o en \(\mathbb{Z}[i]\).
De aquí deducimos que \(\pi|2\), es decir \(\pi=1+i\) (o un asociado).
Por tanto \((1+i)|(x+iy)\), así que por el lema anterior \(x\equiv y\)
mod \(2\), lo cual es una contradicción.

\(\Leftarrow\) Cualquier divisor común de \(x\) e \(y\) divide tanto a
\(x+iy\) como a \(x-iy\), por tanto \(\operatorname{mcd}(x,y)=1\).
Además \(x\not\equiv y\) mod \(2\) ya que en caso contrario, por el lema
anterior, \((1+i)|(x+iy)\) y por tanto \((1-i)|(x-iy)\). Como \(1+i\) y
\(1-i\) son asociados, ambos dividirían tanto a \(x+iy\) como a
\(x-iy\), que no podrían ser coprimos. \End{proof}

\Begin{lemma}\label{lem:DFU} En un DFU \(R\), las soluciones no nulas de
la ecuación \(xy=z^2\) tales \(\operatorname{mcd}(x,y)=1\) son, salvo
asociados, todas de la forma \(x=a^2\), \(y=b^2\) y \(z=ab\) con
\(a,b\in R\), \(\operatorname{mcd}(a,b)=1\). \End{lemma}

\Begin{proof}

Si \(z\) fuera una unidad, entonces también lo tendrían que ser \(z^2\),
\(x\) e \(y\), por tanto, salvo asociados, \(x=y=z=1=1^2\). Supongamos
ahora que \(z\) no es una unidad. Sea \(z=p_1\cdots p_n\) una
factorización como producto de primos. Como
\(xy=z^2=p_1^2\cdots p_n^2\), por la unicidad de las factorizaciones en
\(R\) los factores primos de \(z^2\) se han de repartir entre \(x\) e
\(y\), salvo asociados. Además, como \(\operatorname{mcd}(x,y)=1\), los
dos factores de cada \(p_i^2\) tienen que quedar del mismo lado, por lo
que tanto \(x\) como \(y\) son cuadrados, \(x=a^2\) e \(y=b^2\), y
\(z=ab\), de nuevo salvo asociados. Además \(\operatorname{mcd}(a,b)=1\)
porque \(1=\operatorname{mcd}(x,y)=\operatorname{mcd}(a,b)^2\).
\End{proof}

\Begin{theorem}

Las ternas pitagóricas primitivas con segunda coordenada par son las de
la forma \((a^2-b^2, 2ab, a^2+b^2)\) con \(a,b\in\mathbb Z\), \(a>b>0\),
\(\operatorname{mcd}(a,b)=1\), \(a\not\equiv b\) mod \(2\).
\End{theorem}

\Begin{proof}

La ecuación de Pitágoras, vista en \(\mathbb Z[i]\), es
\[(x+iy)(x-iy)=z^2.\] Según hemos visto
\protect\hyperlink{lem:mcdconj}{mas arriba}, la condición de que una
terna pitagórica sea primitiva equivale a
\(\operatorname{mcd}(x+iy,x-iy)=1\). Por el
\protect\hyperlink{lem:DFU}{lema anterior}, tanto \(x+iy\) como \(x-iy\)
son cuadrados, o asociados de cuadrados, necesariamente conjugados. Es
decir, \(x+iy=u(a+ib)^2\) para cierto \(u=1,-1,i,-i\). Esto da lugar a
las siguientes posibilidades: \[
(x,y)=\left\{\begin{array}{l}
(a^2-b^2,2ab),\cr
(b^2-a^2,-2ab),\cr
(-2ab,a^2-b^2),\cr
(2ab,b^2-a^2).
\end{array}\right.
\] Las dos últimas no dan lugar al tipo de terna pitagórica primitiva
que estamos considerando, pues \(x\) sería par. La primera sí, siempre
que \(a>b>0\), pues \(x,y>0\). También si \$ a \textless{} b \textless{}
0\$, pero por este camino llegaríamos a las mismas ternas. La segunda
posibilidad también da lugar a las mismas ternas que la primera. Así que
podemos suponer que \(x+iy=(a+ib)^2\) con \(a>b>0\) y por tanto
\(x-iy=(a-ib)^2\).

De nuevo por el \protect\hyperlink{lem:DFU}{lema anterior},
\[z=u(a+ib)(a-ib)=u(a^2+b^2)\] para cierta unidad
\(u\in\{\pm1,\pm i\}\). Como la ecuación \((x+iy)(x-iy)=z^2\) ha de
satisfacerse, la unidad ha de ser tal que \(u^2=1\), con lo que
\(u=\pm 1\). El caso \(u=-1\) lo excluimos ya que entonces
\(z=-(a^2+b^2)<0\), por tanto \(z=a^2+b^2\). Además, una vez más por el
\protect\hyperlink{lem:DFU}{lema anterior},
\(\operatorname{mcd}(a+ib,a-ib)=1\), es decir
\(\operatorname{mcd}(a,b)=1\) y \(a\not\equiv b\) mod \(2\), usando el
\protect\hyperlink{lem:mcdconj}{lema de más arriba}. Esto reduce todas
las posibilidades a las que aparecen en el enunciado del teorema. Veamos
que todas ellas son en efecto ternas pitagóricas primitivas.

Claramente, las ternas del enunciado resuelven la ecuación de Pitágoras,
es decir, \[(a^2-b^2)^2+(2ab)^2=(a^2+b^2)^2.\] Las tres coordenadas son
positivas, pues \(a>b>0\). La segunda coordenada es claramente par. Solo
queda comprobar que las dos primeras son coprimas. Para ello,
demostraremos que la primera no es divisible por \(2\) ni por ningún
factor primo de \(a\) o de \(b\). Como \(n\equiv n^2\) mod \(2\) para
todo \(n\in\mathbb{Z}\), \(a^2-b^2\equiv a-b\not\equiv 0\) mod \(2\),
pues \(a\not\equiv b\) mod \(2\), así que \(a^2-b^2\) es impar. Sea
\(p\in\mathbb{Z}\) un primo. Si \(p\mid a\) entonces \(p\mid a^2\). Si
fuera cierto que \(p\mid (a^2-b^2)\) entonces tendriamos que
\(p\mid b^2\) y por tanto \(p\mid b\). Esto contradeciría que
\(\operatorname{mcd}(a,b)=1\), luego en realidad \(p\nmid (a^2-b^2)\).
Análogamente se prueba que si \(p\mid b\) entonces \(p\nmid (a^2-b^2)\).
Esto concluye la demostración. \End{proof}

El siguiente gráfico muestra los pares \((x,y)\) que forman parte de una
terna pitagórica cualquiera con \(x,y\leq 4500\).

\begin{figure}
\centering
\includegraphics{static/images/Pythagorean_triple_scatterplot.png}
\caption{Ternas pitagóricas}
\end{figure}

La siguiente aplicación muestra los pares \((x,y)\) que forman parte de
una terna pitagórica primitiva con \(x\) impar y \(x,y\leq n\), donde
\(n\) puede ser cualquier múltiplo de \(10\) comprendido entre \(10\) y
\(3000\).
