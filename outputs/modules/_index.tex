
¿Cómo es el álgebra lineal que resulta al reemplazar el papel de los
cuerpos por anillos más generales? El objeto de estudio de esta álgebra
lineal generalizada son los módulos. Un módulo \(M\) es a un anillo
\(R\) lo que un espacio vectorial es a un cuerpo. Es decir, el módulo
\(M\) está dotado de las siguientes operaciones:

\begin{itemize}
\item
  Suma.
\item
  Resta.
\item
  Producto por escalares de \(R\).
\end{itemize}

Estas operaciones deben satisfacer las propiedades habituales. Además el
módulo ha de contener el siguiente elemento:

\begin{itemize}
\tightlist
\item
  Cero \(0\).
\end{itemize}

Tanto este \(0\in M\) como el \(1\in R\) han de satisfacer las
propiedades habituales con respecto a la suma y la multiplicación.

Los módulos sobre el anillo \(\mathbb Z\) son simplemente los grupos
abelianos. Los espacios vectoriales sobre un cuerpo cualquiera están
clasificados por su dimensión. Es decir, dos espacios vectoriales son
isomorfos si y solo si tienen la misma dimensión. En este tema
estudiaremos fundamentalmente la clasificación de los módulos
finitamente generados sobre un dominio de ideales principales. En
particular la clasificación de los grupos abelianos finitamente
generados. Aplicaremos estos resultados a la resolución de sistemas de
ecuaciones lineales diofánticas y al estudio de un tema de álgebra
lineal clásica: los endomorfismos de espacios vectoriales de dimensión
finita.

\hypertarget{definiciuxf3n}{%
\subsection{Definición}\label{definiciuxf3n}}

\Begin{definition}

Dado un anillo \(R\), un \textbf{\(R\)-módulo} es un conjunto \(M\)
equipado con dos aplicaciones, llamadas \emph{suma} y \emph{producto por
escalares}, \[
\begin{array}{ccc}
M\times M\rightarrow M, &\qquad& R\times M\rightarrow M,\cr
(a,b)\mapsto a+b;&&(r,a) \mapsto ra;
\end{array}
\] que satisfacen las siguientes propiedades, donde \(a,b,c\in M\) y
\(r,s\in R\):

\begin{itemize}
\item
  Asociativa: \[a+(b+c)=(a+b)+c,\qquad r(sa)=(rs)a.\]
\item
  Conmutativa: \[a+b=b+a.\]
\item
  Distributiva: \[r(a+b)=ra+rb,\qquad (r+s)a=ra+sa.\]
\item
  Existencia de \emph{elemento neutro} \(0\in M\) para la suma y
  comportamiento del \(1\in R\) como elemento neutro para el producto
  por escalares: \[0+a=a,\qquad 1a=a.\]
\item
  Existencia de un \emph{elemento opuesto} para la suma \(-a\in M\) para
  todo \(a\in M\) de modo que \[a+(-a)=0.\]
\end{itemize}

\End{definition}

\Begin{remark}

Cuando quede claro por el contexto o no sea relevante especificarlo,
omitiremos \(R\) de la notación. La suma en un módulo lo dota de
estructura de grupo abeliano. Recordemos que el elemento neutro de un
grupo es único, no puede haber dos distintos que satisfagan la misma
propiedad. Los opuestos para la suma también. Restar es sumar el
elemento opuesto \(a-b=a+(-b)\). Además \(0a=0=r0\) y
\(r(-a)=-ra=(-r)a\). Si \(R=k\) es un cuerpo la noción de \(R\)-módulo
coincide con la de \(k\)-espacio vectorial. \End{remark}

\Begin{example}\textrm{\normalfont (El módulo trivial)}

El conjunto unitario \(\{0\}\), dotado de las únicas operaciones suma
y producto por escalares posibles, es un módulo sobre cualquier anillo
\(R\).

\End{example}

\Begin{example}\textrm{\normalfont ($R^n$)} Sus elementos son vectores
conlumna con entradas en \(R\), aunque a veces, cuando convenga, se
denotarán por filas. La suma se define coordenada a coordenada, y el
producto por escalares se define multiplicando el escalar por todas las
coordenadas:
\[\left(\begin{array}{c}a_1\cr\vdots\cr a_n\end{array}\right)+\left(\begin{array}{c}b_1\cr\vdots\cr b_n\end{array}\right)=\left(\begin{array}{c}a_1+b_1\cr\vdots\cr a_n+b_n\end{array}\right),\qquad r\left(\begin{array}{c}a_1\cr\vdots\cr a_n\end{array}\right)=\left(\begin{array}{c}ra_1\cr\vdots\cr ra_n\end{array}\right).\]
Para \(n=1\) observamos que el propio \(R\) puede considerarse como un
\(R\)-módulo. Para \(n=0\) obtenemos el módulo trivial. \End{example}

\Begin{example}\textrm{\normalfont (Producto de módulos)} Dados dos
\(R\)-módulos \(M\) y \(N\), su \textbf{producto} cartesiano
\(M\times N\) posee una estructura de \(R\)-módulo con las siguientes
operaciones:
\[\begin{array}{rcl}(a_1,b_1)+(a_2,b_2)&=&(a_1+a_2,b_1+b_2),\cr r(a,b)&=&(ra,rb).\end{array}\]
¿Cuál es el elemento neutro para la suma? ¿Cuáles son los opuestos?
\End{example}

\Begin{proposition}

Todo grupo abeliano \(A\) posee una única estructura de
\(\mathbb Z\)-módulo. \End{proposition}

\Begin{proof}

Dado \(a\in A\), si \(n>0\) en \(\mathbb Z\), estaríamos forzados a
definir: \[\begin{array}{rcl}
n a&=&(1+\stackrel{n}{\cdots}+1)a\cr&=&1a+\stackrel{n}{\cdots}+1a\cr&=&a+\stackrel{n}{\cdots}+a,\cr
(-n)a&=&-na,\cr
0a&=&0.
\end{array}\] Es fácil comprobar que estas fórmulas definen una
estructra de \(\mathbb Z\)-módulo en \(A\), necesariamente única.\\
\End{proof}

\Begin{definition}

Un subconjunto \(N\subset M\) de un \(R\)-módulo \(M\) es un
\textbf{submódulo} si:

\begin{itemize}
\item
  \(0\in N\).
\item
  \(a+b\in N\) para todo \(a,b\in N\).
\item
  \(-a\in N\) para todo \(a\in N\).
\item
  \(ra\in N\) para todo \(r\in R\) y \(a\in N\).
\end{itemize}

\End{definition}

\Begin{remark}

Un submódulo \(N\subset M\) es un módulo por derecho propio con la suma
y el producto por escalares heredados de \(M\). El módulo trivial
\(\{0\}\) es un submódulo de cualquier otro. Los submódulos de \(R\)
coinciden con los ideales del anillo. \End{remark}

\hypertarget{homomorfismos}{%
\subsection{Homomorfismos}\label{homomorfismos}}

Los homomorfismos de módulos son aplicaciones que preservan la
estructura, es decir, la suma y el producto por escalares.

\Begin{definition}

Dados dos \(R\)-módulos \(M\) y \(N\), un \textbf{homomorfismo}
\(f\colon M\rightarrow N\) es una aplicación tal que, para todo
\(r\in R\) y \(a,b\in M\),
\[\begin{array}{rcl} f(a+b)&=&f(a)+f(b),\cr f(ra)&=&rf(a).\end{array}\]
Un \textbf{isomorfismo} es un homomorfismo biyectivo. Un
\textbf{automorfismo} es un isomorfismo de un \(R\)-módulo en sí mismo.
\End{definition}

\Begin{remark}

Los homomorfismos satisfacen \(f(-a)=-f(a)\) y \(f(0)=0\). La identidad
\(\operatorname{id}_M\colon M\rightarrow M\) es un isomorfismo.
Comprueba que si
\[M\stackrel{f}\longrightarrow N\stackrel{g}\longrightarrow P\] son
homomorfismos entonces la composición \(g\circ f\colon M\rightarrow P\)
también. Lo mismo es cierto para isomorfismos. Es más, demuestra que si
\(f\colon M\rightarrow N\) es un isomorfismo entonces su aplicación
inversa \(f^{-1}\colon N\rightarrow M\) también. El símbolo \(\cong\) se
usará para denotar la relación de ser isomorfos \(M\cong N\). Prueba que
esta relación es de equivalencia. \End{remark}

\Begin{example}\textrm{\normalfont (La inclusión)} Si \(M\) es un módulo
y \(P\subset M\) es un submódulo, la \textbf{inclusión}
\(i\colon P\hookrightarrow M\), \(i(a)=a\), es un homomorfismo. ¿Qué
diferencia a la inclusión de la identidad? \End{example}

\Begin{example}\textrm{\normalfont (Objeto cero)} Todo módulo \(M\)
admite homomorfismos únicos desde \(\{0\}\rightarrow M\) y hasta
\(M\rightarrow \{0\}\) el módulo trivial. \End{example}

\Begin{example}\textrm{\normalfont (Homomorfismo trivial)} Dados dos
\(R\)-módulos cualesquiera \(M\) y \(N\), la aplicación
\(M\rightarrow N\) definida como \(x\mapsto 0\) para todo \(x\in M\) es
un homomorfismo, el \textbf{homomorfismo trivial}. ¿Puede ser un
isomorfismo? \End{example}

\Begin{example}\textrm{\normalfont (Producto por escalares)} Dado un
\(R\)-módulo \(M\) y \(r\in R\), la aplicación
\(M\stackrel{r\cdot}\rightarrow M\) definida como \(x\mapsto r\cdot x\)
es un homomorfismo. ¿Puede ser un isomorfismo? \End{example}

\Begin{example}\textrm{\normalfont (Inclusiones y proyecciones de un producto)}
Dados dos \(R\)-módulos \(M\) y \(N\), definimos dos homomorfismos de
\textbf{inclusión}
\[i_1\colon M\longrightarrow M\times N,\qquad i_2\colon N\longrightarrow M\times N,\]
mediante las fórmulas \[i_1(a)=(a,0),\qquad i_2(b)=(0,b),\] y otros
dos de \textbf{proyección}
\[p_1\colon M\times N\longrightarrow M,\qquad p_2\colon M\times N\longrightarrow N,\]
como \[p_1(a,b)=a,\qquad p_2(a,b)=b.\] Comprueba que
\(p_1\circ i_1=\operatorname{id}_M\) y
\(p_2\circ i_2=\operatorname{id}_N\). ¿Son estos homomorfismos
isomorfismos? \End{example}

\Begin{example}\textrm{\normalfont (Conmutatividad del producto)} El
producto de \(R\)-módulos es conmutativo salvo isomorfismo. Dados dos
\(R\)-módulos \(M\) y \(N\) tenemos un isomorfismo
\[\begin{array}{rcl}M\times N&\stackrel{\cong}\longrightarrow&N\times M,\cr (a,b)&\mapsto&(b,a).\end{array}\]
¿Cuál es su inverso? \End{example}

\Begin{example}\textrm{\normalfont (Matrices)} Toda matriz \(B\) de
tamaño \(m\times n\) con entradas en \(R\) da lugar a un homomorfismo
definido por la multiplicación de matrices:
\[\begin{array}{rcl} R^n&\stackrel{B\cdot}\longrightarrow&R^m,\cr \left(\begin{smallmatrix}a_1\cr\vdots\cr a_n\end{smallmatrix}\right)&\mapsto& B\left(\begin{smallmatrix}a_1\cr\vdots\cr a_n\end{smallmatrix}\right). \end{array}\]
La composición de este tipo de homomorfismos es el producto de matrices.
En particular \(B\) define un isomorfismo si y solo si es una matriz
invertible. Cualquier homomorfismo \(f\colon R^n\rightarrow R^m\) es de
este tipo. En efecto, si para cada \(1\leq i\leq n\) consideramos el
elemento
\[e_i=\left(\begin{smallmatrix}0\cr\vdots\cr1\cr\vdots\cr0\end{smallmatrix}\right)\in R^n\]
cuya única coordenada no trivial es la \(i\)-ésima, que vale \(1\),
puedes comprobar la matriz que define \(f\colon R^n\rightarrow R^m\) es
aquella cuyas columnas son los \(f(e_i)\),
\[\left(f(e_1)\mid\cdots\mid f(e_n)\right).\] \End{example}

\Begin{exercise}

Demuestra que si \(M\cong M'\) y \(N\cong N'\) entonces
\(M\times N\cong M'\times N'\). \End{exercise}

\Begin{proposition}

Si \(M\) es un \(R\)-módulo y \(S=\{a_1,\dots,a_n\}\subset M\) un
subconjunto cualquiera, existe un único homomorfismo
\(\phi=\phi_{S}\colon R^n\rightarrow M\) tal quie \(\phi(e_i)=a_i\).
\End{proposition}

\Begin{proof}

Todo \(x=(x_1,\dots,x_n)\in R^n\) se puede escribir como
\(x=\sum_{i=1}^nx_ie_i\). Por tanto, si existiera \(\phi\) tendría
que cumplir que \[
\begin{array}{rcl}
\phi(x)&=&\phi(\sum_{i=1}^nx_ie_i)\cr
&=&\sum_{i=1}^nx_i\phi(e_i)\cr
&=&\sum_{i=1}^nx_ia_i.
\end{array}
\] Esto demostraría la unicidad, caso de que existiera. Es más, es fácil
comprobar que la fórmula \(\phi(x)=\sum_{i=1}^nx_ia_i\) define un
homomorfismo, luego existe. \End{proof}

\Begin{proposition}

Dado un homomorfismo \(f\colon M\rightarrow N\), su \textbf{imagen}
\(\operatorname{im} f\subset N\) es un submódulo. \End{proposition}

\Begin{proof}

\begin{itemize}
\item
  \(0=f(0)\in \operatorname{im} f\).
\item
  \(f(a)+f(b)=f(a+b)\in \operatorname{im} f\) para \(a,b\in M\).
\item
  \(-f(a)=f(-a)\in \operatorname{im} f\) para todo \(a\in M\).
\item
  \(rf(a)=f(ra)\in \operatorname{im} f\) para todo \(r\in R\) y
  \(a\in M\).
\end{itemize}

\End{proof}

La siguiente proposición posee un análogo para anillos que
\href{/estalg/rings/definitions/\#factorimage}{ya ha sido demostrado}.

\Begin{proposition}\label{factorimagemodules} Dado un homomorfismo
\(f\colon M\rightarrow N\) y un submódulo \(U\subset N\), si
\(\operatorname{im} f\subset U\) entonces \(f\) factoriza de manera
única a través de la inclusión, es decir, existe un único homomorfismo
\(g\colon M\rightarrow U\) tal que \(f=i\circ g\),
\[f\colon M\stackrel{g}\rightarrow U\stackrel{i}\hookrightarrow N.\]
\End{proposition}

\Begin{proof}

Si \(f=i\circ g\) entonces tendríamos
\[f(a)=(i\circ g)(a)=i(g(a))=g(a).\] La unicidad de \(g\) sería
consecuencia de esta fórmula ya que fuerza su definición. Definimos pues
la aplicación \(g\colon M\rightarrow U\) como \(g(a)=f(a)\). Tiene
sentido porque \(\operatorname{im}f\subset U\). La aplicación \(g\) es
un homomorfismo pues está definida por la misma fórmula que el
homomorfismo \(f\).\\
\End{proof}

\Begin{remark}

En la proposición anterior podemos siempre tomar
\(U=\operatorname{im} f\). \End{remark}

\Begin{proposition}

El \textbf{núcleo} de un homomorfismo \(f\colon M\rightarrow N\),
\[\ker f=\{a\in M\mid f(a)=0\},\] es un submódulo \(\ker f\subset M\).
\End{proposition}

\Begin{proof}

\begin{itemize}
\item
  \(0\in\ker f\) pues \(f(0)=0\).
\item
  Si \(a,b\in\ker f\) entonces \(a+b\in \ker f\) puesto que
  \(f(a+b)=f(a)+f(b)=0+0=0\).
\item
  Si \(a\in\ker f\) entonces \(-a\in \ker f\) puesto que
  \(f(-a)=-f(a)=0\).
\item
  Si \(a\in\ker f\) y \(r\in R\) entonces \(ra\in \ker f\) pues
  \(f(ra)=rf(a)=r0=0\).
\end{itemize}

\End{proof}

\Begin{remark}

Como ocurre con los grupos y con los anillos, un homomorfismo de módulos
\(f\colon M\rightarrow N\) es inyectivo si y solo si \(\ker f=\{0\}\).
De otro modo, la inyectividad de \(f\) equivale a que si \(a\in M\) es
tal que \(f(a)=0\) entonces \(a=0\). \End{remark}

\hypertarget{muxf3dulos-libres}{%
\subsection{Módulos libres}\label{muxf3dulos-libres}}

\Begin{definition}

Sea \(M\) un \(R\)-módulo y \(S=\{a_1,\dots,a_n\}\subset M\) un
subconjunto. Decimos que \(S\) \textbf{genera} \(M\) si todo elemento de
\(x\in M\) es \textbf{combinación lineal} de \(S\), es decir, de la
forma \[x=r_1a_1+\cdots+r_na_n\] para ciertos
\(r_1,\dots,r_n\in R\). Decimos que \(S\) es \textbf{linealmente
independiente} si la única combinación lineal de \(S\) que da como
resultado \(0\) es aquella que tiene todos los cieficientes nulos, es
decir si \(r_1,\dots,r_n\in R\) son tales que
\[r_1a_1+\cdots+r_na_n=0\] entonces \(r_1=\cdots=r_n=0\). Decimos
además que \(S\) es una \textbf{base} de \(M\) si lo genera y es
linealmente independiente. Un \(R\)-módulo es \textbf{finitamente
generado} si posee un subconjunto finito que genera, y es \textbf{libre}
si posee una base. \End{definition}

\Begin{remark}

Si \(R=k\) es un cuerpo todo \(R\)-módulo es libre puesto que todo
\(k\)-espacio vectorial posee una base. Los \(k\)-espacios vectoriales
finitamente generados se denominan también de dimensión finita. Es
posible considerar bases de módulos no finitamente generados, pero aquí
no lo haremos. \End{remark}

\Begin{example}\textrm{\normalfont ($R^n$ es libre)} El sub conjunto
\(\{e_1,\dots,e_n\}\subset R^n\) es una base denominada
\textbf{canónica}. \End{example}

Podemos definir un submódulo con un conjunto prefijado de generadores al
igual que lo hicimos con los
\href{/estalg/rings/definitions/\#generators}{ideales}.

\Begin{definition}

El \textbf{submódulo generado por} un conjunto finto de elementos
\(a_1,\dots,a_n\in M\) está formado por todas las combinaciones lineales
de los generadores con coeficientes en el anillo:
\[(a_1,\dots,a_n)=\{r_1a_1+\dots+r_na_n\;|\; r_1,\dots,r_n\in R\}\subset M.\]
Un \textbf{módulo cíclico} es uno que está generado por un único
elemento \((a)=\{ra\,|\, r\in R\}\) y que por tanto está formado por
sus múltiplos. \End{definition}

\Begin{exercise}

Comprueba que \((a_1,\dots,a_n)\subset M\) es en efecto un submódulo.
Observa que \(a_1,\dots,a_n\in (a_1,\dots, a_n)\). Es más, demuestra que
si \(N\subset M\) es un submódulo y \(a_1,\dots,a_n\in N\) entonces
\((a_1,\dots,a_n)\subset N\). Intenta dar una definición razonable de
ideal generado por un conjunto infinito de elementos que satisfaga las
propiedades análogas al caso finito. \End{exercise}

\Begin{remark}

En términos del homomorfismo \[\phi_S\colon R^n\longrightarrow M\]
definido antes, podemos afirmar lo siguiente sobre el conjunto
\(S=\{a_1,\dots,a_n\}\subset M\):

\begin{itemize}
\tightlist
\item
  \(S\) genera si y solo si \(\phi_S\) es sobreyectivo.
\end{itemize}

En particular, \((a_1,\dots,a_n)=\operatorname{im} \phi_S\).

\begin{itemize}
\item
  \(S\) es linealmente independiente si y solo si \(\phi_S\) es
  inyectivo.
\item
  \(S\) es una base si y solo si \(\phi_S\) es un isomorfismo.
\end{itemize}

En particular, si \(S\) es una base de \(M\) entonces para todo
\(x\in M\) existe un único \((r_1,\dots,r_n)\in R^n\) tal que
\[x=\phi_S(r_1,\dots,r_n)=r_1a_1+\cdots+r_na_n.\] Decimos
entonces que \((r_1,\dots,r_n)\) son las \textbf{coordenadas} de \(x\)
respecto de \(S\). Es más, la aplicación que envía cada elemento a sus
coordenadas respecto de \(S\) es
\[\phi_S^{-1}\colon M\longrightarrow R^n,\] que es un isomorfismo, por
tanto la asignación de coordenadas respecto de una base preserva sumas y
productos por escalares. \End{remark}

\Begin{remark}

Sea \(f\colon M\rightarrow N\) es un homomorfismo de \(R\)-módulos y
\(S=\{a_1,\dots,a_n\}\subset M\). Los siguientes enunciados son
consecuencia de las observaciones anteriores:

\begin{itemize}
\item
  Si \(S\subset M\) genera y \(f\) es sobreyectivo \(\Rightarrow\)
  \(f(S)\subset N\) genera.
\item
  Si \(S\subset M\) es linealmente independiente y \(f\) es inyectivo
  \(\Rightarrow\) \(f(S)\subset N\) es linealmente independiente.
\item
  Si \(S\subset M\) es una base y \(f\) es biyectivo \(\Rightarrow\)
  \(f(S)\subset N\) es una base.
\item
  Si \(S\subset M\) genera entonces \(f(S)\subset\operatorname{im}f\)
  genera.
\end{itemize}

Concluimos que un \(R\)-módulo es libre si y solo si es isomorfo a algún
\(R^n\). \End{remark}

Seguidamente definimos la noción de determinante para matrices sobre un
anillo igual que se hacía para los cuerpos.

\Begin{definition}

El \textbf{determinante} \(|A|\) de una matriz cuadrada \(A=(a_{ij})\)
de tamaño \(n\times n\) con entradas en un anillo conmutativo \(R\) se
define como
\[|A|=\sum_{\sigma\in S_n}\operatorname{signo}(\sigma)a_{1\sigma(1)}\cdots a_{n\sigma(n)}.\]
Aquí \(S_n\) denota el grupo de permutaciones de \(n\) elementos.
\End{definition}

\Begin{remark}

Los determinantes de matrices con entradas en un anillo conmutativo
satisfacen las siguientes propiedades habituales. Puedes comprobarlas
como ejercicio.

\begin{itemize}
\item
  El determinante de la matriz identidad es \(|I|=1\).
\item
  Si \(A\) tiene una fila de ceros entonces \(|A|=0\).
\item
  Una matriz \(A\) y su traspuesta \(A^t\) tienen el mismo determinante,
  \(|A|=|A^t|\).
\item
  El determinante preserva productos, \(|AB|=|A||B|\).
\item
  Las fórmulas del desarrollo de un determinante por los adjuntos de una
  fila o columna también son válidas en este contexto.
\item
  Si \(A\) es una matriz cuadrada invertible entonces
  \(|A|\in R^{\times}\) es una unidad y \(|A^{-1}|=|A|^{-1}\).
\item
  Recíprocamente, si \(A\) es cuadrada y \(|A|\in R^{\times}\) es una
  unidad entonces \(A\) es invertible. Su inversa es la matriz
  traspuesta de la adjunta de \(A\) dividida por \(|A|\).
\end{itemize}

Las matrices invertibles son necesariamente cuadradas si \(R\) no es el
anillo trivial. Esto se comprueba por reducción al absurdo. En efecto,
sean \(A\) y \(B\) matrices tales que \(AB=I\) y \(BA=I\). Como las
matrices identidad \(I\) son cuadradas, si \(A\) tiene tamaño
\(m\times n\) entonces el tamaño de \(B\) tiene que ser \(n\times m\).
Podemos suponer sin pérdida de generalidad que \(m>n\). Completamos la
columnas de \(A\) y las filas de \(B\) con ceros hasta formar matrices
cuadradas y observamos que
\[\left(\begin{array}{c|c}A&0\end{array}\right)\left(\begin{array}{c}B\cr \hline 0\end{array}\right)=AB+0=AB=I,\]
pero esto es imposible porque los determinantes de las matrices de la
izquierda son \(0\), pues contienen alguna fila o columna de ceros, pero
el determinante de \(I\) es \(1\). \End{remark}

\Begin{proposition}

Todas las bases de un mismo \(R\)-módulo libre \(M\) tiene el mismo
número de elementos. \End{proposition}

\Begin{proof}

Si \(S\) y \(S'\) son bases con \(n\) y \(m\) elementos,
respectivamente, cada una de ellas define un isomorfismo
\(\phi_S\colon R^n\rightarrow M\),
\(\phi_{S'}\colon R^m\rightarrow M\). Componiendo el primero con el
inverso del segundo obtenemos un isomorfismo
\(\phi_{S'}^{-1}\circ\phi_S\colon R^n\rightarrow R^m\). Este
isomorfismo tiene que estar definido por una matriz \(A\) de tamaño
\(m\times n\) invertible. Como las matrices invertibles son cuadradas
deducimos que \(m=n\). \End{proof}

\Begin{definition}

El \textbf{rango} de un \(R\)-módulo libre \(M\) es el número de
elementos de una base. \End{definition}

Cuando \(R=k\) es un cuerpo, el rango de un \(k\)-espacio vectorial se
denomina dimensión.

\Begin{remark}

Si \(S=\{a_1,\dots,a_n\}\) y \(S'=\{a_1',\dots,a_n'\}\) son
bases de un mismo \(R\)-módulo libre de rango \(n\), el isomorfismo
\(\phi_{S'}^{-1}\circ\phi_S\colon R^n\rightarrow R^n\) considerado en
la demostración anterior está definido por una matriz \(B=(b_{ij})\)
invertible \(n\times n\) sobre \(R\), que es la única que satisface las
siguientes ecuaciones para todo \(1\leq i\leq n\),
\[a_i=b_{1i}a_1'+\cdots+b_{ni}a_n'.\] Es decir, las columnas de
\(B\) son las coordenadas de los elementos de \(S\) respecto de la base
\(S'\). Si \(x\in M\) tiene coordenadas \((r_1,\dots, r_n)\) respecto
de \(S\) y \((r_1',\dots, r_n')\) respecto de \(S'\) entonces se
satisface que
\[B\left(\begin{smallmatrix}r_1\cr\vdots\cr r_n\end{smallmatrix}\right)=\left(\begin{smallmatrix}r_1'\cr\vdots\cr r_n'\end{smallmatrix}\right).\]
Por eso \(B\) se denomina \textbf{matriz de cambio de base} de \(S\) a
\(S'\). \End{remark}

\Begin{example}\textrm{\normalfont (No todos los módulos son libres)} El
\(\mathbb Z\)-módulo \(\mathbb Z/(2)\) está generado por el conjunto
\(S=\{\bar 1\}\) pero \(S\) no es linealmente independiente porque
\(2\cdot\bar 1=\bar 2=\bar 0\) y \(0\neq 2\in\mathbb Z\). De hecho el
\(\mathbb Z\)-módulo \(\mathbb Z/(2)\) no puede tener ninguna base ya
que los subconjuntos de \(\mathbb Z/(2)\) son \(\varnothing\),
\(\{\bar 0\}\), \(\{\bar 1\}\) y \(\{\bar 0,\bar 1\}\), los dos
primeros no generan y los dos últimos no son linealmente independientes.
Este argumento, por elemental, es algo complejo. Es más sencillo
observar que los \(\mathbb Z\)-módulos libres poseen un único elemento,
\(\mathbb Z^0=\{0\}\), o infinitos, \(\mathbb Z^n\) con \(n>0\), por
tanto \(\mathbb Z/2\), que tiene dos elementos, no puede ser libre.
\End{example}

\hypertarget{torsiuxf3n}{%
\subsection{Torsión}\label{torsiuxf3n}}

En este apartado \(R\) denotará siempre un dominio.

\Begin{definition}

Dado un \(R\)-módulo \(M\), decimos que \(a\in M\) es un elemento de
\textbf{torsion} si existe algún \(r\in R\) no nulo, \(r\neq 0\), tal
que \(ra=0\). \End{definition}

\Begin{remark}

El \(0\in M\) es siempre un elemento de torsión ya que \(1\neq 0\) y
\(1\cdot 0 = 0\). Dado un entero no nulo \(0\neq n\in\mathbb Z\), todo
elemento \(\bar a\) del \(\mathbb Z\)-módulo \(\mathbb Z/(n)\) es de
torsión puesto que \(n\bar a=\overline{na}=\bar 0\). Más generalmente,
si \(I\subset R\) es un ideal no nulo entonces todo elemento de \(R/I\)
es de torsión, pues existe \(0\neq a\in I\) y dado \(\bar r\in R/I\),
\(a\bar r=\overline{ar}=0\). Como \(R\) no tiene divisores de cero,
\(T( {R} )=\{0\}\), y más generalmente \(T(R^n)=\{0\}\),
\(n\geq 0\). \End{remark}

Veamos una condición suficiente, aunque no necesaria, para que un módulo
no sea libre.

\Begin{proposition}

Si \(M\) es un \(R\)-módulo que tiene elementos no triviales de torsión
entonces \(M\) no es libre. \End{proposition}

\Begin{proof}

Supongamos por reducción al absurdo que
\(\{a_1,\dots,a_n\}\subset M\) fuera una base. Tomamos un elemento
no trivial de torsión \(0\neq x\in M\) y lo escribimos como
\[x=r_1a_1+\cdots+r_na_n\] con \(r_1,\dots,r_n\in R\). Ha de haber
algún \(r_i\neq 0\) para cierto \(1\leq i\leq n\) ya que \(x\neq 0\).
Como \(x\in M\) es de torsión existe \(0\neq s\in R\) tal que
\[0=sx=sr_1a_1+\cdots+sr_na_n.\] Uno de los coeficientes de esta
combinación lineal es \(sr_i\neq 0\) que es no nulo porque \(R\) es un
dominio. Esto contradice la independencia lineal. \End{proof}

\Begin{remark}

El \(\mathbb Z\)-módulo \(\mathbb Q\) no tiene torsión, pero se puede
comprobar que no es libre, es decir, no posee niguna base, ni finita ni
infinita. \End{remark}

Los elementos de torsión forman un submódulo. \Begin{proposition} Si
\(M\) es un \(R\)-módulo, el subconjunto \(T(M)\subset M\) formado por
los elementos de torsión es un submódulo. \End{proposition}

\Begin{proof}

Dados \(a,b\in T(M)\) existen \(s,t\in R\) no nulos, \(s\neq 0\neq t\),
tales que \(sa=0=tb\).

\begin{itemize}
\item
  \(0\in T(M)\) tal como hemos visto antes.
\item
  \(a+b\in T(M)\) pues \(st\neq 0\) y \(st(a+b)=t(sa)+s(tb)=0\).
\item
  \(-a\in T(M)\) pues \(s(-a)=-sa=0\).
\item
  Dado \(r\in R\), \(ra\in T(M)\) pues \(s(ra)=r(sa)=0\).
\end{itemize}

\End{proof}

Los homomorfismos preservan la torsión.

\Begin{proposition}\label{homotorsion} Si \(f\colon M\rightarrow N\) es
un homomorfismo de \(R\)-módulos entonces \(f(T(M))\subset T(N)\). Si
\(f\) es un isomorfismo entonces \(f(T(M))= T(N)\). \End{proposition}

\Begin{proof}

Dado \(a\in T(M)\) existe \(0\neq r\in R\) tal que \(ra=0\) luego
\(rf(a)=f(ra)=f(0)=0\) y por tanto \(f(a)\) es de torsión. Esto prueba
la inclusión. Si \(f\) además es un isomorfismo entonces podemos
aplicarle la parte ya probada a \(f^{-1}\colon N\rightarrow M\), con lo
que tenemos \(f^{-1}(T(N))\subset T(M)\), lo cual equivale a la otra
inclusión \(T(N)\subset f(T(M))\). \End{proof}

\Begin{corollary}\label{torsionquotient} Si dos módulos son isomorfos
\(M\cong N\) entonces sus submódulos de torsión también
\(T(M)\cong T(N)\), y los correspondientes cocientes
\(M/T(M)\cong N/T(N)\). \End{corollary}

La torsión preserva productos.

\Begin{proposition}\label{torsionproduct} Dados dos \(R\)-módulos \(M\)
y \(N\), tenemos que \(T(M\times N)=T(M)\times T(N)\) .
\End{proposition}

\Begin{proof}

Veamos primero \(\subset\). Si \((a,b)\in T(M\times N)\) existe
\(0\neq r\in R\) tal que \(r(a,b)=(ra,rb)=(0,0)\), es decir \(ra=0\) y
\(rb=0\), por lo que \(a\in T(M)\) y \(b\in T(N)\), o dicho de otro modo
\((a,b)\in T(M)\times T(N)\).

Veamos ahora \(\supset\). Si \((a,b)\in T(M)\times T(N)\), es decir
\(a\in T(M)\) y \(b\in T(N)\), entonces existen \(r,s\in R\) no nulos
tales que \(ra=0\) y \(sb=0\), por tanto \(rs\neq 0\) y
\(rs(a,b)=(s(ra),r(sb))=(0,0)\), luego \((a,b)\in T(M\times N)\).
\End{proof}

\hypertarget{cocientes}{%
\subsection{Cocientes}\label{cocientes}}

\Begin{definition}

Dado un \(R\)-módulo \(M\) y un submódulo \(N\subset M\), el
\textbf{\(R\)-módulo cociente} \(M/N\) es el cociente de los grupos
abelianos subyacentes dotado del producto por escalares
\[r(a+N)=(ra)+N.\] \End{definition}

\Begin{remark}

Recordemos que \(M/N=\{a+N\,|\, a\in M\}\) de modo que \(a+N=b+N\) si
y solo si \(a-b\in N\). En particular \(a+N=0+N\) si y solo si
\(a\in N\). El elemento \(a+N\) del cociente se denomina \textbf{clase}
de \(a\) \textbf{módulo} \(N\). Cuando el submódulo \(N\) se
sobreentiende se escribe simplemente \[a+N=\bar a=[a].\] La suma en el
cociente se define como \((a+N)+(b+N)=(a+b)+N\). El cero en el cociente
es \(0+N\). Comprueba que \(M/M\) es el módulo trivial y
\(M/\{0\}\cong M\). Si \(R\) es un dominio y \(0\neq x\in R\) todo
elemento del \(R\)-módulo cociente \(\bar a\in R/(x)\) es de torsión
pues \(x\bar a=\overline{xa}=0\). \End{remark}

\Begin{theorem}

El \(R\)-módulo cociente \(M/N\) está bien definido. Su estructura es la
única que hace que la \textbf{proyección natural}
\(p\colon M\twoheadrightarrow M/N\), \(p(a)=a+N\), sea un homomorfismo.
El núcleo de esta proyección es \(\ker p=N\). \End{theorem}

\Begin{proof}

Para ver que la producto por escalares está bien definido hay que
comprobar que \[
a+N=a'+N\Rightarrow(ra)+N=(ra')+N.
\] Esto equivale a \[
a-a'\in N\Rightarrow ra-ra'= r(a-a')\in N.
\] Las propiedades que la suma y el producto por escalares deben
satisfacer se cumplen obviamente pues se derivan de las correspondientes
propiedades de \(M\).

Probemos la unicidad. Si \(p\colon M\rightarrow M/N\) es un homomorfismo
entonces \[\begin{array}{rcl}
p(a+b)&=&p(a)+p(b),\cr p(ra)&=&rp(a),
\end{array}\] lo cual equivale a \[\begin{array}{rcl}
(a+b)+N&=&(a+N)+(b+N),\cr (ra)+N&=&r(a+N).
\end{array}\]

El núcleo de la proyección natural es
\[\ker p =\{a\in M\;|\; p(a)=0\},\] pero \(p(a)=a+N\) y \(a+N=0+N\) si
y solo si \(a\in N\), luego \(\ker p=N\).\\
\End{proof}

\Begin{example}\textrm{\normalfont ($R[x]/(p(x))$ como $R$-módulo)} Sea
\(R\) un anillo y \(p(x)\in R[x]\) un polinomio mónico de grado \(n\).
El \(R[x]\)-módulo cociente \(R[x]/(p(x))\) es también un \(R\)-módulo,
restringiendo el producto por escalares al subanillo \(R\subset R[x]\).
Hemos \href{static/rings/definitions/\#uniquerep}{visto} que todo elemento
del cociente está representado por un único polinomio de grado \(<n\).
Es decir, todo elemento de \(R[x]/(p(x))\) se puede escribir como
combinación lineal de \(S=\{1,\bar{x},\dots,\bar{x}^{n-1}\}\) de
manera única. Por tanto, \(R[x]/(p(x))\) es libre como \(R\)-módulo y
\(S\) es una base. Recuerda que, sin embargo, cuando estudiamos la
torsión vimos que, si \(R\) es un dominio, \(R[x]/(p(x))\) no es libre
como \(R[x]\)-módulo. \End{example}

La siguiente proposición también tiene un análogo para anillos
\href{/estalg/rings/definitions/\#factorquotient}{ya demostrado}.

\Begin{proposition}\label{factorquotientmodules} Dado un submódulo
\(N\subset M\) y un homomorfismo \(f\colon M\rightarrow P\) tal que
\(N\subset \ker f\), \(f\) factoriza de manera única a través de la
proyección natural, es decir existe un único homomorfismo
\(g\colon M/N\rightarrow P\) tal que \(f=g\circ p\),
\[f\colon M\stackrel{p}\twoheadrightarrow M/N\stackrel{g}\rightarrow P.\]
\End{proposition}

\Begin{proof}

Si \(f=g\circ p\) entonces tendríamos
\[f(a)=(g\circ p)(a)=g(p(a))=g(a+N).\] Definimos la aplicación
\(g\colon M/N\rightarrow P\) como \[g(a+N)=f(a).\] Veamos que en efecto
está bien definida. La unicidad se seguirá de la primera fórmula.

Si \(a+N=a'+N\) entonces \(a-a'\in N\subset\ker f\) luego
\[0=f(a-a')=f(a)-f(a').\] Por tanto \[g(a+N)=f(a)=f(a')=g(a'+N).\]
Claramente \(g\) es un homomorfismo pues se definie como el homomorfismo
\(f\) en los representantes.\\
\End{proof}

\Begin{remark}

En la proposición anterior podemos tomar siempre \(N=\ker f\).
\End{remark}

El siguiente resultado es una versión para módulos del primer teorema de
isomoría, \href{/estalg/rings/definitions/\#primer}{ya visto para
anillos}.

\Begin{theorem}\textrm{\normalfont (Primer Teorema de Isomorfía)}\label{isomodules}
Dado un homomorfismo \(f\colon M\rightarrow N\), existe un único
homomorfismo \(\bar f\colon M/\ker f\rightarrow \operatorname{im}f\) tal
que \(f\) factoriza como \(f=i\circ\bar f\circ p\), es decir, \(f\)
encaja en el siguente \textbf{diagrama conmutativo},

\begin{figure}
\centering
\includegraphics{static/images/isomorfiamodulos.png}
\caption{Primer teorema de isomorfía}
\end{figure}

Aquí \(p\) es la proyección natural e \(i\) es la inclusión. Además
\(\bar f\) es un isomorfismo. \End{theorem}

\Begin{proof}

\protect\hyperlink{factorimagemodules}{Por un lado} podemos factorizar
\(f\\colon M\\rightarrow N\) de manera única como \(f=i\circ g\),
\[f\colon M\stackrel{g}\rightarrow \operatorname{im} f\stackrel{i}\hookrightarrow N,\]
donde \(g(a)=f(a)\). En particular \[\ker g = \ker f.\]

\protect\hyperlink{factorquotientmodules}{Por otro lado} podemos
factorizar \(g\colon M\rightarrow \operatorname{im} f\) de manera única
como \(g=\bar f\circ p\),
\[g\colon M\stackrel{p}\twoheadrightarrow N/\ker f\stackrel{\overline{f}}\rightarrow \operatorname{im} f,\]
donde \(\bar f(\bar{a})=g(a)=f(a)\).

Por tanto \(f=i\circ g= i\circ(\overline{f}\circ i)\), como queríamos.
La unicidad de \(\bar f\) se deduce de esta fórmula, ya que fuerza su
definición: \[
\begin{array}{rcl}
f(a)&=&(i\circ\bar f\circ p)(a)\cr
&=&i(\bar{f}(p(a)))\cr
&=&i(\bar{f}(\bar{a}))\cr
\bar{f}(\bar{a}).
\end{array}
\]

Veamos que \(\bar f\colon M/\ker f\rightarrow \operatorname{im} f\) es
un isomorfismo. Comenzamos probando que es inyectivo. Sea
\(\bar{a}\in M/\ker f\) tal que \(\bar f(\bar{a})=0\). Como
\(\bar f(\bar{a})=f(a)\), deducimos que \(a\in \ker f\), por lo que
\(\bar{a}=\bar{0}\).

Veamos que \(\bar f\colon M/\ker f\rightarrow \operatorname{im} f\) es
sobreyectiva. Dado \(b\in\operatorname{im} f\) existe \(a\in M\) tal que
\(f(a)=b\). Por tanto \(\bar f(\bar{a})=f(a)=b\).\\
\End{proof}

\Begin{example}\textrm{\normalfont (El cociente de un producto por un factor)}
Recordemos que dados dos \(R\)-módulos \(M\) y \(N\) podemos considerar
el \(R\)-módulos producto \(M\times N\) y los homomorfismos de inclusión
\(i_1\colon M\rightarrow M\times N\), \(i_1(a)=(a,0)\), y proyección
\(p_2\colon M\times N\rightarrow N\), \(p_2(a,b)=b\). El primero es
inyectivo y el segundo sobreyectivo. Claramente
\(\operatorname{im}i_1=M\times\{0\}=\ker p_2\), por tanto el teorema
anterior nos proporciona isomorfismos \[\begin{array}{rclrcl}
\frac{M\times N}{M\times\{0\}}&\stackrel{\cong}\longrightarrow&N,&\qquad M&\stackrel{\cong}\longrightarrow&M\times\{0\},\cr 
[(a,b)]&\mapsto& b,&a&\mapsto&(a,0).\end{array}\] Además, el
homomorfismo inyectivo \(i_1\) induce un isomorfismo
\(M\cong M\times\{0\}\). Análogamente podemos obtener isomorfismos
\[\begin{array}{rclrcl}
\frac{M\times N}{\{0\}\times N}&\stackrel{\cong}\longrightarrow&M,&\qquad N&\stackrel{\cong}\longrightarrow&\{0\}\times N,\cr 
[(a,b)]&\mapsto& a,&b&\mapsto&(0,b).\end{array}\] \End{example}

Recordemos que un \(R\)-módulo es \emph{cíclico} si se puede generar por
un solo elemento.

\Begin{proposition}

Un \(R\)-módulo \(M\) es cíclico \(\Leftrightarrow\) \(M\cong R/I\) para
algún ideal \(I\subset R\). \End{proposition}

\Begin{proof}

\(\Leftarrow\) El módulo \(R/I\) es cíclico pues
\(\{\\bar 1\}\subset R/I\) genera, así que cualquier módulo isomorfo a
\(R/I\) será también cíclico.

\(\Rightarrow\) Sea \(\{a\}\subset M\) un generador. El homomorfismo
\(\phi_{\{a\}}\colon R\rightarrow M\) que envía \(1\mapsto a\) es por
tanto sobreyectivo, así que por el primer
\protect\hyperlink{isomodules}{teorema} de isomorfía,
\(R/\ker \phi_{\{a\}}\cong M\), con lo que podemos tomar
\(I=\ker \phi_{\{a\}}\).\\
\End{proof}

\Begin{proposition}

Dado un \(R\)-módulo \(M\) y un submódulo \(N\subset M\), si \(N\) y
\(M/N\) son finitamente generados entonces \(M\) también lo es.
\End{proposition}

\Begin{proof}

Sean \(\{\bar a_1,\dots,\bar a_p\}\subset M/N\) y
\(\{b_1,\dots,b_q\}\subset N\) conjuntos de generadores. Veamos que
\(S=\{ a_1,\dots, a_p,b_1,\dots,b_q\}\subset M\) genera. Dado
\(x\in M\), consideramos \(\bar x\in M/N\) y lo escribimos como
combinación lineal de los generadores de \(M/N\) con coeficientes en
\(R\) \[\begin{array}{rcl}\bar x&=&r_1\bar a_1+\cdots+r_p\bar a_p\cr
&=&\overline{r_1a_1+\cdots+r_pa_p}.
\end{array}\] Tenemos entonces que \[x-(r_1a_1+\cdots+r_pa_p)\in N\]
así que lo podemos escribir como combinación lineal de los geberadores
de \(N\) con coeficientes en \(R\),
\[x-(r_1a_1+\cdots+r_pa_p)=s_1b_1+\cdots+s_qb_q.\] Despejando
vemos que \(x\) es combinación lineal de \(S\), y por tanto \(S\) genera
\(M\).\\
\End{proof}

\Begin{remark}

Si \(M\) es finitamente generado, \(M/N\) también lo es, porque si
\(\{a_1,\dots, a_n\}\subset M\) genera entonces
\(\{\bar{a}_1,\dots, \bar{a}_n\}\) genera \(M/N\). En cambio, en
general, \(N\) podría no ser finitamente generado. \End{remark}

\Begin{corollary}

Dado un dominio de ideales principales \(R\), un \(R\)-módulo libre
finitamente generado \(M\) y un submódulo \(N\subset M\), el
\(R\)-módulo \(N\) es finitamente generado. \End{corollary}

\Begin{proof}

Por inducción en el rango de \(M\), que denotamos \(n\). Para \(n=0\) es
obvio ya que en este caso \(M=\{0\}\) tendría que ser el módulo
trivial, que no tiene más submódulos que él mismo.

Sea ahora \(n>0\) y supongamos el resultado cierto para \(R\)-módulos
libres de rango \(n-1\). Como \(M\cong R^n\), basta probar el resultado
para \(R^n\). Observamos que \(R^n=R^{n-1}\times R\) y consideramos el
homomorfismo de proyección sobre la última coordenada
\(p=p_2\colon R^n\rightarrow R\), cuyo núcleo es
\(R^{n-1}\times \{0\}\cong R^{n-1}\), que es libre de rango \(n-1\).
Sea \(N\subset R^n\) un submódulo y \(p_{|_{N}}\colon N\rightarrow R\)
la restricción del homomorfismo anterior. Como
\[\ker p_{|_{N}}=N\cap \ker p= N\cap (R^{n-1}\times \{0\})\subset R^{n-1}\times \{0\},\]
deducimos que \(\ker p_{|_{N}}\) es finitamente generado por hipótesis
de inducción. Es más, por el primer
\protect\hyperlink{isomodules}{teorema} de isomorfía aplicado a
\(p_{|_{N}}\),
\[\frac{N}{\ker p_{|_{N}}}\cong \operatorname{im} p_{|_{N}}.\] Más
aún, \(\operatorname{im} p_{|_{N}}\subset R\) es un submódulo, por
tanto un ideal, y \(R\) es un dominio de ideales principales, así que
\(\operatorname{im} p_{|_{N}}\) es finitamente generado (por un solo
elemento). Ahora podemos deducir, haciendo uso de la proposición
anterior, que \(N\) es finitamente generado. \End{proof}

\hypertarget{generadores-y-relaciones}{%
\subsection{Generadores y relaciones}\label{generadores-y-relaciones}}

Supongamos que deseamos construir un \(\mathbb Z\)-módulo \(M\) generado
por tres elementos \(\{a_1,a_2,a_3\}\subset M\) que satisfagan las
siguientes ecuaciones (relaciones): \[\begin{array}{rcrcrcl}
3a_1&+&2a_2&+&a_3&=&0,\cr
8a_1&+&4a_2&+&2a_3&=&0,\cr
7a_1&+&6a_2&+&2a_3&=&0,\cr
9a_1&+&6a_2&+&a_3&=&0.
\end{array}\] ¿Cómo hacerlo? Las relaciones anteriores pueden ser
codificadas en una matriz que tiene por columnas a los coeficientes de
cada una de las ecuaciones, \[A=\left(\begin{array}{cccc}
3&8&7&9\cr
2&4&6&6\cr
1&2&2&1
\end{array}\right).\] Esta matriz define un homomorfismo
\[\mathbb{Z}^4\stackrel{A}\longrightarrow \mathbb{Z}^3\] Veamos que el
cociente \[M=\frac{\mathbb{Z}^3}{\operatorname{im}A}\] satisface las
condiciones deseadas. En efecto, está generado por las clases de los
elementos de la base canónica de \(\mathbb{Z}^3\), \[\begin{array}{rcl}
a_1&=&\bar{e}_1,\cr
a_2&=&\bar{e}_2,\cr
a_3&=&\bar{e}_3.
\end{array}\] Además \(\operatorname{im}A\) está generado por las
imágenes de los elementos de la base canónica de \(\mathbb{Z}^4\),
\[\begin{array}{rcrcrcr}
Ae_1&=&3e_1&+&2e_2&+&e_3,\cr
Ae_2&=&8e_1&+&4e_2&+&2e_3,\cr
Ae_3&=&7e_1&+&6e_2&+&2e_3,\cr
Ae_4&=&9e_1&+&6e_2&+&e_3,
\end{array}\] cuyas clases en el cociente se anulan, lo cual equivale a
las ecuaciones del principio. Más aún, esta construcción es universal ya
que, por la \protect\hyperlink{factorquotientmodules}{proposición} de
factorización de homomofismos a través de cocientes, dado un
\(\mathbb Z\)-módulo \(N\) y tres elementos
\(\{b_1,b_2,b_3\}\subset N\) que satisfacen las ecuaciones
\[\begin{array}{rcrcrcl}
3b_1&+&2b_2&+&b_3&=&0,\cr
8b_1&+&4b_2&+&2b_3&=&0,\cr
7b_1&+&6b_2&+&2b_3&=&0,\cr
9b_1&+&6b_2&+&b_3&=&0,
\end{array}\] existe un único homomorfismo
\[g\colon M\longrightarrow N\] que satisface \[\begin{array}{rcl}
g(a_1)&=&b_1,\cr
g(a_2)&=&b_2,\cr
g(a_3)&=&b_3.
\end{array}\] Este homomorfismo es la factorización de
\(\phi_{\{b_1,b_2,b_3\}}\colon\mathbb{Z}^3\rightarrow N\) a través
de la proyección natural al cociente
\(M=\mathbb{Z}^3/\operatorname{im}A\). Tendríamos que comprobar que las
hipótesis de la \protect\hyperlink{factorquotientmodules}{proposición}
mencionada se cumplen, es decir que
\(\operatorname{im} A\subset \ker \phi_{\{b_1,b_2,b_3\}}\). Como
\(\operatorname{im} A\) está generado por
\(\{Ae_1,Ae_2,Ae_3,Ae_4\}\), basta ver que
\(Ae_i\in \ker \phi_{\{b_1,b_2,b_3\}}\) para todo \(i\), es decir
que \(\phi_{\{b_1,b_2,b_3\}}(Ae_i)=0\), \(i=1,2,3,4\). Esto
equivale a las cuatro ecuaciones anteriores para los \(b_i\).

Esta construcción se puede generalizar de manera obvia del siguiente
modo. Dado un anillo cualquiera \(R\), queremos construir un
\(R\)-módulo \(M\) con \emph{generadores}
\(\{a_1,\dots,a_m\}\subset M\) donde se satisfagan las ecuaciones
(\emph{relaciones})
\[r_{1j}a_1+\cdots+r_{mj}a_m=0,\quad 1\leq j\leq n,\] donde
\(r_{ij}\in R\), \(1\leq i\leq m\), \(1\leq j\leq n\). Estas relaciones
están determinadas por la matriz \(A=(r_{ij})\), que define un
homomorfismo \[R^n\stackrel{A}\longrightarrow R^m.\] Podemos tomar
\[M=\frac{R^m}{\operatorname{im}A}\] ya que las clases de los elementos
de la base canónica de \(R^m\) generan \(M\),
\[a_i=\bar{e}_i,\quad 1\leq i\leq m,\] y las imágenes de los elementos
de la base canónica de \(R^n\) generan \(\operatorname{im}A\),
\[Ae_j=r_{1j}e_1+\cdots+r_{mj}e_m,\quad 1\leq j\leq n.\] Estas
imágenes se anulan en el cociente, lo cual equivale a las ecuaciones
(relaciones) del principio. Esta construcción es universal en virtud de
la \protect\hyperlink{factorquotientmodules}{proposición} de
factorización de homomorfismos a través de cocientes, ya que dado un
\(R\)-módulo \(N\) y elementos \(\{b_1,\dots,b_m\}\subset N\) que
satisfacen \[r_{1j}b_1+\cdots+r_{mj}b_m=0,\quad 1\leq j\leq n,\]
existe un único homomorfismo \[g\colon M\longrightarrow N\] que
satisface \[g(a_i)=b_i,\quad 1\leq i\leq m,\] que es la factorización
de \(\phi_{\{b_1,\dots,b_m\}}\colon R^m\rightarrow N\) a través de
la proyección natural al cociente \(M=R^m/\operatorname{im}A\). Las
hipótesis de la \protect\hyperlink{factorquotientmodules}{proposición}
mencionada se cumplen porque
\(\operatorname{im}A\subset \ker \phi_{\{b_1,\dots,b_m\}}\) ya que
\(\operatorname{im}A\) está generado por los \(Ae_i\) y
\(\phi_{\{b_1,\dots,b_m\}}(Ae_i)=0\) debido a que los \(b_i\)
satisfacen las mismas ecuaciones (relaciones) que los \(a_i\).

\Begin{definition}

Una \textbf{presentación} de un \(R\)-módulo \(M\) consiste en dos
homomorfismos
\[R^n\stackrel{A}\longrightarrow R^m\stackrel{f}\twoheadrightarrow M\]
tales que \(f\) es sobreyectivo e \(\operatorname{im} A=\ker f\). Esto,
en virtud del primer \protect\hyperlink{isomodules}{teorema} de
isomorfía, equivale a dar una matriz \(A\) y un isomorfismo
\[\bar{f}\colon \frac{R^m}{\operatorname{im} A}\stackrel{\cong}\longrightarrow M.\]
Decimos que un módulo es \textbf{finitamente presentado} si admite una
presentación. \End{definition}

\Begin{remark}

En las condiciones de la definición, el \(R\)-módulo \(M\) está generado
por \(\{f(e_1),\dots,f(e_m)\}\subset M\), y estos generadores
satisfacen las relaciones determinadas por la matriz \(A\). \End{remark}

\Begin{proposition}\label{fgfp} Dado un dominio de ideales principales
\(R\), todo \(R\)-módulo finitamente generado \(M\) admite una
presentación. \End{proposition}

\Begin{proof}

Sea \(S=\{a_1,\dots,a_m\}\subset M\) un conjunto de generadores. Por
serlo, el homomorfismo \(\phi_S\colon R^m\rightarrow M\) es
sobreyectivo, así que \[\frac{R^m}{\ker\phi_{S}}\cong M.\] Según hemos
visto anteriormente, el submódulo \(\ker \phi_S\subset R^n\) es
finitamente generado. Escogemos un conjunto de generadores
\(S'=\{b_1,\dots,b_n\}\subset \ker \phi_{S}\), que por tanto
inducen otro homomorfismo sobreyectivo
\(\phi_{S'}\colon R^n\rightarrow \ker \phi_{S}\). Consideramos su
composición con la inclusión,
\[A\colon R^n\stackrel{\phi_{S'}}\twoheadrightarrow \ker \phi_S\hookrightarrow R^m,\]
que estará definida por una matriz \(A\). Al ser \(\phi_{S'}\)
sobreyectiva, \(\operatorname{im}A=\ker \phi_{S}\), con lo que \(A\) es
una presentación de \(M\). \End{proof}

Una presentación de un módulo se puede modificar y simplificar de los
siguientes modos.

\Begin{proposition}\label{simplify} Si el \(R\)-módulo \(M\) está
presentado por la matriz \(A\) de tamaño \(m\times n\) entonces también
está presentado por la matriz \(A'\) en los siguientes casos:

\begin{itemize}
\item
  Si \(A'=QAP^{-1}\) siendo \(P\) y \(Q\) matrices invertibles.
\item
  Si \(A'\) se obtienen a partir de \(A\) eliminando una columna de
  ceros, \[
  A=\left(
  \begin{array}{cccc}
  &&0&\cr
  &&\vdots&\cr
  &&0&
  \end{array}
  \right).
  \]
\item
  Si la \(j\)-ésima columna de \(A\) es \(ue_i\), donde
  \(u\in R^\times\) es una unidad, y \(A'\) se obtiene borrando la
  \(i\)-ésima fila y la \(j\)-ésima columna de \(A\), \[
  A=\left(
  \begin{array}{cccc}
  &&0&\cr
  &&\vdots&\cr
  \cdots&\cdots&u&\cdots\cr
  &&\vdots&\cr
  &&0&\cr
  \end{array}
  \right).
  \]
\end{itemize}

\End{proposition}

\Begin{proof}

\begin{itemize}
\item
  Las matrices invertibles \(P\) y \(Q\) se corresponden con meros
  cambios de base en \(R^m\) y \(R^n\), respectivamente, con lo cual
  tenemos un isomorfismo \[\begin{array}{rcl}
  R^m/\operatorname{im}A&\stackrel{\cong}\longrightarrow& R^m/\operatorname{im}A'\cr [x]&\mapsto& [Qx].\end{array}\]
  Usando el primer \protect\hyperlink{isomodules}{teorema} de isomorfía,
  podemos comprobar que esta aplicación está en efecto bien definida y
  es un isomorfismo.
\item
  Una columna de ceros se corresponde con la relación \(0=0\), que no
  aporta nada, con lo cual puede eliminarse.
\item
  En este caso la \(j\)-ésima columna se corresponde con la relación
  \(ua_i=0\), que equivale a \(a_i=0\), pues \(u\) es una unidad, así
  que podemos simplemente eliminar \(a_i\) de la lista de generadores y
  \(a_i=0\) de la de relaciones. Esto se corresponde con la eliminación
  de la \(i\)-ésima fila y la \(j\)-ésima columna de \(A\).
\end{itemize}

\End{proof}

Cuando una matriz es especialmente sencilla resulta fácil identificar el
módulo que presenta.

\Begin{proposition}\label{easy} El \(R\)-módulo
\(R^m/\operatorname{im}D\) presentado por la matriz
\[D=\left( \begin{array}{ccc} d_1&&\cr &\ddots&\cr &&d_n\cr \hline &0& \end{array} \right)\]
de tamaño \(m\times n\) con una caja superior diagonal de tamaño
\(n\times n\) y una caja inferior trivial de tamaño \((m-n)\times n\),
es isomorfo a
\[\frac{R}{(d_1)}\times \cdots \times\frac{R}{(d_n)}\times R^{m-n}.\]
\End{proposition}

\Begin{proof}

Es obvio, ya que este módulo está generado por
\(\{a_1,\dots,a_m\}\), donde \(a_i=\bar e_i\), y las relaciones
correspondientes a la matriz \(D\) son
\[d_ia_i=0,\quad 1\leq i\leq n.\] Por tanto las únicas relaciones
existentes nos dicen que hemos de considerar la \(i\)-ésima coordenada
módulo \((d_i)\), \(1\leq i\leq n\). El isomorfismo está simplemente
definido por \[
\begin{array}{rcl}
\frac{R^m}{\operatorname{im} D}
&\stackrel{\cong}\longrightarrow&
\frac{R}{(d_1)}\times \cdots \times\frac{R}{(d_n)}\times R^{m-n},\cr
{[(a_1,\dots,a_m)]}&\mapsto&(\bar{a}_1,\dots, \bar{a}_n,a_{n+1},\dots,a_m).
\end{array}
\] \End{proof}

\hypertarget{forma-normal-de-smith}{%
\subsection{Forma normal de Smith}\label{forma-normal-de-smith}}

En esta sección veremos cómo la matriz de una presentación de un módulo
se puede simplificar mediante operaciones elementales.

\Begin{definition}

Las \textbf{operaciones elementales por filas} para matrices con
entradas en un anillo \(R\) son las siguientes:

\begin{enumerate}
\def\labelenumi{\arabic{enumi}.}
\item
  Añadirle a una fila un múltiplo de otra, \(F_i+rF_j\), \(i\neq j\),
  \(r\in R\).
\item
  Intercambiar dos filas \(F_i\leftrightarrow F_j\), \(i\neq j\).
\item
  Multiplicar una fila por una unidad \(u\in R^\times\), \(uF_i\) .
\end{enumerate}

Las \textbf{operaciones elementales por columnas} se definen
análogamente.

\End{definition}

Las operaciones elementales anteriores se corresponden con el producto
por los siguientes tipos de matrices.

\Begin{definition}

Las \textbf{matrices elementales} son las que se obtienen a partir de la
identidad realizando una de las operaciones elementales por filas
anteriores. Concretamente:

\begin{enumerate}
\def\labelenumi{\arabic{enumi}.}
\item
  \(E_{ij}( r )=\left(\begin{array}{ccccccc}1&&&&&&\cr &\ddots&&&&&\cr &&1&\cdots&r&&\cr &&&\ddots&\vdots&&\cr &&&&1&&\cr &&&&&\ddots&\cr &&&&&&1\end{array}\right)\).
\item
  \(E_{ij}=\left(\begin{array}{ccccccc}1&&&&&&\cr &\ddots&&&&&\cr &&0&\cdots&1&&\cr &&\vdots&\ddots&\vdots&&\cr &&1&\cdots&0&&\cr &&&&&\ddots&\cr &&&&&&1\end{array}\right)\).
\item
  \(E_{ii}(u)=\left(\begin{array}{ccccccc}1&&&&&&\cr &\ddots&&&&&\cr &&1&&&&\cr &&&u&&&\cr &&&&1&&\cr &&&&&\ddots&\cr &&&&&&1\end{array}\right)\).
\end{enumerate}

\End{definition}

\Begin{remark}

Las matrices elementales son invertibles, concretamente:

\begin{enumerate}
\def\labelenumi{\arabic{enumi}.}
\item
  \(E_{ij}( r )^{-1}=E_{ij}(-r)\).
\item
  \(E_{ij}^{-1}=E_{ij}\).
\item
  \(E_{ii}(u)^{-1}=E_{ii}(u^{-1})\).
\end{enumerate}

Sus determinantes son \[\begin{array}{rcl}
|E_{ij}( r )|&=&1,\cr
|E_{ij}|&=&-1,\cr
|E_{ii}(u)|&=&u.
\end{array}\]

Las operaciones elementales por filas y columnas se corresponden con los
productos por matrices elementales a izquierda y derecha,
respectivamente: \[\begin{array}{rclrcl}
A&\stackrel{F_i+rF_j}\longrightarrow& E_{ij}( r )A,& A&\stackrel{C_i+rC_j}\longrightarrow& AE_{ji}( r ),\cr
A&\stackrel{F_i\leftrightarrow F_j}\longrightarrow &E_{ij}A=A,&  A&\stackrel{C_i\leftrightarrow C_j}\longrightarrow& AE_{ij},\cr
A&\stackrel{uF_i}\longrightarrow& E_{ii}(u)A,& A&\stackrel{uC_i}\longrightarrow& AE_{ii}(u).
\end{array}\]

Por tanto, si \(A'\) se obtiene a partir de \(A\) a través de
operaciones elementales por filas y columnas, entonces existen matrices
invertibles \(P\) y \(Q\) tales que \[A'=QAP^{-1}.\] En particular \(A\)
y \(A'\) presentan el mismo módulo.

También deducimos que el determinante de una matriz no varía cuando se
realiza una operación elemental de tipo 1, cambia de signo al hacer una
operación elemental de tipo 2, y pasa a ser un asociado al realizar una
operación elemental de tipo 3. \End{remark}

\Begin{theorem}\textrm{\normalfont (Forma normal de Smith)}\label{smith}
Dada una matriz \(A\) de tamaño \(m\times n\) sobre un dominio euclídeo
\(R\), existen matrices invertibles \(P\) y \(Q\), que son de hecho
productos de matrices elementales, tales que
\[QAP^{-1}=D=\left( \begin{array}{ccc|c} d_1&&&\cr &\ddots&&0\cr &&d_k&\cr \hline &0&&0 \end{array} \right)\]
es una matriz con una descomposición de tamaño \(2\times 2\) por cajas
cuya única caja no trivial es la superior izquierda, que es diagonal con
entradas diagonales no nulas y satisface \(d_i|d_{i+1}\) para todo
\(1\leq i {<} k\). Esta matriz \(D\) se denomina \textbf{forma normal de
Smith} de \(A\). \End{theorem}

\Begin{proof}

La estrategia para probar la existencia consiste en aplicarle a \(A\)
una serie de operaciones elementales para llegar a una matriz de la
forma \[\left(\begin{array}{c|c}
d_1&0\cr \hline 0&B
\end{array}\right)\] donde \(d_1\) divide a todas las entradas de \(B\).
Una vez hecho esto, pasamos a trabajar del mismo modo con la matriz
\(B\). De este modo obtenemos el resultado por inducción ya que si un
elemento de \(R\) divide a todas las entradas de una matriz \(B\)
entonces también divide a las entradas de una matriz obtenida a partir
de \(B\) mediante una operación elemental.

Sea \(A\) una matriz no nula (si fuera nula no habría nada que hacer).
Para llegar a una matriz como la anterior a partir de \(A\) aplicamos el
siguiente procedimiento:

\begin{enumerate}
\def\labelenumi{\arabic{enumi}.}
\item
  Permutando filas y columnas, mueve la entrada no nula de menor tamaño
  a la esquina superior izquierda.
\item
  Dada una entrada no nula de la primera columna \(a_{i1}\), \(i>1\),
  realiza la división euclídea \(a_{i1}=c\cdot a_{11}+r\) y la
  operación \(F_{i}-c\cdot F_1\). La entrada \((i,1)\) de la nueva
  matriz es el resto \(r\). Si este resto es no nulo entonces tiene
  tamaño menor que \(a_{11}\), así que volvermos al paso 1. Si no,
  continuamos con otra entrada no nula de la primera columna. Si el
  resto de entradas de la primera columna son \(0\), pasamos a hacer lo
  análogo con la primera fila, es decir, buscamos una entrada
  \(a_{1j}\) no nula, \(j>1\), realizamos la división euclídea
  \(a_{1j}=c\cdot a_{11}+r\) y la operación \(C_{j}-c\cdot C_1\). La
  entrada \((1,j)\) de la nueva matriz es el resto \(r\). Si este resto
  no es nulo entonces tiene tamaño menor que el de \(a_{11}\) y pasamos
  al paso 1. Si no, realizamos el mismo procedimiento con otra entrada
  no nula de la primera fila.
\item
  Cuando lleguemos aquí es porque el único elemento no nulo de la
  primera fila y de la primera columna es el \(a_{11}\). Si hay algún
  elemento no nulo \(a_{ij}\) que no es divisible por \(a_{11}\)
  realizamos la operación \(F_{1}+F_{i}\). La matriz resultante tiene
  el mismo \(a_{11}\), pero en la entrada \((1,j)\) nos encontramos con
  \(a_{ij}\), que no es múltiplo de \(a_{11}\), así que volvemos al
  paso 1. (También podríamos hacer la operación \(C_{1}+C_{j}\) y
  pasar al paso 1.) Si no lo hay, es porque nuestra matriz ya es de la
  forma \[\left(\begin{array}{c|c}
  d_1&0\cr \hline 0&B
  \end{array}\right)\] y \(d_1\) divide a todas las entradas de \(B\).
\end{enumerate}

Este procedimiento termina porque cada vez que realizamos una división
euclídea con resto no nulo, el mínimo de los tamaños de los elementos no
nulos disminuye. Como este número es un entero no negativo, no puede
disminuir indefinidamente. Esto asegura que el procedimiento acaba tras
un número finito de pasos.

Este método para llegar a la forma normal de Smith se conoce como
\textbf{algoritmo de diagonalización} de matrices con entradas en \(R\).

En virtud de la correspondencia entre operaciones y matrices
elementales, la matriz \(Q\) del enunciado se obtiene al realizar sobre
la matriz identidad \(I_m\) de tamaño \(m\times m\) todas las
operaciones elementales por filas que hemos usado para hallar \(D\), en
el mismo orden. Análogamente, \(P^{-1}\) se obtiene al aplicarle a
\(I_n\) todas las operaciones elementales por columnas realizadas para
calcular \(D\).

\End{proof}

La siguiente aplicación es una calculadora de la forma normal de Smith
paso a paso. El dato de entrada es una matriz con entradas en
\(\mathbb{Z}\) expresada como lista de filas.

El teorema de la forma normal de Smith es cierto más generalmente para
dominios de ideales principales. La demostración es análoga pero hace
uso de la identidad de Bézout en lugar de la división euclídea y de un
tipo más general de operación elemental. La forma normal de Smith es
única salvo asociados, aunque no lo hemos probado.

\Begin{corollary}

Toda matriz invertible con entradas en un dominio euclídeo es producto
de matrices elementales. \End{corollary}

\Begin{proof}

El determinante de una matriz \(A\) de tamaño \(n\times n\) es asociado
del determinante de cualquier otra matriz que se obtenga a partir de
\(A\) tras realizar una operación elemental. En particular, \(|A|\) es
asociado del determinante de su forma normal de Smith \(D\). El
determinante \(|D|\) es \(0\) si \(k{<}n\) y \(d_1\cdots d_n\) si
\(k=n\). Si \(A\) es invertible entonces \(|A|\) es una unidad, luego
necesariamente \(k=n\) y \(d_1\cdots d_n\) también es una unidad. En
particular todos los \(d_i\) son unidades, es decir, la forma normal de
Smith es un producto de matrices elementales tipo 3,
\(D=E_{11}(d_1)\cdots E_{nn}(d_n)\). Despejando \(A=Q^{-1}DP\) vemos
que \(A\) es producto de matrices elementales. \End{proof}

\Begin{corollary}

Dado un dominio de ideales principales \(R\), todo submódulo de \(R^m\)
es libre y de rango \(\leq m\). \End{corollary}

\Begin{proof}

Sea \(M\subset R^m\) un submódulo. Al
\protect\hyperlink{fgfp}{demostrar} que todo \(R\)-módulo finitamente
generado es finitamente presentado vimos que se puede suponer sin
pérdida de generalidad que \(M=\operatorname{im}A\) para cierto
homomorfismo \(A\colon R^n\rightarrow R^m\) definido por una matriz
\(A\) de tamaño \(m\times n\). El \(R\)-módulo \(M\) está pues generado
por las columnas de \(A\). Tomamos la forma normal de Smith
\(D=QAP^{-1}\). Esta ecuación equivale a \(DP=QA\), que es lo mismo que
decir que el siguiente diagrama de \(R\)-módulos libres conmuta,

\begin{figure}
\centering
\includegraphics{static/images/libres.png}
\caption{Cuadrado conmutativo}
\end{figure}

Como \(P\) y \(Q\) son invertibles, los homomorfismos que definen son
isomorfismos, por tanto \(Q\) induce in isomorfismo
\(M=\operatorname{im}A\cong \operatorname{im}D\). La imagen del
homomorfismo definido por\\
\[
D=\left( \begin{array}{ccc|c} d_1&&&\cr &\ddots&&0\cr &&d_k&\cr \hline &0&&0 \end{array} \right)
\] está también generada por sus columnas. Claramente, para generar
\(\operatorname{im} D\) bastan las primeras \(k\) columnas de \(D\), que
además son linealmente independientes, por tanto forman una base
\[\{d_1e_1,\dots, d_ke_k\}\subset \operatorname{im} D\] y este módulo
es libre re rango \(k\leq m\). Esto implica que
\(M=\operatorname{im}A\cong \operatorname{im} D\) es libre del mismo
rango con base \[\{d_1Q^{-1}e_1,\dots, d_kQ^{-1}e_k\}.\] \End{proof}

\hypertarget{teoremas-de-estructura}{%
\subsection{Teoremas de estructura}\label{teoremas-de-estructura}}

Veamos que sobre un dominio de ideales principales todo módulo
finitamente generado se descompone salvo isomorfismo como producto de
módulos cíclicos.

\Begin{theorem}\textrm{\normalfont (Estructura de módulos finitamente generados sobre un DIP, 1ª forma)}
Dado un dominio de ideales principales \(R\), todo \(R\)-módulo
finitamente generado \(M\) es isomorfo a uno de la forma
\[\frac{R}{(d_1)}\times \cdots \times\frac{R}{(d_n)}\times R^{r}\]
donde \(d_1,\dots,d_n\in R\) no son cero ni unidades y satisfacen
\(d_i|d_{i+1}\), \(1\leq i{<}n\). \End{theorem}

\Begin{proof}

Hemos visto en una \protect\hyperlink{fgfp}{proposición} anterior que
\(M\) es finitamente presentado. Sea la matriz \(A\) una presentación de
\(M\cong R^m/\operatorname{im}A\). En virtud de
\protect\hyperlink{simplify}{otra}, su forma normal de
\protect\hyperlink{smith}{Smith} \(D=QAP^{-1}\) también presenta
\(M\cong R^m/\operatorname{im}D\). Es más, podemos eliminar las columnas
de ceros. Más aún, algún \(d_i\) podría ser una unidad (los anteriores
también tendrían que serlo, pues lo dividen). En este caso, podríamos
eliminar la \(i\)-ésima fila y la \(i\)-ésima columna. De este modo
acabamos con una matriz
\[\left( \begin{array}{ccc} d_1&&\cr &\ddots&\cr &&d_n\cr \hline &0& \end{array} \right)\]
donde los \(d_i\) no son nulos ni unidades, y además satisfacen
\(d_i|d_{i+1}\), \(1\leq i{<}n\). El módulo presentado por esta matriz
es el del enunciado, en virtud de una
\protect\hyperlink{easy}{proposición} anterior más. \End{proof}

La descomposición anterior de un \(R\)-módulo \(M\) como producto de
\(R\)-módulos cíclicos se puede agrupar en dos factores, la
\textbf{parte libre} y la \textbf{parte de torsión},
\[\underbrace{\frac{R}{(d_1)}\times \cdots \times\frac{R}{(d_n)}}_{\text{parte de torsión}}\times \underbrace{R^{r}}_{\text{parte libre}}.\]

\Begin{remark}

Este primer teorema de estructura demuestra que todo \(R\)-módulo
finitamente generado sobre un DIP \(R\) es isomorfo a un producto de
módulos cíclicos de un tipo muy particular. Este producto es de hecho
único en el siguiente sentido. Si \[
\frac{R}{(d_1)}\times \cdots \times\frac{R}{(d_n)}\times R^{r}
\cong  
\frac{R}{(e_1)}\times \cdots \times\frac{R}{(e_m)}\times R^{s},
\] donde los \(d_i\) y los \(e_j\) no son nulos ni unidades y
satisfacen \(d_i\mid d_{i+1}\) y \(e_j\mid e_{j+1}\), entonces
\(r=s\), \(n=m\) y cada \(d_i\) es asociado de \(e_i\).

La demostración de que \(r=s\) la veremos en general, pero el resto solo
lo probaremos para \(R=\mathbb{Z}\) y lo esbozaremos par \(R=k[x]\) con
\(k\) un cuerpo. \End{remark}

\Begin{watch}

A pesar de la unicidad de la forma del primer teorema de estructura, el
isomorfismo no es único. Por ejemplo, de
\(\frac{\mathbb{Z}}{(2)}\times \frac{\mathbb{Z}}{(2)}\) en sí mismo
tenemos dos isomorfismos, la identidad y el intercambio de coordenadas
\((a,b)\mapsto (b,a)\). \End{watch}

\Begin{remark}

Todo módulo cíclico \(R/(d)\) sobre un DIP \(R\) está en la forma del
primer teorema de estructura, por tanto la unicidad antes mencionada
demuestra que un \(R\)-módulo es cíclico si y solo si la descomposición
dada por el teorema de estructura posee un único factor. \End{remark}

Dado un módulo \(M\) finitamente generado sobre un DIP, vamos a ver que
\(T(M)\) es isomorfo a la parte de torsión de la descomposición del
teorema de estructura y que \(M/T(M)\) es isomorfo a la parte libre.

\Begin{proposition}\label{partsoffg} Si \(M\) es un \(R\)-módulo sobre
un dominio \(R\) tal que
\[M\cong \frac{R}{(d_1)}\times \cdots \times\frac{R}{(d_n)}\times R^{r}\]
con \(d_i\neq 0\) para todo \(i\), entonces \[
\begin{array}{rcl}
T(M)&\cong&\frac{R}{(d_1)}\times \cdots \times\frac{R}{(d_n)},\cr
M/T(M)&\cong& R^r.
\end{array}
\] En particular, el \(R\)-módulo \(M/T(M)\) es libre de rango \(r\).
\End{proposition}

\Begin{proof}

Hemos visto antes que la torsión se preserva por
\protect\hyperlink{torsionquotient}{isomorfismos} y además preserva
\protect\hyperlink{torsionproduct}{productos}, así que, por un lado, \[
\begin{array}{rcl}
T(M)&\cong&T\left(\frac{R}{(d_1)}\times \cdots \times\frac{R}{(d_n)}\times R^{r}\right)\cr
&=&\frac{R}{(d_1)}\times \cdots \times\frac{R}{(d_n)}\times \{0\}\cr
&\cong&\frac{R}{(d_1)}\times \cdots \times\frac{R}{(d_n)}.
\end{array}
\] Por otro lado, \[
\begin{array}{rcl}
\frac{M}{T(M)}&\cong&
\frac{\frac{R}{(d_1)}\times \cdots \times\frac{R}{(d_n)}\times R^{r}}{T\left(\frac{R}{(d_1)}\times \cdots \times\frac{R}{(d_n)}\times R^{r}\right)}\cr
&=& \frac{\frac{R}{(d_1)}\times \cdots \times\frac{R}{(d_n)}\times R^{r}}{\frac{R}{(d_1)}\times \cdots \times\frac{R}{(d_n)}\times\{0\}}\cr
&\cong&R^r.
\end{array}\]\\
\End{proof}

Veamos que el rango de la parte libre de la descomposición del teorema
de structura solo depende del módulo \(M\).

\Begin{corollary}\label{equalrank} Dado un dominio \(R\) y dos
\(R\)-módulos isomorfos \(M\cong N\) tales que \[
\begin{array}{rcl}
M&\cong&\frac{R}{(d_1)}\times \cdots \times\frac{R}{(d_n)}\times R^{r},
\cr
N&\cong &\frac{R}{(e_1)}\times \cdots \times\frac{R}{(e_m)}\times R^{s},
\end{array}
\] y los \(d_i\) y los \(e_j\) son no nulos para todos los \(i\) y
\(j\). Entonces \(r=s\). \End{corollary}

\Begin{proof}

Como acabamos de ver, \(M/T(M)\) es libre de rango \(r\) y \(N/T(N)\) es
libre de rango \(s\). También hemos
\protect\hyperlink{torsionquotient}{visto} que \(M/T(M)\cong N/T(N)\).
Deducimos que \(r=s\), ya que el rango de un módulo libre es invariante
por isomorfismos, pues los isomorfismos preservan bases. \End{proof}

\Begin{theorem}\textrm{\normalfont (Teorema chino del resto)} Si \(R\)
es un DIP y \(\operatorname{mcd}(a,b)=1\) entonces tenemos un
isomorfismo
\[\begin{array}{rcl}g\colon \frac{R}{(ab)}&\stackrel{\cong}\longrightarrow&\frac{R}{(a)}\times \frac{R}{(b)},\cr\bar r&\mapsto &(\bar r,\bar r).\end{array}\]
\End{theorem}

\Begin{proof}

Consideramos el homomorfismo de \(R\)-módulos
\[f=\phi_{\{(\bar 1,\bar 1)\}}\colon R\longrightarrow\frac{R}{(a)}\times \frac{R}{(b)}\]
definido por \(f(1)=(\bar 1,\bar 1)\). Para un \(r\in R\) cualquiera,
como \(f\) preserva el producto por escalares
\[f( r )=rf(1)=r(\bar 1,\bar 1)=(\bar r,\bar r).\] Por el primer
\protect\hyperlink{isomodules}{teorema} de isomorfía, \[
\frac{R}{\ker f}\cong\\operatorname{im} f
\] y este isomorfismo está definido como
\(\bar r\mapsto f( r )=(\bar r,\bar r)\). Por tanto bastará probar que
\(\ker f=(ab)\) y que \(f\) es sobreyectivo.

Veamos primero que \(\ker f=(ab)\). En efecto \(ab\in\ker f\) ya que
\(ab\equiv 0\) módulo \((a)\) y módulo \((b)\). Por lo tanto,
\((ab)\subset \ker f\). Por otro lado, si
\(f( r )=(\bar r,\bar r)=(\bar 0,\bar 0)\), es decir si \(r\equiv 0\)
módulo \((a)\) y módulo \((b)\), entonces \(a|r\) y \(b|r\) luego
\(ab=\operatorname{mcm}(a,b)|r\), esto es \(r\in (ab)\).

Veamos ahora que \(f\) es sobreyectivo, es decir, que dado cualquier
\((\bar c,\bar d)\in \frac{R}{(a)}\times \frac{R}{(b)}\) existe
\(x\in R\) tal que \(f(x)=(\bar{c},\bar{d})\). Esto equivale a resolver
el sistema de ecuaciones
\[\left\{\begin{array}{rcl}x&\equiv& c\mod (a),\cr x&\equiv& d\mod (b),\end{array}\right.\]
para \(c,d\in R\) cualesquiera. Tomamos una identidad de Bézout
\(sa+tb=1\) y observamos que \(x=sad+tbc\) resuelve la ecuación.
\End{proof}

\Begin{theorem}\textrm{\normalfont (2ª forma del teorema de estructura)}
Dado un dominio de ideales principales \(R\), todo \(R\)-módulo
finitamente generado \(M\) es isomorfo a uno de la forma
\[\frac{R}{(p_1^{m_1})}\times \cdots \times\frac{R}{(p_n^{m_n})}\times R^{r}\]
donde \(p_1,\dots,p_n\in R\) son primos y \(m_i\geq 1\).
\End{theorem}

\Begin{proof}

En virtud de la primera forma del teorema de estructura, basta ver que
si \(d\in R\) no es nulo ni una unidad, entonces \(R/(d)\) es isomorfo a
un producto de módulos cíclicos de la forma \(R(p^m)\) con \(p\) primo.

Todo DIP es un DFU, así que \(d\) se puede descomponer, salvo asociados,
como producto de potencias de primos no asociados,
\(p_{1}^{m_1}\cdots p_{n}^{m_n}\). Entonces, aplicando
reiteradamente el teorema chino del resto, obtenemos que
\[\frac{R}{(d)}=\frac{R}{(p_{1}^{m_1}\cdots p_{n}^{m_n})}\cong \frac{R}{(p_{1}^{m_1})}\times\cdots\times \frac{R}{(p_{n}^{m_n})}.\]
El isomorfismo está definido simplemente como
\(\bar r\mapsto (\bar r,\dots,\bar r)\).\\
\End{proof}

Las dos formas del teorema de estructura de módulos finitamente
generados sobre un DIP son de hecho equivalentes, se puede pasar de una
a otra mediante el isomorfismo dado por el teorema chino del resto. La
segunda forma del teorema de estructura es por tanto también única, esta
vez salvo reordenamiento de los factores de la parte de torsión.

\Begin{example}\textrm{\normalfont (Las dos formas del teorema de estructura)}
El \(\mathbb{Z}\)-módulo
\[\frac{\mathbb{Z}}{(2)}\times \frac{\mathbb{Z}}{(2)}\times \frac{\mathbb{Z}}{(2^2)}\times \frac{\mathbb{Z}}{(3)}\times \frac{\mathbb{Z}}{(5)}\times\frac{\mathbb{Z}}{(5^2)}\]
está descompuesto según la segunda forma del teorema de estructura. Para
descomponerlo según la primera, comenzamos agrupando los factores que en
su denominador tiene las potencias de mayor grado de todos los primos
que aparecen, usando el teorema chino del resto,
\[ \frac{\mathbb{Z}}{(2^2)}\times \frac{\mathbb{Z}}{(3)}\times\frac{\mathbb{Z}}{(5^2)}\cong 
\frac{\mathbb{Z}}{(2^2\cdot 3\cdot 5^2)}=\frac{\mathbb{Z}}{(300)}.\]
Seguidamente, agrupamos las potencias del segundo (si lo hubiere) mayor
grado al que aparecen elevados los primos,
\[\frac{\mathbb{Z}}{(2)}\times \frac{\mathbb{Z}}{(5)}\cong \frac{\mathbb{Z}}{(2\cdot 5)}=\frac{\mathbb{Z}}{(10)}.\]
Observamos que aquí ya no aparece ninguna potencia de 3, ya que la única
que había ha sido incluida en el grupo anterior. Así seguiríamos hasta
agotar todos los factores. En este ejemplo solo quedaría uno más,
\[\frac{\mathbb{Z}}{(2)}.\] El resultado es
\[\frac{\mathbb{Z}}{(2)}\times \frac{\mathbb{Z}}{(10)}\times \frac{\mathbb{Z}}{(300)}.\]
Un isomorfismo concreto entre este grupo abeliano y el del comienzo
viene dado como composición de dos, primero \[
\begin{array}{rcl}
\frac{\mathbb{Z}}{(2)}\times \frac{\mathbb{Z}}{(10)}\times \frac{\mathbb{Z}}{(300)}&\stackrel{\cong}\longrightarrow&
\frac{\mathbb{Z}}{(2)}\times \frac{\mathbb{Z}}{(2)}\times \frac{\mathbb{Z}}{(5)}\times \frac{\mathbb{Z}}{(2^2)}\times \frac{\mathbb{Z}}{(3)}\times \frac{\mathbb{Z}}{(5^2)},\cr
(\bar a,\bar b, \bar c)&\mapsto&
(\bar a,\bar b,\bar b, \bar c, \bar c, \bar c),
\end{array}
\] que es un isomorfismo por el teorema chino del resto, y luego \[
\begin{array}{rcl}
\frac{\mathbb{Z}}{(2)}\times \frac{\mathbb{Z}}{(2)}\times \frac{\mathbb{Z}}{(5)}\times \frac{\mathbb{Z}}{(2^2)}\times \frac{\mathbb{Z}}{(3)}\times \frac{\mathbb{Z}}{(5^2)}
&\!\!\!\!\stackrel{\cong}\rightarrow\!\!\!\!&
\frac{\mathbb{Z}}{(2)}\times \frac{\mathbb{Z}}{(2)}\times \frac{\mathbb{Z}}{(2^2)}\times \frac{\mathbb{Z}}{(3)}\times \frac{\mathbb{Z}}{(5)}\times\frac{\mathbb{Z}}{(5^2)},\cr
(\bar{a}_1,\bar{a}_2,\bar{a}_3,\bar{a}_4,\bar{a}_5,\bar{a}_6)&\!\!\!\!\mapsto\!\!\!\!&
(\bar{a}_1,\bar{a}_2,\bar{a}_4,\bar{a}_5,\bar{a}_3,\bar{a}_6),
\end{array}
\] que es un isomorfismo por la conmutatividad del producto salvo
isomorfismo. \End{example}

\Begin{example}\textrm{\normalfont (Grupos abelianos con el mismo número de elementos)}
Los grupos abelianos
\[\frac{\mathbb{Z}}{(4)}\times \frac{\mathbb{Z}}{(4)},\qquad\frac{\mathbb{Z}}{(2)}\times \frac{\mathbb{Z}}{(2)}\times \frac{\mathbb{Z}}{(4)},\]
tienen \(16\) elementos, pues \(4\cdot 4=16=2\cdot 2\cdot 4\), luego
existe una biyección entre ambos. Veamos que, a pesar de ello, no son
isomorfos. Para verlo jugaremos con la noción de \textbf{orden} de un
elemento de torsión \(a\in A\) de un grupo abeliano \(A\), que es el
menor entero positivo \(n\in\mathbb Z\) tal que \(na=0\). El orden de
\(a\) divide a \(n\) si y solo si \(na=0\). En particular, para cada
entero \(n\), el subconjunto \(T_n(A)\subset A\) formado por los
elementos cuyo orden divide a \(n\) es un subgrupo, que se puede
describir como \[T_n(A)=\{a\in A\mid n\cdot a=0\}.\] Denotaremos
\(t_n(A)\) al orden de \(T_n(A)\). Además
\(T_n(A\times B)=T_n(A)\times T_n(B)\), luego
\(t_n(A\times B)=t_n(A)t_n(B)\). Es más, todo isomorfismo
\(A\cong B\) se restringe a \(T_n(A)\cong T_n(B)\), por tanto en este
caso \(t_n(A)=t_n(B)\). Dado \(m\neq 0\), el orden de
\(\bar a\in\mathbb Z/(m)\) divide a \(n\) si y solo si \(m\mid na\). Si
denotamos \(d=\operatorname{mcd}(n,m)\), esto equivale a decir que
\(\frac{m}{d}\mid a\), por tanto
\[T_n\left(\frac{\mathbb{Z}}{(m)}\right)=\left(\overline{\frac{m}{d}}\right)=\left\{1\cdot\overline{\frac{m}{d}},\dots,(d-1)\cdot \overline{\frac{m}{d}}\right\},\]
pues \(\overline{\frac{m}{d}}\) tiene orden \(d\). Luego
\[t_n\left(\frac{\mathbb{Z}}{(m)}\right)=\operatorname{mcd}(n,m).\]
Aplicando esto a los dos grupos del principio, vemos que el primero
cumple
\[t_2\left(\frac{\mathbb{Z}}{(4)}\times \frac{\mathbb{Z}}{(4)}\right)=2\cdot 2=4,\]
mientras que el segundo satisface
\[t_2\left(\frac{\mathbb{Z}}{(2)}\times \frac{\mathbb{Z}}{(2)}\times \frac{\mathbb{Z}}{(4)}\right)=2\cdot 2\cdot 2=8,\]
con lo cual no pueden ser isomorfos. \End{example}

El siguiente teorema, combinado con la
\protect\hyperlink{partsoffg}{proposición} y el
\protect\hyperlink{equalrank}{corolario} anterior, demuestra que la
descomposición de un \(\mathbb Z\)-módulo finitamente generado \(M\)
dada por el primer teorema de estructura es única. La del segundo
también lo será, salvo cambio de orden de los factores de la parte de
torsión, pues ambos son equivalentes.

\Begin{theorem}

Dados dos \(\mathbb Z\)-módulos \[
\begin{array}{rlc}
A&=&\frac{\mathbb Z}{(d_1)}\times \cdots \times\frac{\mathbb Z}{(d_n)},\cr
B&=&\frac{\mathbb Z}{(e_1)}\times \cdots \times\frac{\mathbb Z}{(e_m)},
\end{array}
\] donde los \(d_i\) y los \(e_j\) no son nulos ni unidades y
satisfacen \(d_i|d_{i+1}\) y \(e_j|e_{j+1}\), si \(A\cong B\)
entonces \(n=m\) y cada \(d_i\) es asociado de \(e_i\). \End{theorem}

\Begin{proof}

Podemos suponer sin pérdida de generalidad que los \(d_i\) y los
\(e_j\) son positivos. Como ambos grupos son finitos e isomorfos,
tienen el mismo orden, es decir, \[d_1\cdots d_n=e_1\cdots e_m.\]

En \(A\), el orden de cualquier elemento divide a \(d_n\), y en \(B\) a
\(e_m\). Ambos grupos son isomorfos, por tanto el orden de cualquier
elemento de \(A\) o de \(B\) divide a \(d_n\) y a \(e_m\). En \(A\)
hay un elemento de orden \(d_n\), el \((\bar 0,\dots,\bar 0,\bar 1)\),
por tanto \(d_n|e_m\). En \(B\) hay otro con orden \(e_m\), por tanto
\(e_m|d_n\), así que \(d_n=e_m\). De aquí deducimos también que
\[d_1\cdots d_{n-1}=e_1\cdots e_{m-1}.\]

Demostremos ahora que \(d_{n-1}=e_{m-1}\). Por un lado,
\[t_{d_{n-1}}(A)=d_1\cdots d_{n-2}\cdot d_{n-1}^2\] y por otro
\[\begin{array}{rcl}
t_{d_{n-1}}(B)&=&\operatorname{mcd}(d_{n-1},e_1)\cdots\operatorname{mcd}(d_{n-1},e_{m-1})\operatorname{mcd}(d_{n-1},e_{m})\cr
&=&\frac{d_{n-1}e_1}{\operatorname{mcm}(d_{n-1},e_1)}\cdots\frac{d_{n-1}e_{m-1}}{\operatorname{mcm}(d_{n-1},e_{m-1})}d_{n-1}\cr
&=&\frac{e_1\cdots e_{m-1}d_{n-1}^m}{\operatorname{mcm}(d_{n-1},e_1)\cdots\operatorname{mcm}(d_{n-1},e_{m-1})}\cr
&=&\frac{d_1\cdots d_{n-2}\cdot d_{n-1}^{m+1}}{\operatorname{mcm}(d_{n-1},e_1)\cdots\operatorname{mcm}(d_{n-1},e_{m-1})}.
\end{array}\] Como necesariamente \(t_{d_{n-1}}(A)=t_{d_{n-1}}(B)\)
deducimos que
\[d_{n-1}^{m-1}=\operatorname{mcm}(d_{n-1},e_1)\cdots\operatorname{mcm}(d_{n-1},e_{m-1}).\]
Pero \(d_{n-1}|\operatorname{mcm}(d_{n-1},e_{j})\) para todo
\(1\leq j\leq m-1\), así que la única posibilidad de que ambos productos
coincidan es que en todos los casos
\(d_{n-1}=\operatorname{mcm}(d_{n-1},e_{j})\), es decir
\(e_j|d_{n-1}\). Los papeles de \(A\) y \(B\), y en particular los de
los \(d_i\) y los \(e_j\), son intercambiables, así que también
concluimos que \(d_i|e_{m-1}\) para todo \(1\leq i\leq n-1\). En
particular \(e_{m-1}|d_{n-1}\) y \(d_{n-1}|e_{m-1}\), por tanto
\(d_{n-1}=e_{m-1}\) y deducimos también que
\[d_1\cdots d_{n-2}=e_1\cdots e_{m-2}.\]

Este argumento se puede iterar, probando así que los últimos \(d_i\)
coinciden con los últimos \(e_j\). Veamos por reducción al absurdo que
\(n=m\), con lo cual \(d_i=e_i\) para todo \(1\leq i\leq n\). Si
\(n\neq m\) podemos suponer sin pérdida de generalidad que \(n{<}m\). En
ese caso acabaríamos probando que \(1=e_1\cdots e_{m-n}\), pero esto
implicaría que estos primeros \(e_j\) son unidades, lo cual sería una
contradicción. \End{proof}

\Begin{example}\textrm{\normalfont ($k[x]$-módulos de torsión con la misma dimensión)}
Si \(p(x)\in k[x]\) tiene grado \(n\), el cociente \(k[x]/(p(x))\) tiene
dimensión \(n\) como \(k\)-espacio vectorial pues
\(\{1,\dots,\bar x^{n-1}\}\) es una base. Por tanto los
\(k[x]\)-módulos
\[\frac{k[x]}{(x^2)}\times \frac{k[x]}{(x^2)},\qquad \frac{k[x]}{(x)}\times \frac{k[x]}{(x)}\times \frac{k[x]}{(x^2)}\]
tienen dimensión \(2+2=4=1+1+2\), así que son isomorfos como
\(k\)-espacios vectoriales, pero no como \(k[x]\)-módulos. En efecto,
para verlo, podemos razonar con de \textbf{orden} de un elemento de
torsión \(a\in M\) de un \(k[x]\)-módulo \(M\), que es el polinomio
mónico no nulo \(p(x)\in k[x]\) de menor grado tal que
\(p(x)\cdot a=0\). Esta noción de orden satisface propiedades formales
similares a las de grupos abelianos. Por ejemplo, al orden de \(a\)
divide a un polinomio \(p(x)\in k[x]\) si y solo si \(p(x)\cdot a=0\).
Los números relevantes aquí son los \(t_{p(x)}(M)\), que es la
dimensión como \(k\)-espacio vectorial del submódulo
\(T_{p(x)}(M)\subset M\) formado por los elementos cuyo orden divide a
\(p(x)\in k[x]\), \[T_{p(x)}(M)=\{a\in M\mid p(x)\cdot a=0\}.\]
Tenemos que \(T_{p(x)}(M\times N)=T_{p(x)}(M)+ T_{p(x)}(N)\), luego
\(t_{p(x)}(M\times N)=t_{p(x)}(M)+ t_{p(x)}(N)\) pues la dimensión de
un producto de espacios vectoriales es la suma de las dimensiones de los
factores. Es más, todo isomorfismo de \(k[x]\)-módulos \(M\cong N\) se
restringe a \(T_{p(x)}(M)\cong T_{p(x)}(N)\), así que en este caso
\(t_{p(x)}(M)= t_{p(x)}(N)\). Además podemos comprobar que
\(t_{p(x)}(k[x]/(q(x)))\) es el grado de
\(\operatorname{mcd}(p(x),q(x))\).

Los dos \(k[x]\)-módulos del comienzo no pueden ser isomorfos porque
\[\begin{array}{rcl} t_{x}\left(\frac{k[x]}{(x^2)}\times \frac{k[x]}{(x^2)}\right)&=&1+1=2,\cr t_{x}\left(\frac{k[x]}{(x)}\times \frac{k[x]}{(x)}\times \frac{k[x]}{(x^2)}\right)&=&1+1+1=3. \end{array}\]
La demostración de la unicidad de las descomposicións de los teoremas de
estructura de \(R\)-módulos finitamente generados para \(R=k[x]\) es
análoga al caso de \(R=\mathbb Z\), usando para \(R=k[x]\) los números
\(t_{p(x)}(M)\) y el orden de un \(k[x]\)-módulo \(M\) de torsión
\(M=T(M)\), que es simplemente su dimensión como \(k\)-espacio
vectorial. La dejamos como ejercicio. \End{example}

\hypertarget{sistemas-de-ecuaciones-lineales-diofuxe1nticas}{%
\subsection{Sistemas de ecuaciones lineales
diofánticas}\label{sistemas-de-ecuaciones-lineales-diofuxe1nticas}}

Consideramos un sistema de \(m\) ecuaciones lineales con \(n\)
incógnitas y coeficientes y términos independientes enteros, \[\left\{
\begin{array}{ccl}
a_{11}x_1+\cdots+a_{1n}x_n&=&b_1,\cr \vdots&&\vdots \cr
a_{m1}x_1+\cdots+a_{mn}x_n&=&b_m.
\end{array}
\right.\] Estamos interesados en hallar las soluciones enteras, es
decir, lo consideramos como un sistema de ecuaciones diofánticas.

Si llamamos \[\begin{array}{ccc}A=
\left(\begin{array}{ccc}a_{11}&\cdots&a_{1n}\cr\vdots&&\vdots\cr a_{m1}&\cdots&a_{mn}\end{array}\right)
,&\vec{x}=\left(\begin{array}{c}x_1\cr\vdots\cr x_n\end{array}\right),
&\vec{b}=\left(\begin{array}{c}b_1\cr\vdots\cr b_m\end{array}\right)\end{array},\]
podemos expresar el sistema simplemente como \[A\vec{x}=\vec{b}.\]

Si \(A\) está en forma normal de Smith,
\[A=D=\left( \begin{array}{ccc|c} d_1&&&\cr &\ddots&&0\cr &&d_k&\cr \hline &0&&0 \end{array} \right)\]
el sistema se reduce a \[\left\{
\begin{array}{ccl}
d_1x_1&=&b_1,\cr 
\vdots&&\vdots \cr
d_kx_k&=&b_k,\cr
0&=&b_{k+1},\cr 
\vdots&&\vdots \cr
0&=&b_{m}.
\end{array}
\right.\] Este sistema claramente tiene solución si y solo si
\(d_i|b_i\) para todo \(1\leq i\leq k\) y \(b_i=0\) para
\(k{<}i\leq m\). En dicho caso las soluciones son
\[x_i=\frac{b_i}{d_i},\quad 1\leq i\leq k;\qquad x_{k+1},\dots,x_n\in\mathbb Z;\]
siendo estos últimos valores paramétricos cualesquiera. Observa que la
solución es única si además \(k=n\).

En general, \(A\) tiene una forma normal de Smith \(D\) que satisface
\(QAP^{-1}=D\), es decir \(A=Q^{-1}DP\). Tenemos que
\[A\vec{x}=\vec{b}\Leftrightarrow DP\vec{x}=Q\vec{b}.\] Llamando
\[\vec{y}=P\vec{x},\] lo cual es un simple cambio de veriables, podemos
resolver esta ecuación en \(\vec{y}\) como arriba,
\[D\vec{y}=Q\vec{b}.\] Las soluciones de la ecuación original se
obtienen deshaciendo el cambio de variables \[\vec{x}=P^{-1}\vec{y}.\]

\hypertarget{operadores-lineales}{%
\subsection{Operadores lineales}\label{operadores-lineales}}

Dado un cuerpo \(k\) y un \(k\)-espacio vectorial \(V\), un
\textbf{operador lineal} en \(V\) es un endomorfismo
\(f\colon V\rightarrow V\).

\Begin{proposition}

Un \(k[x]\)-módulo es lo mismo que un \(k\)-espacio vectorial equipado
con un operador lineal. \End{proposition}

\Begin{proof}

Un \(k[x]\)-módulo \(M\) es también un \(k\)-módulo, es decir, un
\(k\)-espacio vectorial, restringiendo el producto por escalares a
\(k\subset k[x]\). La multiplicación por \(x\) es un homomorfismo de
\(k[x]\)-módulos
\[\begin{array}{rcl}M&\stackrel{x\cdot}\longrightarrow& M,\cr a&\mapsto &x\cdot a,\end{array}\]
en particular también es un homomorfismo de \(k\)-módulos, es decir, es
un operador lineal en el \(k\)-espacio vectorial \(M\).

Recíprocamente, dado un \(k\)-espacio vectorial \(V\) y un operador
lineal \(f\colon V\rightarrow V\), podemos definir una estructura de
\(k[x]\)-módulo en \(V\) del siguiente modo. Dado \(v\in V\) y
\(p(x)=a_nx^n+\cdots+a_1x+a_0\in k[x]\), definimos el producto por
escalares como \[p(x)\cdot v=a_nf^n(v)+\cdots+a_1f(v)+a_0v.\] Dejamos
como ejercicio comprobar que este producto por escalares satisface las
propiedades requeridas. \End{proof}

\Begin{remark}

Si \(V=k^n\) y consideramos el operador lineal
\(A\colon k^n\rightarrow k^n\) definido por una matriz \(A\) de tamaño
\(n\times n\) con entradas en \(k\), entonces la estructura de
\(k[x]\)-módulo en \(k^n\) viene dada por \(p(x)\cdot v=p(A)v\) para
todo \(p(x)\in k[x]\) y \(v\in k^n\). \End{remark}

\Begin{proposition}

Dado un operador lineal \(A\colon k^n\rightarrow k^n\), el
\(k[x]\)-módulo asociado \(k^n\) está presentado por la matriz \(A-xI\).
\End{proposition}

\Begin{proof}

Hemos de construir un isomorfismo entre el \(k[x]\)-modulo \(k^n\) y el
\(k[x]\)-modulo cociente \[\frac{k[x]^n}{\operatorname{im} (A-xI)}.\]
Para ello, comenzamos considerando el homomorfismo de \(k[x]\)-modulos
\[\phi=\phi_{\{e_1,\dots,e_n\}}\colon k[x]^n\longrightarrow k^n,\] que
está definido por \(\phi(e_i)=e_i\). Aquí estamos usando la notación
\(e_i\) tanto para los elementos de la base canónica del \(k[x]\)-módulo
\(k[x]^n\) como para los de la base canónica del \(k\)-espacio vectorial
\(k^n\). El homomorfismo \(\phi\) es sobreyectivo porque su imagen
contiene un conjunto de generadores de \(k^n\).

Veamos que \(\operatorname{im} (A-xI)\subset\ker \phi\). Como
\(\operatorname{im} (A-xI)\) está generado por las columnas de la matriz
\(B=A-xI\), basta ver que estas columnas están en el \(\ker \phi\). La
\(j\)-ésima columna es
\[b_{*j}=(a_{ij})_{i=1}^n-xe_j=\sum_{i=1}^na_{ij}e_i-xe_j.\] Por
tanto, \[\phi(b_{*j})=\sum_{i=1}^na_{ij}e_i-Ae_j=0,\] puesto que
\(\sum_{i=1}^na_{ij}e_i=Ae_j\) es la \(j\)-ésima columna de \(A\).
Esto demuestra que \(\phi\) factoriza de manera única a través de la
proyección natural, \(\phi=g\circ p\),
\[k[x]^n\stackrel{p}\twoheadrightarrow \frac{k[x]^n}{\operatorname{im} (A-xI)}\stackrel{g}\longrightarrow k^n.\]
Como \(\phi\) y \(p\) son sobreyectivos, \(g\) también lo será.

Queremos probar que \(g\) es un isomorfismo. Este homomorfismo
sobreyectivo de \(k[x]\)-módulos también lo es de \(k\)-modulos, es
decir, de \(k\)-espacios vectoriales. Si demostramos que la dimensión
del cociente como \(k\)-espacio vectorial es \(\leq n\), entonces \(g\)
será necesariamente biyectivo. Sabemos que
\(S=\{\bar{e}_1,\dots, \bar{e}_n\}\) es un conjunto de generadores
del cociente como \(k[x]\)-módulo. Es decir, todo elemento del cociente
es una combinación lineal de \(S\) con coeficientes polinómicos. Veamos
que \(S\) es también una base del cociente como \(k\)-espacio vectorial,
es decir, que todo elemento es combinación lineal de \(S\) con
coeficientes constantes. Para ello basta ver que \(x \bar{e}_j\) es
siempre una combinación lineal de \(S\) con coeficientes constantes, ya
que de aquí se deduciría por inducción que \(x^m\bar{e}_i\) también es
una combinación lineal de \(S\) con coeficientes constantes para todo
\(m\geq 1\). La \(j\)-ésima columna de \(A-xI\) es
\(\sum_{i=1}^na_{ij}e_i-xe_j\), así que en efecto
\[x\bar{e}_j=\sum_{i=1}^na_{ij}\bar{e}_i.\] \End{proof}

\Begin{proposition}

Un \(k[x]\)-módulo \(M\) finitamente generado es de torsión, \(M=T(M)\),
si y solo si tiene dimensión finita como \(k\)-espacio vectorial.
\End{proposition}

\Begin{proof}

\(\Rightarrow\) Si \(M\) es de torsión entonces por el teorema de
estructura de \(k[x]\)-módulos finitamente generados, \(M\) es isomorfo
a un producto de una cantidad finita de \(k[x]\)-módulos cíclicos
\(k[x]/(p(x))\) con \(p(x)\in k[x]\) un polinomio no trivial. Como
\(k[x]/(p(x))\) tiene dimensión finita como \(k\)-espacio vectorial (su
dimensión es el grado de \(p(x)\)), deducimos que \(M\) también tiene
dimensión finita como \(k\) espacio vectorial (la suma de las
dimensiones de los factores cíclicos del producto).

\(\Leftarrow\) Recíprocamente, si \(M\) tiene dimensión finita como
\(k\) espacio vectorial, entonces no puede tener parte libre en su
descomposición como producto de \(k[x]\)-módulos cíclicos ya que
\(k[x]\) no tiene dimensión finita como \(k\)-espacio vectorial.
\End{proof}

Una \textbf{caja de Jordan} es una matriz cuadrada con una constante
\(\alpha \in k\), denominada \textbf{autovalor}, en todas las entradas
diagonal principal, \(1\) en todas las entradas de la diagonal que está
justo por encima de la principal y \(0\) en el resto,

\begin{figure}
\centering
\includegraphics{static/images/jordanblock.png}
\caption{Caja de Jordan}
\end{figure}

\Begin{theorem}\textrm{\normalfont (Forma normal de Jordan)} Sea \(k\)
un cuerpo algebraicamente cerrado. Dado un \(k\)-espacio vectorial de
dimensión finita \(V\) equipado con un operador lineal
\(f\colon V\rightarrow V\) existe una base de \(V\) respecto de la cual
la matriz de \(f\) es una matriz diagonal por cajas de Jordan. Esta
matriz diagonal por cajas es única salvo permutación de las cajas y se
denomina \textbf{forma normal de Jordan}. \End{theorem}

\Begin{proof}

Como \(k\) es algebraicamente cerrado, los primos en \(k[x]\) son los
polinomios mónicos de grado \(1\) y sus asociados. Sabemos que una base
de \[\frac{k[x]}{(x^m)}\] como \(k\)-espacio vectorial es
\(\{\bar x^{m-1},\dots,\bar x,1\}\). Haciendo un cambio de variables
es fácil ver que una base de \[\frac{k[x]}{((x-\alpha)^m)}\] como
\(k\)-espacio vectorial es
\(\{(\bar x-\alpha)^{m-1},\dots,\bar x-\alpha,1\}\). Como
\[\begin{array}{rcl}
x\cdot(\bar x-\alpha)^j&=&(x-\alpha+\alpha)\cdot(\bar x-\alpha)^j\cr
&=&(x-\alpha)\cdot(\bar x-\alpha)^j+\alpha\cdot(\bar x-\alpha)^j\cr
&=&(\bar x-\alpha)^{j+1}+\alpha\cdot(\bar x-\alpha)^j,
\end{array}\] la matriz de la multiplicación por \(x\),
\[\frac{k[x]}{((x-\alpha)^m)}\stackrel{x\cdot}\longrightarrow \frac{k[x]}{((x-\alpha)^m)}\]
respecto de la base anterior es la caja de Jordan de tamaño
\(m\times m\) y autovalor \(\alpha\).

Usando la primera proposición, consideramos \(V\) como un
\(k[x]\)-módulo de torsión con \(x\cdot a=f(a)\). En virtud del segundo
teorema de estructura, \(V\) se descompone como un producto finito de
\(k[x]\)-módulos cíclicos, cocientes por potencias de primos,
\[V\cong\frac{k[x]}{((x-\alpha_1)^{m_1})}\times\cdots\times \frac{k[x]}{((x-\alpha_n)^{m_n})}.\]
Como \(k\)-espacio vectorial, el \(k[x]\)-módulo de la derecha tiene
base
\[\bigcup_{i=1}^n\{(0,\dots,(\bar x-\alpha_i)^j,\dots,0)\}_{j=m_i-1}^0,\]
donde la coordenada no trivial \((\bar x-\alpha_i)^j\) es la
\(i\)-ésima. Respecto de esta base, la matriz del homomorfismo de
multiplicación por \(x\) es la matriz diagonal por cajas de Jordan de
tamaños \(m_i\times m_i\) y autovalores \(\alpha_i\),
\(1\leq i\leq n\).

\begin{figure}
\centering
\includegraphics{static/images/jordanmatrix.png}
\caption{Matriz de Jordan}
\end{figure}

Traslandando esta base a \(V\) por el isomorfismo dado por el segundo
teorema de estructura, obtenemos una base de \(V\) respecto de la cual
la matriz de \(f\) es esta misma.

La unicidad de la forma normal de Jordan se corresponde con la de la
segunda forma del teorema de estructura. Observa que, sin embargo, la
base respecto de la cual la matriz de \(f\) está en forma normal de
Jordan no es única en general. \End{proof}
