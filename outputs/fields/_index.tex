
¿Sabías que es imposible construir un hepágono regular con una regla y
un compás? ¿Sabías que también es imposible construir de este modo un
cuadrado con la misma área que un círculo dado? Este último problema se
conoce como la \textbf{cuadratura del círculo}. Fue planteado en la
antigüedad y permaneció abierto hasta finales del siglo XIX.

Seguro que sabes que la única raíz de un polinomio de grado 1, \(ax+b\),
es \[x=-\frac{b}{a}.\] También sabes que las raíces de uno de grado 2,
\(ax^2+bx+c\), son \[x=\frac{-b\pm\sqrt{b^2-4ac}}{2a}.\] Es menos
conocido que las raíces de un polinomio de grado 3, \(ax^3+bx^2+cx+d\),
son \[
x=
\left\{
\begin{array}{l}
S+T-\frac{b}{3a},\cr
-\frac{S+T}{2}-\frac{b}{3a}+\frac{i\sqrt{3}}{2}(S-T),\cr
-\frac{S+T}{2}-\frac{b}{3a}-\frac{i\sqrt{3}}{2}(S-T),
\end{array}
\right.
\] donde \[
\begin{array}{rcl}
S&=&\sqrt[3]{R+\sqrt{Q^3+R^2}},\cr
T&=&\sqrt[3]{R-\sqrt{Q^3+R^2}},\cr
Q&=&\frac{3ac-b^2}{9a^2},\cr
R&=&\frac{9abc-27a^2d-2b^3}{54a^3}.
\end{array}
\] De aquí surge por tanto la siguiente cuestión natural: ¿Es posible
expresar las raíces de un polinomio de cualquier grado a partir de sus
coeficientes mediante sumas, multiplicaciones y raíces iteradas? Esto se
denomina resolver una ecuación polinómica por \textbf{radicales}. Esta
importante pregunta es también de origen antiguo y permaneció abierta
hasta el siglo XIX, cuando fue resuelta por
\href{https://en.wikipedia.org/wiki/\%C3\%89variste_Galois}{Galois}. La
respuesta es sencilla, aunque llegar a ella no es fácil: hasta grado 4
sí, de grado 5 en adelante, en general, no. Un ejemplo de polinomio de
grado 5 cuyas raíces no se pueden hallar por radicales es el siguiente,
a pesar de su aparente sencillez, \[x^5-16x+2.\]

A lo largo de este capítulo estudiaremos las matemáticas necesarias para
resolver estas y otras cuestiones relacionadas.
