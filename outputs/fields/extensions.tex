
\hypertarget{extensiones-de-cuerpos}{%
\subsection{Extensiones de cuerpos}\label{extensiones-de-cuerpos}}

\Begin{definition}

Una \textbf{extensión (de cuerpos)} \[F\subset K\] es un par formado por
un cuerpo \(K\) y un subanillo \(F\) que también es un cuerpo. Decimos
en este caso que \(K\) es una extensión de \(F\). Observa que \(K\) es
un \(F\)-espacio vectorial con la suma y el producto por escalares de
\(F\). La extensión es \textbf{finita} si \(\dim_FK<\infty\), en dicho
caso definimos el \textbf{grado} de la extensión como
\[[K:F]=\dim_FK.\] \End{definition}

\Begin{remark}

El grado de una extensión \(F\subset K\) es \([K:F]\geq 1\). No hay
extensiones de grado \(0\) ya que todo cuerpo \(K\neq {0}\).
\End{remark}

\Begin{example}\textrm{\normalfont (Ejemplos de extensiones)}

\begin{itemize}
\item
  \(\mathbb R\subset\mathbb C\) es finita de grado
  \([\mathbb{C},\mathbb{R}]=2\), ya que \(\{1,i\}\subset\mathbb C\) es
  una base como \(\mathbb R\)-espacio vectorial.
\item
  \(\mathbb Q\subset\mathbb R\) no es finita porque cualquier
  \(\mathbb Q\)-espacio vectorial de dimensión finita es numerable, pero
  \(\mathbb R\) no lo es.
\item
  Todo cuerpo \(F\) posee la \textbf{extensión trivial} \(F\subset F\),
  que es la única de grado \(1\), el resto tienen grado \(>1\). En
  efecto, \(\dim_FF=1\) así que, como \(F\subset K\),
  \([K:F]=\dim_FK\geq 1\) dándose la igualdad si y solo si \(F=K\).
\item
  \(F\subset F(x)\) tampoco es finita.
\item
  \(F\subset F[x]/(p(x))\), donde \(p(x)\in F[x]\) es un polinomio
  irreducible. En efecto, por ser \(F[x]\) un DFU tenemos que
  \(p(x)\in F[x]\) es primo, y por ser \(F[x]\) un dominio de ideales
  principales tenemos que el ideal primo \((p(x))\subset F[x]\), al ser
  no trivial, es maximal, por tanto \(F[x]/(p(x))\) es un cuerpo.
  Sabemos además que \[[F[x]/(p(x)):F]=\text{grado }p(x).\]
\end{itemize}

\End{example}

\Begin{definition}

Dadas dos extensiones \(F\subset K\) y \(F\subset L\) de un mismo cuerpo
\(F\), un \textbf{homomorfismo} de extensiones
\(f\colon K\rightarrow L\) es un homomorfismo de anillos que deja fijo a
\(F\), es decir, que satisface \(f(\alpha)=\alpha\) para todo
\(\alpha\in F\). Un \textbf{endomorfismo} de una extensión
\(F\subset K\) un homomorfismo de extensiones
\(f\colon K\rightarrow K\). Un \textbf{isomorfismo} de extensiones es un
homomorfismo biyectivo. Un \textbf{automorfismo} de una extensión
\(F\subset K\) es un endomorfismo biyectivo. \End{definition}

La conjugación \(\mathbb{C}\rightarrow \mathbb{C}\),
\(z\mapsto\bar{z}\), es un homomorfismo de extensiones de \(\mathbb{R}\)
ya que \(x=\bar{x}\) para todo \(x\in\mathbb{R}\).

\Begin{remark}

La identidad es un homomorfismo de extensiones y la composición de
homomorfismos de extensiones es también un homomorfismo de extensiones.
Lo mismo ocurre con isomorfismos y automorfismos. Además, la aplicación
inversa de un isomorfismo de extensiones es otro isomorfismo de
extensiones, e igual para los automorfismos. Los homorfismos de cuerpos
son inyectivos, así que los homomorfismos de extensiones también.
\End{remark}

\Begin{proposition}

Un homorfismo \(f\colon K\rightarrow L\) de extensiones de \(F\) es
también un homomorfismo de \(F\)-espacios vectoriales. \End{proposition}

\Begin{proof}

Como \(f\) es un homomorfismo de anillos, preserva sumas. Dado
\(\alpha\in F\subset K\) y \(x\in K\), por ser \(f\) un homomorfismo de
anillos, \(f(\alpha x)=f(\alpha)f(x)\). Como \(f\) es un homomorfismo de
extensiones de \(F\), \(f(\alpha)=\alpha\). Por tanto
\(f(\alpha x)=\alpha f(x)\), es decir, \(f\) preserva el producto por
escalares de \(F\). \End{proof}

\Begin{corollary}

Si \(f\colon K\rightarrow L\) es un isomorfismo de extensiones de \(F\)
entonces \([K:F]\cong [L:F]\). \End{corollary}

Más adelante veremos ejemplos de extensions no isomorfas del mismo
grado.

\Begin{corollary}

Todo endomorfismo \(f\colon K\rightarrow K\) de una extensión finita
\(F\subset K\) es un automorfismo. \End{corollary}

\Begin{proof}

Como \(f\) es un homomorfismo inyectivo de \(F\)-espacios vectoriales y
su partida y su llegada poseen la misma dimensión, ha de ser un
isomorfismo. \End{proof}

\Begin{proposition}

Dadas dos extensiones \(\mathbb Q\subset K\) y \(\mathbb Q\subset L\) de
\(\mathbb Q\), cualquier homomorfismo de anillos
\(f\colon K\rightarrow L\) es un homomorfismo de extensiones.
\End{proposition}

\Begin{proof}

Por ser \(f\) un homomorfismo de anillos, \(f(0)=0\). Es más, como
\(f(1)=1\) y \(f\) preserva sumas, es fácil ver que \(f(n)=n\) para
cualquier \(n\in\mathbb Z\), \(n>0\). Además \(f\) preserva opuestos,
luego \(f(-n)=-f(n)=-n\). Esto prueba que \(f\) deja fijo a
\(\mathbb Z\). Todo racional se puede expresar como
\(\frac{p}{q}=pq^{-1}\) para \(p,q\in\mathbb Z\), \(q\neq 0\). Los
homomorfismos de anillos preservan productos e inversos, así que
\[\begin{array}{rcl}f\left(\frac{p}{q}\right)&=&f(pq^{-1})\cr &=&f(p)f(q^{-1})\cr &=&f(p)f(q)^{-1}\cr &=&pq^{-1}\cr &=&\frac{p}{q}.\end{array}\]\\
\End{proof}

\Begin{corollary}

Dada una extension finita \(\mathbb{Q}\subset K\), todo homomorfismo de
anillos \(f\colon K\rightarrow K\) es un automorfismo de la extensión
\(\mathbb{Q}\subset K\). \End{corollary}

\Begin{definition}

Dada una extensión \(F\subset K\), decimos que \(\alpha\in K\) es
\textbf{algebraico} si existe \(p(x)\in F[x]\) no nulo tal que
\(p(\alpha)=0\). En caso contrario decimos que \(\alpha\) es
\textbf{trascendente}. \End{definition}

\Begin{remark}

Si tenemos dos extensiones sucesivas \(F\subset K\subset L\) y
\(\alpha\in L\) es algebraico sobre \(F\) entonces también es algebraico
sobre \(K\) ya que \(F[x]\subset K[x]\). ¡Ojo! El recíproco no es
cierto. Todo \(\alpha\in F\) es algebraico sobre \(F\) ya que es raíz de
\(x-\alpha\in F[x]\). \End{remark}

El elemento \(\sqrt{2}\in\mathbb{R}\) es algebraico sobre
\(\mathbb{Q}\), aunque \(\sqrt{2}\notin\mathbb{Q}\). Análogamente
\(i\in\mathbb{C}\) es algebraico sobre \(\mathbb{R}\) ya que es raíz de
\(x^2+1\in\mathbb{R}[x]\), y también sobre \(\mathbb{Q}\).

\Begin{example}\textrm{\normalfont (Existencia de elementos trascendentes en $\mathbb Q\subset\mathbb C$)}
Como \(\mathbb Q\) es numerable, \(\mathbb Q[x]\) también. Además, todo
polinomio tiene una cantidad finita de soluciones en \(\mathbb C\). Por
tanto hay una cantidad numerable de elementos algebraicos para la
extensión \(\mathbb Q\subset\mathbb C\). Como \(\mathbb C\) no es
numerable, han de existir elementos trascendentes, de hecho una cantidad
no numerable de ellos. Lo mismo se aplica a la extensión
\(\mathbb Q\subset\mathbb R\). Dar un ejemplo concreto de número
trascendente es sin embargo bastante complicado. Es conocido que \(\pi\)
es trascendente sobre \(\mathbb Q\) pero no es fácil probarlo.
\End{example}

\Begin{definition}

Dada una extensión \(F\subset K\) y un elemento algebraico
\(\alpha\in K\), su \textbf{polinomio irreducible} \(p(x)\in F[x]\) es
el único polinomio mónico irreducible con coeficientes en \(F\) que
tiene a \(\alpha\) como raíz. El \textbf{grado} de \(\alpha\) sobre
\(F\) es el de su polinomio irreducible. \End{definition}

\Begin{remark}

La existencia del polinomio irreducible de un elemento algebraico no es
obvia y la veremos como consecuencia del siguiente teorema. También es
posible probarla usando que \(F[x]\) es un DFU. La condición de ser
mónico es solo para garantizar su unicidad. Si encontramos un polinomio
irreducible en \(F[x]\) que tiene a \(\alpha\) como raíz, basta
dividirlo por su coeficiente líder para convertirlo en mónico.
\End{remark}

\Begin{theorem}

Dada una extensión \(F\subset K\) y un elemento algebraico
\(\alpha\in K\), el polinomio irreducible de \(\alpha\) es el generador
mónico del núcleo del homomorfismo \(f\colon F[x]\rightarrow K\),
\(f(p(x))=p(\alpha)\). Es más, \(F[\alpha]\) es un cuerpo y \(f\) induce
un isomorfismo de extensiones de \(F\),
\[\frac{F[x]}{(p(x))}\cong F[\alpha].\] En particular,
\[[F[\alpha]:F]=\text{grado }\alpha.\] \End{theorem}

\Begin{proof}

El homomorfismo \(f\) está bien definido por el principio de
sustitución, ya que es el único tal que
\(f_{|_F}\colon F\hookrightarrow K\) es la inclusión y \(f(x)=\alpha\).

Ser \(\alpha\) algebraico equivale a \(\ker f\neq (0)\), pues los
elementos de \(\ker f\) son los polinomios en \(F[x]\) que tienen a
\(\alpha\) como raíz. En particular, \(\ker f\subsetneq F[x]\) ya que
los polinomios constantes no nulos no tienen raíces.

Supongamos que \(\alpha\) tiene polinomio irreducible \(p(x)\). Entonces
\(p(x)\in\ker f\), así que \((p(x))\subset\ker f\). Como \(F[x]\) es un
DIP, \((p(x))\) es maximal por ser \(p(x)\) irreducible, así que
\((p(x))=\ker f\).

Recíprocamente, supongamos que \(\ker f=(p(x))\) (este ideal es
principal porque \(F[x]\) es un DIP). Por el primer teorema de
isomorfía, \(f\) induce un isomorfismo
\[\overline{f}\colon\frac{F[x]}{(p(x))}\stackrel{\cong}\longrightarrow F[\alpha].\]
Como \(F[\alpha]\subset K\) es un dominio, el ideal
\((p(x))\subset F[x]\) es primo. Como \(F[x]\) es un DFU, esto equivale
a decir que \(p(x)\) es irreducible. Podemos además suponer que es
mónico, dividiendo por su coeficiente líder si fuera necesario. En estas
condiciones hemos visto arriba que el cociente es un cuerpo, más
cocretamente una extensión de \(F\) del mismo grado que \(p(x)\). Esto
implica que el anillo \(F[\alpha]\) es también un cuerpo, por ser
isomorfo al cociente. Es más, según vimos en el tema de factorización,
el isomorfismo \(\overline{f}\) se comporta sobre \(F\) como la
identidad, por tanto es un isomorfismo de extensiones, así que el grado
de \(F[\alpha]\) sobre \(F\) es también el de \(p(x)\). \End{proof}

El siguiente corolario se basa en el hecho de que \(F[x]\) es un dominio
euclídeo. Su importancia estriba en que da un método para calcular el
polinomio irreducible de un elemento algebraico sin necesidad de
comprobar la irreducibilidad por otros métodos.

\Begin{corollary}

Dada una extensión \(F\subset K\), el polinomio irreducible de un
elemento algebraico \(\alpha\in K\) es el polinomio mónico no nulo de
menor grado en \(F[x]\) que tiene a \(\alpha\) como raíz.
\End{corollary}

\Begin{corollary}

Dada una extensión \(F\subset K\) y un elemento algebraico
\(\alpha\in K\) de grado \(n\),
\(\{1,\alpha,\dots,\alpha^{n-1}\}\subset F[\alpha]\) es una base como
\(F\)-espacio vectorial. \End{corollary}

\Begin{proof}

El isomorfismo de extensiones del teorema anterior es también un
isomorfismo de \(F\)-espacios vectoriales. Sabemos que
\(\{1,\bar{x},\dots,\bar{x}^{n-1}\}\) es una base de \(F[x]/(p(x))\),
donde \(p(x)\) es el polinomio irreducible de \(\alpha\), así que su
imagen, \(\{1,\alpha,\dots,\alpha^{n-1}\}\), es una base de
\(F[\alpha]\). \End{proof}

\Begin{proposition}

Si \(F\subset K\) es una extensión, \(\alpha\in K\) y \(q(x)\in F[x]\)
es un polinomio no nulo que tiene a \(\alpha\) como raíz, entonces el
polinomio irreducible de \(\alpha\) divide a \(q(x)\), en particular el
grado de \(\alpha\) sobre \(F\) es menor o igual que el grado de
\(q(x)\). \End{proposition}

\Begin{proof}

Consideramos el homomorfismo \(f\colon F[x]\rightarrow K\) del teorema
anterior. Si \(p(x)\) es el polinomio irreducible de \(\alpha\),
\(\ker f=(p(x))\). Como \(\alpha\) es una raíz de \(q(x)\),
\(q(x)\in\ker f\), así que \(q(x)\) es un multiplo no nulo de \(p(x)\).
\End{proof}

\Begin{example}\textrm{\normalfont (Grado de algunos elementos)} Sea
\(F\subset K\) una extensión y \(\alpha\in K\) un elemento algebraico.

\begin{itemize}
\item
  No hay elementos de grado \(0\) porque los polinomios no nulos de
  grado \(0\) no tienen raíces.
\item
  El grado de \(\alpha\) es \(1\) si y solo si \(\alpha\in F\). En
  efecto, esto equivale a decir que \(\alpha\) es raíz de un polinomio
  mónico de grado \(1\) en \(F[x]\) (todos ellos irreducibles) que no
  puede ser otro que \(x-\alpha\).
\item
  El grado de \(\alpha\) es \(2\) si y solo si \(\alpha\notin F\) pero
  es raíz de un polinomio de grado \(2\) en \(F[x]\).
\item
  Dado \(\alpha\in K\) tal que \(\alpha\notin F\) pero
  \(\alpha^2\in F\), el grado de \(\alpha\) es \(2\) y su polinomio
  irreducible es \(x^2-\alpha^2\in F[x]\).
\item
  Si \(F\subset\mathbb R\), el grado de \(i\in\mathbb C\) sobre \(F\) es
  \(2\) pues \(i\notin F\) pero es raíz de \(x^2+1\in F[x]\), que es su
  polinomio irreducible.
\item
  Si \(n\in\mathbb Z\) es libre de cuadrados, el grado de
  \(\sqrt{n}\in\mathbb C\) sobre \(\mathbb Q\) es \(2\) pues
  \(\sqrt{n}\notin\mathbb Q\) pero es raíz de \(x^2-n\in\mathbb Q[x]\).
\item
  Si \(p\in\mathbb Z\) es primo, el grado de \(\sqrt[n]{p}\in\mathbb C\)
  sobre \(\mathbb Q\) es \(n\) puesto que es raíz del polinomio
  irreducible \(x^n-p\in\mathbb Q[x]\). Este polinomio es irreducible
  por el criterio de
  \href{static/rings/factorization/\#eisenstein}{Eisenstein} para el
  primo \(p\). Hay por tanto números complejos, incluso reales, de grado
  cualquiera sobre \(\mathbb Q\).
\item
  Si \(\mathbb C\subset K\) es una extensión, los únicos elementos
  algebraicos son los de \(\mathbb C\) ya que los únicos polinomios
  irreducibles en \(\mathbb C[x]\) son los de grado \(1\), así que todo
  elemento algebraico tiene grado \(1\). Deducimos por tanto que la
  única extensión finita de \(\mathbb C\) es la trivial.
\end{itemize}

\End{example}

Veamos que los homomorfismos de extensiones de \(F\) preservan raíces de
polinomios con coeficientes en \(F\).

\Begin{proposition}

Dadas dos extensiones \(F\subset K\) y \(F\subset L\) del mismo cuerpo
\(F\) y un homomorfismo de extensiones \(f\colon K\rightarrow L\), si
\(\alpha\in K\) es raíz de un polinomio \(p(x)\in F[x]\) entonces
\(f(\alpha)\in L\) también es raíz de \(p(x)\). \End{proposition}

\Begin{proof}

Como \(f\colon K\rightarrow L\) es un homomorfismo de extensiones, \(f\)
deja fijo a \(F\). Si \(p(x)=a_nx^n+\cdots+a_1x+a_0\) con
\(a_i\in F\) y \(\alpha\in K\) es una raíz entonces
\[a_n\alpha^n+\cdots+a_1\alpha+a_0=0,\] por tanto
\[\begin{array}{rcl} 0&=&f(0)\cr &=&f(a_n\alpha^n+\cdots+a_1\alpha+a_0)\cr &=&f(a_n)f(\alpha)^n+\cdots+f(a_1)f(\alpha)+f(a_0)\cr &=&a_nf(\alpha)^n+\cdots+a_1f(\alpha)+a_0, \end{array}\]
así que \(f(\alpha)\in L\) también es raíz de \(p(x)\). \End{proof}

\Begin{example}\textrm{\normalfont (Extensiones no isomorfas del mismo grado)}
Las extensiones \(\mathbb Q[i]\) y \(\mathbb Q[\sqrt{2}]\) de
\(\mathbb Q\) tienen grado \(2\) pero no son isomorfas porque el
polinomio \(x^2+1\in\mathbb Q[x]\) tiene raíces en \(\mathbb Q[i]\) pero
no en \(\mathbb Q[\sqrt{2}]\subset\mathbb R\). \End{example}

\Begin{proposition}

Dada una extensión \(F\subset K\), \(\alpha\in K\) es trascendente si y
solo si hay un isomorfismo \(F[x]\cong F[\alpha]\) que deja fijo a
\(F\). \End{proposition}

\Begin{proof}

En virtud del teorema anterior, el elemento \(\alpha\) es trascendente
si y solo si el homomorfismo \(f\colon F[x]\rightarrow K\),
\(f(p(x))=p(\alpha)\), tiene núcleo trivial. Por el primer teorema de
isomorfía, esto equivale a que \(f\) induzca un isomorfismo
\[F[x]\cong \frac{F[x]}{(0)}\cong\operatorname{im} f=F[\alpha]\]
definido por la misma fórmula \(p(x)\mapsto p(\alpha)\). Este
isomorfismo obviamente deja fijo a \(F\). \End{proof}

\Begin{corollary}

Dada una extensión \(F\subset K\) y \(\alpha\in K\) tal que
\(F[\alpha]\) tiene dimensión finita como \(F\)-espacio vectorial,
\(\alpha\) es algebraico. \End{corollary}

\Begin{proof}

No puede ser trascendente porque el anillo de polinomios \(F[x]\) no
tiene dimensión finita sobre \(F\).\\
\End{proof}

\Begin{corollary}

Si \(F\subset K\) es una extensión finita, todo \(\alpha\in K\) es
algebraico. \End{corollary}

\Begin{proof}

Es consecuencia de que \(F[\alpha]\subset K\) es un sub-\(F\)-espacio
vectorial.\\
\End{proof}

\Begin{proposition}

Dadas dos extensiones consecutivas \(F\subset K\subset L\), si
\(F\subset L\) es finita entonces también lo son \(F\subset K\) y
\(K\subset L\). \End{proposition}

\Begin{proof}

Como \(K\) es un sub-\(F\)-espacio vectorial de \(L\), si \(F\subset L\)
es finita entonces \(F\subset K\) también. Es más, como \(F\subset K\),
cualquier conjunto de generadores de \(L\) como \(F\)-espacio vectorial
también lo genera como \(K\)-espacio vectorial, así que \(K\subset L\)
también es finita.\\
\End{proof}

\Begin{proposition}

Dadas dos extensiones finitas consecutivas \(F\subset K\subset L\),
\(F\subset L\) es finita de grado \[[L:F]=[L:K][K:F].\]
\End{proposition}

\Begin{proof}

Dada una base \(\{x_1,\dots, x_p\}\subset K\) como \(F\)-espacio
vectorial y una base \(\{y_1,\dots, y_q\}\subset L\) como
\(K\)-espacio vectorial, afirmamos que
\[\{x_iy_j\}_{\substack{1\leq i\leq p\cr 1\leq j\leq q}}\subset L\]
es una base como \(F\)-espacio vectorial. Hemos de ver que todo elemento
de \(L\) se puede expresar de manera única como combinación lineal de
este conjunto con coeficientes en \(F\). La base de \(L\) como
\(K\)-espacio vectorial nos garantiza que todo \(\alpha\in L\) se puede
escribir de manera única como
\[\alpha=\beta_1y_1+\cdots+\beta_qy_q,\] con \(\beta_j\in K\). La
base de \(K\) como \(F\)-espacio vectorial nos asegura que cada uno de
estos coeficientes se puede expresar de manera única como
\[\beta_j=\gamma_{1j}x_1+\cdots+\gamma_{pj}x_p\] con
\(\gamma_{ij}\in F\). Por tanto
\[\alpha=\sum_{i=1}^p\sum_{j=1}^q\gamma_{ij}x_iy_j\] y esta
expresión es única.\\
\End{proof}

\Begin{remark}

En las condiciones del enunciado anterior, decimos que \(K\) es una
\textbf{extensión intermedia} de \(F\subset L\). Decimos que es
\textbf{estricta} si ninguna de las dos inclusiones es una igualdad.
\End{remark}

\Begin{corollary}

Dada una extensión \(F\subset K\) y elementos algebraicos
\(\alpha_1,\dots,\alpha_n\in K\), la extensión
\(F\subset F[\alpha_1,\dots,\alpha_n]\) es finita. \End{corollary}

\Begin{proof}

Por inducción en \(n\). Para \(n=1\) está probado en el teorema
anterior. Supongamos que \(F\subset F[\alpha_1,\dots,\alpha_{n-1}]=L\)
es finita. Como \(\alpha_n\) es algebraico sobre \(F\), también lo es
sobre \(L\), así que
\(L\subset L[\alpha_n]=F[\alpha_1,\dots,\alpha_n]\) es finita. El
corolario se deduce ahora de la proposición anterior. \End{proof}

\Begin{example}\textrm{\normalfont ($\mathbb Q[\sqrt[3]{2},i]$)}
Consideremos la extensión \(\mathbb Q\subset\mathbb Q[\sqrt[3]{2},i]\).
Tenemos que
\[\mathbb Q\subset\mathbb Q[\sqrt[3]{2}]\subset\mathbb Q[\sqrt[3]{2},i].\]
Ya hemos visto que la extensión
\(\mathbb Q\subset\mathbb Q[\sqrt[3]{2}]\) tiene grado \(3\). Además,
como \(\mathbb Q[\sqrt[3]{2}]\subset\mathbb R\), la extensión
\(\mathbb Q[\sqrt[3]{2}]\subset\mathbb Q[\sqrt[3]{2},i]\) tiene grado
\(2\). Por tanto
\[\begin{array}{rcl}[\mathbb Q[\sqrt[3]{2},i]:\mathbb Q]&=&[\mathbb Q[\sqrt[3]{2},i]:\mathbb Q[\sqrt[3]{2}]][\mathbb Q[\sqrt[3]{2}]:\mathbb Q]\cr &=&2\cdot 3=6.\end{array}\]
\End{example}

\Begin{corollary}

Dada una extensión \(F\subset K\), el subconjunto \(L\subset K\) formado
por los elementos de \(K\) que son algebraicos sobre \(F\) es un
subcuerpo tal que \(F\subset L\). \End{corollary}

\Begin{proof}

Los elementos de \(F\) son algebraicos de grado \(1\) sobre \(F\), así
que \(F\subset L\). Veamos que \(L\subset K\) es un subanillo. Tenemos
que \(0,1\in F\subset L\). Dados \(\alpha,\beta\in L\), por el corolario
anterior \(F\subset F[\alpha,\beta]\) es una extensión finita, así que
todos sus elementos son algebraicos. Como
\(\alpha+\beta,\alpha\beta,-\alpha\in F[\alpha,\beta]\), deducimos que
estos tres elementos son en efecto algebraicos. Esto prueba que
\(L\subset K\) es un subanillo. Además, si \(\alpha\neq 0\) entonces
\(\alpha^{-1}\in F[\alpha,\beta]\), que por lo mismo será también
algebraico, por tanto \(L\) es un cuerpo. \End{proof}

\Begin{corollary}

Dadas extensiones \(F\subset K\subset L\):

\begin{itemize}
\tightlist
\item
  \([K:F]=[L:F]\Rightarrow K=L\).
\item
  \([L:F]=[L:K]\Rightarrow F=K\).
\end{itemize}

\End{corollary}

\Begin{proof}

Usando la fórmula para el grado de extensiones consecutivas vemos que en
el primero caso \([L:K]=1\) y en el segundo \([K:F]=1\), así que basta
usar que la única extensión de grado \(1\) es la trivial. \End{proof}

\Begin{corollary}

Si \(F\subset K\) es una extensión de grado \([K:F]=p\) primo entonces
no posee extensiones intermedias estrictas. \End{corollary}

\Begin{proof}

Dada una posible extensión intermedia \(F\subset L\subset K\), tenemos
que \(p=[K:F]=[K:L][L:F]\). Por ser \(p\) primo esto implica que bien
\([K:F]=[K:L]\) o bien \([K:F]=[L:F]\), es decir \(F=L\) o \(K=L\).\\
\End{proof}

\Begin{corollary}

Dada una extensión finita \(F\subset K\), el grado de cualquier
\(\alpha\in K\) divide a \([K:F]\). \End{corollary}

\Begin{proof}

Basta observar que tenemos extensiones sucesivas
\(F\subset F[\alpha]\subset K\) y por tanto
\([K:F]=[K:F[\alpha]][F[\alpha]:F]\). \End{proof}

\Begin{corollary}

Dada una extensión finita \(F\subset K\), existe una cantidad finita de
elementos \(\alpha_1,\dots,\alpha_n\in K\) tales que
\(K=F[\alpha_1,\dots,\alpha_n]\), que denominamos \textbf{generadores}
de \(K\) sobre \(F\). \End{corollary}

\Begin{proof}

Por inducción en el grado. Si \([K:F]=1\) no hay nada que demostrar pues
\(K=F\). Supongamos que \([K:F]=n>1\) y que el resultado es cierto para
extensiones de grado \({<}n\). Entonces, como la inclusión
\(F\subsetneq K\) es estricta ha de existir \(\alpha_1\in K\) tal que
\(\alpha_1\notin F\). Por tanto \(F\subsetneq F[\alpha_1]\subset K\).
Esto implica que \([F[\alpha_1]:F]>1\) así que
\[\begin{array}{rcl}n&=&[K:F]\cr &=&[K:F[\alpha_1]][F[\alpha_1]:F]\cr &>&[K:F[\alpha_1]].\end{array}\]
Entonces, por hipótesis de inducción, han de existir
\(\alpha_2,\dots,\alpha_n\in K\) tales que
\[\begin{array}{rcl}K&=&F[\alpha_1][\alpha_2,\dots,\alpha_n]\cr &=&F[\alpha_1,\dots,\alpha_n].\end{array}\]\\
\End{proof}

En la siguiente sección veremos que, para extensiones contenidas en
\(\mathbb{C}\) basta uno.

\hypertarget{elementos-primitivos}{%
\subsection{Elementos primitivos}\label{elementos-primitivos}}

Recuerda que \(\alpha\in\mathbb{C}\) es una \textbf{raíz múltiple} de
\(f(x)\in\mathbb{C}[x]\) si \((x-\alpha)^2|f\).

\Begin{proposition}

Un polinomio \(f(x)\in\mathbb{C}[x]\) tiene una raíz múltiple
\(\alpha\in\mathbb{C}\) si y solo si \(\alpha\) es raíz de \(f\) y de su
derivada \(f'\). \End{proposition}

\Begin{proof}

Sabemos que \(\alpha\) es raíz de \(f\) si y solo si \((x-\alpha)|f\),
es decir, si y solo si \(f(x)=g(x)(x-a)\) para cierto
\(g(x)\in\mathbb{C}[x]\). Por tanto, \(\alpha\) es una raíz múltiple de
\(f(x)\) si y solo si \((x-\alpha)|g\), es decir, si y solo si
\(\alpha\) es también una raíz de \(g\). La derivada de \(f\) es
\[f'(x)=g'(x)(x-\alpha)+g(x),\] luego \(f'(\alpha)=g(\alpha)\), así que
\(\alpha\) es una raíz de \(f'\) si y solo si es raíz de \(g\).
\End{proof}

\Begin{proposition}

Un polinomio \(f(x)\in\mathbb{C}[x]\) tiene alguna raíz múltiple si y
solo si \(f\) y \(f'\) no son coprimos. \End{proposition}

\Begin{proof}

\(\Rightarrow\) Si \(\alpha\) es una raíz múltiple de \(f\), hemos visto
antes que también es raíz de \(f'\), por tanto \(x-\alpha\) divide tanto
a \(f\) como a \(f'\).

\(\Leftarrow\) Si \(f\) y \(f'\) no son coprimos, entonces
\(\operatorname{mcd}(f,f')=g(x)\) es un polinomio no constante. Como
\(\mathbb{C}\) es algebraicamente cerrado, \(g(x)\) tiene alguna raíz
\(\alpha\in\mathbb{C}\). Es más, como \(g|f\) y \(g|f'\), \(\alpha\)
también es raíz de \(f\) y de \(f'\), luego es una raíz múltiple de
\(f\). \End{proof}

\Begin{proposition}

Dada una extensión \(F\subset\mathbb{C}\), si \(f(x)\in F[x]\) es
irreducible entonces \(f\) y \(f'\) son coprimos, en particular \(f\) no
tiene raíces múltiples en \(\mathbb{C}\). \End{proposition}

\Begin{proof}

Como \(f\) es irreducible, no es constante, así que \(f'\neq 0\). Sea
\(g=\operatorname{mcd}(f,f')\). Si \(g\) no es constante, entonces \(g\)
y \(f\) son asociados, ya que \(g|f\) y \(f\) es irreducible. Podemos
pues suponer que \(g=f\). Entonces \(f|f'\), pero esto es imposible
porque \(f'\neq 0\), así que el grado de \(f'\) es \(<\) el grado de
\(f\). \End{proof}

\Begin{definition}

Dada una extensión finita \(F\subset K\), decimos que \(\alpha\in K\) es
un \textbf{elemento primitivo} si \(K=F[\alpha]\). \End{definition}

Como de costumbre, en el enunciado del siguiente resultado ``casi todo''
significa ``todo menos una cantidad finita''.

\Begin{lemma}

Dada una extensión finita \(F\subset K\) tal que \(K\subset\mathbb{C}\)
y \(K=F[\alpha,\beta]\), \(\gamma=\beta+c\alpha\) es un elemento
primitivo para casi todo \(c\in F\). \End{lemma}

\Begin{proof}

Sean \(f(x),g(x)\in F[x]\) los polinomios irreducibles de
\(\alpha,\beta\in K\), respectivamente. Supongamos que sus grados
respectivos son \(m,n\geq 1\). Sean \(\alpha_1,\dots,\alpha_m\) y
\(\beta_1,\dots,\beta_n\) sus raíces en \(\mathbb{C}\), con
\(\alpha=\alpha_1\) y \(\beta=\beta_1\). Como \(f\) y \(g\) no tienen
raíces múltiples por ser irreducibles, los \(\alpha_i\) son todos
distintos, y también los \(\beta_j\). Dado \(c\in F\), denotemos
\[\gamma_{ij}=\beta_j+c\alpha_i.\] Veamos que, si
\((i,j)\neq (k,l)\), la igualdad \(\gamma_{ij}=\gamma_{kl}\) solo
puede ser cierta para un único valor de \(c\in F\). En efecto, esto es
cierto pues equivale a \[c(\alpha_i-\alpha_k)=\beta_l-\beta_j.\] Si
\(i\neq k\) entonces \(\alpha_i\neq \alpha_k\) y podemos despejar
\(c\), que sería única. Si \(i=k\) entonces \(j\neq l\), luego
\(\beta_l\neq \beta_j\) y no hay ningún valor de \(c\) que satisfaga
la ecuación. Por tanto, para casi todos los \(c\in F\), los
\(\gamma_{ij}\) son todos distintos. Fijemos tal \(c\in F\),
necesariamente no nula, y demostremos que \(\gamma=\gamma_{11}\) es un
elemento primitivo.

Consideramos la extensión intermedia
\(F\subset F[\gamma]\subset F[\alpha,\beta]\). Bastará demostrar que
\(\alpha\in F[\gamma]\), ya que entonces también
\(\beta=\gamma-c\alpha\in F[\gamma]\), y por tanto tendríamos la otra
inclusión \(F[\gamma]\supset F[\alpha,\beta]\).

Como \(g(x)\in F[x]\), \(h(x)=g(\gamma-cx)\in F[\gamma][x]\). Tenemos
que \(\alpha\in \mathbb{C}\) es raíz de \(h\) ya que
\(h(\alpha)=g(\gamma-c\alpha)=g(\beta)=0\). También es raíz de
\(f\in F[x]\), que es su polinomio mínimo. Veamos que no poseen más
raíces complejas en común. En efecto, si algún otro \(\alpha_i\),
\(i>0\), fuera raíz de \(h\), entonces
\(0=h(\alpha_i)=g(\gamma-c\alpha_i)\). Como las raíces de \(g\) son
los \(\beta_j\), tendríamos que \(\gamma-c\alpha_i=\beta_j\), así que
\(\gamma_{11}=\beta_j+c\alpha_i=\gamma_{ij}\), lo cual es imposible
porque \(i\neq 1\). De aquí deducimos que
\(\operatorname{mcd}(f,h)=x-\alpha\) en \(\mathbb{C}[x]\). El divisor
común máximo de dos polinomios está bien definido salvo producto por
constantes no nulas. Es más, divisor común máximo de dos polinomios en
\(F[\gamma][x]\) lo es también en \(\mathbb{C}[x]\), ya que toda
división euclídea en el primero lo es también en el segundo, así que que
el resultado de aplicar el arlgoritmo de Euclides en el primero es
también válido en el segundo. Esto demuestra que
\(x-\alpha\in F[\gamma][x]\), así que en efecto \$
\alpha\in F{[}\gamma{]}\$. \End{proof}

\Begin{theorem}\textrm{\normalfont (del elemento primitivo)} Toda
extensión finita contenida en \(\mathbb{C}\) posee un elemento
primitivo. \End{theorem}

\Begin{proof}

Sea \(F\subset K\) una extensión finita. Vimos al final de la sección
anterior que estaba generada por una cantidad finita de elementos
\(\alpha_1,\dots,\alpha_n\in K\), \(K=F[\alpha_1,\dots,\alpha_n]\).
Demostraremos este teorema por inducción en el número \(n\) de
generadores. Para \(n=1\) no hay nada que demostrar. Probémoslo para
\(n\) generadores suponiendo el resultado cierto para \(n-1\). Aplicando
la hipótesis de inducción,
\(F[\alpha_1,\dots,\alpha_{n-1}]=F[\beta]\), así que
\(K=F[\beta,\alpha_n]\), que por el lema anterior posee un elemento
primitivo. \End{proof}

Este teorema es cierto bajo hipótesis mucho más generales, pero la
prueba se complica.

\hypertarget{construcciones-con-regla-y-compuxe1s}{%
\subsection{Construcciones con regla y
compás}\label{construcciones-con-regla-y-compuxe1s}}

\Begin{definition}

Un punto, recta o circunferencia del plano \(\mathbb R^2\) se considera
\textbf{construido} en los siguientes casos:

\begin{itemize}
\tightlist
\item
  Los puntos \((0,0)\) y \((1,0)\).
\end{itemize}

\begin{figure}
\centering
\includegraphics{static/images/constructible-11.png}
\caption{Puntos constructibles}
\end{figure}

\begin{itemize}
\tightlist
\item
  Las rectas que pasan por dos puntos construidos.
\end{itemize}

\begin{figure}
\centering
\includegraphics{static/images/constructible-12.png}
\caption{Recta constructible}
\end{figure}

\begin{itemize}
\tightlist
\item
  Las circunferencias de centro un punto construido que pasan por otro
  punto construido.
\end{itemize}

\begin{figure}
\centering
\includegraphics{static/images/constructible-13.png}
\caption{Circunferencia constructible}
\end{figure}

\begin{itemize}
\tightlist
\item
  El punto de intersección de dos rectas construidas.
\end{itemize}

\begin{figure}
\centering
\includegraphics{static/images/constructible-14.png}
\caption{Intersección de rectas constructibles}
\end{figure}

\begin{itemize}
\tightlist
\item
  Los puntos de intersección de dos circunferencias construidas.
\end{itemize}

\begin{figure}
\centering
\includegraphics{static/images/constructible-16.png}
\caption{Intersección de circunferencias constructibles}
\end{figure}

\begin{itemize}
\tightlist
\item
  Los puntos de intersección de una recta y una circunferencia
  construidas.
\end{itemize}

\begin{figure}
\centering
\includegraphics{static/images/constructible-15.png}
\caption{Intersección de recta y circunferencia constructible}
\end{figure}

Un número real \(a\in\mathbb R\) es \textbf{constructible} si su valor
absoluto \(|a|\) es la distancia entre dos puntos constructibles.

\End{definition}

Deducimos que además podemos construir:

\begin{itemize}
\tightlist
\item
  El punto medio entre dos puntos construidos.
\end{itemize}

\begin{figure}
\centering
\includegraphics{static/images/constructible-17.png}
\caption{Punto medio}
\end{figure}

\begin{itemize}
\tightlist
\item
  La recta perpendicular a una recta construida que pasa por un punto
  consruido.
\end{itemize}

\begin{figure}
\centering
\includegraphics{static/images/constructible-0.png}
\caption{Perpendicular sobre un punto de la recta}
\end{figure}

\begin{figure}
\centering
\includegraphics{static/images/constructible-18.png}
\caption{Perpendicular sobre un punto exterior}
\end{figure}

\begin{itemize}
\tightlist
\item
  La recta paralela a una recta construida que pasa por un punto
  consruido.
\end{itemize}

\begin{figure}
\centering
\includegraphics{static/images/constructible-1.png}
\caption{Paralela}
\end{figure}

\begin{itemize}
\tightlist
\item
  Los puntos que están a una distancia constructible de un punto
  construido dentro de una recta construida. Intuitivamente esta
  propiedad nos dice que podemos transportar distancias constructibles.
\end{itemize}

\begin{figure}
\centering
\includegraphics{static/images/constructible-2.png}
\caption{Transporte de longitud}
\end{figure}

\Begin{proposition}

Un punto \((a,b)\in\mathbb R^2\) es constructible si y solo si sus
coordenadas \(a,b\in\mathbb R\) son números constructibles.
\End{proposition}

\Begin{proof}

\(\Rightarrow\) Trazando paralelas y perpendiculares por puntos
constructibles, podemos construir los ejes de coordenadas y las
proyecciones de \((a,b)\) sobre los mismos. La distancia de las
proyecciones al origen son \(|a|\) y \(|b|\), así que las coordenadas
son constructibles.

\begin{figure}
\centering
\includegraphics{static/images/constructible-3.png}
\caption{Coordenadas}
\end{figure}

\(\Leftarrow\) Recíprocamente, asi \(a\) y \(b\) son constructibles
podemos construir los puntos sobre los ejes de coordenadas que están a
distancia \(|a|\) y \(|b|\) del origen, es decir, \((\pm a,0)\) y
\((0,\pm b)\), y obtener \((a,b)\) como punto de intersección de las
paralelas a los ejes que pasan por \((a,0)\) y \((0,b)\). \End{proof}

\Begin{proposition}

El subconjunto de \(\mathbb R\) formado por los números constructibles
es un cuerpo. \End{proposition}

\Begin{proof}

El \(0\) y el \(1\) son constructibles ya que el \((0,0)\) y el
\((1,0)\) están construidos.

Dados \(a\geq b\geq 0\) constructibles, podemos construir \(a+b\) y
\(a-b\) tomando a partir del origen puntos del eje horizonal a
distancias \(a\) y \(b\),

\begin{figure}
\centering
\includegraphics{static/images/constructible-4.png}
\caption{Suma}
\end{figure}

\begin{figure}
\centering
\includegraphics{static/images/constructible-5.png}
\caption{Resta}
\end{figure}

Por tanto también son constructibles \(-a-b\) y \(-a+b\). Esto demuestra
que la suma de dos números constructibles cualesquiera y el opuesto de
uno dado son constructibles.

Para construir el producto de dos números constructibles \(a,b> 0\)
usamos triángulos semejantes. Construimos primero el triángulo
ractángulo con base en el eje horizontal, de longitud 1, vértice en el
origen y altura \(a\). El triángulo semejante de base \(b\) tiene altura
\(ab\).

\begin{figure}
\centering
\includegraphics{static/images/constructible-6.png}
\caption{Producto}
\end{figure}

Esto demuestra que también son constructibles \((-a)b\), \(a(-b)\) y
\((-a)(-b)\), es decir, el producto de dos números constructibles
cualesquiera (multiplicar por \(0\) da \(0\), que es constructible). Con
esto hemos visto que los números constructibles forman un subanillo de
\(\mathbb R\).

La construcción del inverso de un número constructible \(a> 0\) se lleva
a cabo del mismo modo

\begin{figure}
\centering
\includegraphics{static/images/constructible-7.png}
\caption{Inverso}
\end{figure}

Por tanto \((-a)^{-1}=-a^{-1}\) también es constructible. Esto demuestra
que el anillo de los números constructibles es un cuerpo. \End{proof}

\Begin{remark}

El cuerpo de los números constructibles contiene a \(\mathbb Q\) ya que
está contenido en \(\mathbb R\) y cualquier racional se puede obtener a
partir del \(1\) sumando, tomando opuestos y dividiendo por números no
nulos. Esto se aplica también a cualquier cuerpo contenido en
\(\mathbb C\) pero obviamente no es válido para los cuerpos finitos
\(\mathbb Z/(p)\). \End{remark}

\Begin{proposition}

Si \(a\in\mathbb R\) es positivo \(a>0\) y constructible entonces
\(\sqrt{a}\) también es constructible. \End{proposition}

\Begin{proof}

Es consecuencia del conocido teorema de la media geométrica. En el eje
horizontal tomamos el punto a la izquierda del origen a distancia \(a\).
Trazamos una circunferencia que pase por él y que tenga centro en el
punto medio entre este punto y el \((1,0)\). La distancia del origen al
punto de corte con la circunferencia de la perpendicular al eje
horizontal es \(\sqrt{a}\).

\begin{figure}
\centering
\includegraphics{static/images/constructible-8.png}
\caption{Raíz cuadrada}
\end{figure}

\End{proof}

Hasta ahora hemos demostrado que podemos construir números
constructibles a partir del \(1\) sumando, restando, dividiendo por
números no nulos, y tomando raíces cuadradas de números positivos. Los
siguientes resultados demuestras que no hay más números constructibles
que los que se pueden obtener de este modo.

\Begin{proposition}

Dados cuatro puntos en \(\mathbb R^2\) cuyas coordenadas están en un
subcuerpo \(F\subset\mathbb R\), los puntos de intersección de las
rectas y circunferencias que se pueden dibujar apoyándose en dichos
puntos tienen coordenadas en \(F\) o en \(F[\sqrt{r}]\) para cierto
\(r\in F\) positivo \(r>0\) que no sea el cuadrado de ningún número de
\(F\). \End{proposition}

\Begin{proof}

Dados dos puntos \((a_0,b_0), (a_1,b_1)\in\mathbb R^2\), la recta
que pasa por ambos tiene ecuación
\[(a_1-a_0)(y-b_0)=(b_1-b_0)(x-a_0),\] y la circunferencia de
centro el primero que pasa por el segundo está definida por
\[(x-a_0)^2+(y-b_0)^2=(a_1-a_0)^2+(b_1-b_0)^2.\] Si las
coordenadas está en \(F\) entonces los coeficientes de ambas ecuaciones
también.

La intersección de dos de estas rectas tiene coordenadas en \(F\) porque
las soluciones de un sistema de ecuaciones lineales con coeficientes en
un cuerpo siempre están en dicho cuerpo.

Para hallar la intersección de una recta y una circunferencia,
despejamos una incógnita de la ecuación de la recta y la sustituimos en
la ecuación de la circunferencia. Esto nos da una ecuación de grado
\(2\) con coeficientes en \(F\). Para que esta ecuación tenga solución
su discriminante ha de ser \(D\geq 0\). En ese caso la solución está en
\(F[\sqrt{D}]\). Por tanto las coordenadas del punto de intersección
están en este cuerpo. Si \(D\) es el cuadrado de un número de \(F\)
entonces \(F[\sqrt{D}]=F\).

Para intersecar dos circunferencias, observamos que la diferencia de
ambas ecuaciones es de grado \(1\), por tanto este caso se reduce al
anterior.

\End{proof}

Recuerda que antes hemos visto que \([F[\sqrt{r}]:F]=2\) si \(r\in F\) y
\(\sqrt{r}\notin F\).

\Begin{theorem}

Dados números reales constructibles \(a_1,\dots,a_m\in\mathbb R\), hay
una cadena de extensiones
\[\mathbb Q=F_0\subset F_1\subset F_2\subset\cdots\subset F_n=K\]
tales que

\begin{itemize}
\item
  \(K\subset\mathbb R\) es un subcuerpo,
\item
  \(a_1,\dots,a_m\in K\),
\item
  Cada \(F_{i+1}=F_i[\sqrt{r_i}]\), \(0\leq i{<}n\), donde
  \(r_i\in F_i\) es un número positivo \(r_i>0\) que no es un
  cuadrado en \(F_i\).
\end{itemize}

En particular \([K:\mathbb Q]=2^n\). \End{theorem}

\Begin{proof}

La construtibilidad de los números \(a_i\) equivale a la de los puntos
\((a_i,0)\). Los puntos constructibles se construyen a partir de los
básicos, \((0,0)\) y \((1,0)\), trazando e intersecando rectas y
circunferencias mendiante los métodos permitidos. Los puntos básicos
tienen coordenadas en \(\mathbb Q\). Por la propisición anterior, los
puntos que se construyen a partir de ellos tendrán coordenadas en
extensiones sucesivas de \(\mathbb Q\) obtenidas al añadir nuevas raíces
cuadradas de números positivos, por tanto el teorema se sigue de la
proposición anterior por inducción. La observación sobre el grado se
siguie de la fórmula del grado para extensiones intermedias, que en este
caso nos dice que
\[[K:\mathbb Q]=\prod_{i=0}^{n-1}[F_{i+1}:F_i]=2^n\] ya que por el
tercer apartado \([F_{i+1}:F_i]=2\). \End{proof}

La cantidad de raíces cuadradas que hemos de añadir a \(\mathbb Q\) para
construir \(K\) (\(n\) según la notación del teorema) no tiene relación
con la cantidad de números constructibles \(a_1,\dots,a_m\) que
queremos que \(K\) posea.

\Begin{corollary}

Los números constructibles son algebraicos sobre \(\mathbb Q\) y el
grado de un número constructible es siempre una potencia de \(2\).
\End{corollary}

\Begin{proof}

Por el teorema anterior, si \(a\in \mathbb R\) es constructible entonces
\(a\in K\) para cierta extensión finita \(\mathbb{Q}\subset K\) de grado
\(2^n\). En particular \(a\) es algebraico sobre \(\mathbb Q\) y su
grado divide a \(2^n\), así que ha de ser una potencia de \(2\).
\End{proof}

\Begin{example}\textrm{\normalfont (Números constructibles de grado $2^m$ cualquiera)}
Si \(p\in\mathbb Z\) es un primo positivo, \(\sqrt[n]{p}\) es
constructible si y solo si \(n\) es una potencia de \(2\). Sabemos que
este número tiene grado \(n\) sobre \(\mathbb Q\), así que solo puede
ser constructible si \(n=2^m\). Además en este caso podemos ver por
inducción en \(m\) que de hecho es constructible. Para \(m=0\) es obvio
porque es entero y si \(\sqrt[2^{m-1}]{p}\) es constructible entonces
\[\sqrt[2^m]{p}=\sqrt{\sqrt[2^{m-1}]{p}}\] también, por ser la raíz
cuadrada de un número constructible. \End{example}

\Begin{remark}

Más adelante veremos que hay números cuyo grado es una potencia de \(2\)
pero que no son constructibles, por ejemplo, las raíces reales del
polinomio \(x^4-6x+3\in\mathbb{Q}[x]\), que al ser irreducible tienen
grado \(4=2^2\). \End{remark}

\Begin{definition}

Un ángulo \(\theta\in[0,2\pi)\) es \textbf{constructible} si el número
\(\cos \theta\in\mathbb R\) es constructible. \End{definition}

Por la construcción geométrica de senos y cosenos, está claro que la
definición anterior es equivalente a decir que \(\sin \theta\) es
constructible, o que la recta que pasa por el origen y hace ángulo
\(\theta\) con el eje horizontal es constructible, o más generalmente
que podemos construir la recta que pasa por un punto constructible y que
hace ángulo \(\theta\) con otra recta constructible que pasa por él.

\begin{figure}
\centering
\includegraphics{static/images/constructible-9.png}
\caption{Ángulo constructible}
\end{figure}

Veamos que en general es imposible trisecar un ángulo cualquiera con
regla y compás.

\Begin{proposition}

El ángulo de \(60º\) es constructible pero su trisección no.
\End{proposition}

\Begin{proof} z Este ángulo se puede construir porque
\(\cos 60º=\frac{1}{2}\) es constructible. Cada ángulo de su trisección
tendría \(20º\) y el ángulo de \(20º\) no es constructible. En efecto,
la siguiente fórmula trigonométrica es cierta en general
\[\cos 3\theta=4\cos^3\theta-3\cos\theta.\] Tomando \(\theta= 20º\)
deducimos que \(\alpha=\cos 20º\) es una raíz del polinomio
\(8x^3-6x-1\). Vamos a ver que este polinomio es irreducible sobre
\(\mathbb Q\), por tanto \(\alpha\) tendrá grado \(3\) sobre
\(\mathbb Q\), así que no podrá ser constructible. El polinomio
\(8x^3-6x-1\) es primitivo, por tanto es irreducible sobre \(\mathbb Q\)
si y solo si lo es sobre \(\mathbb Z\). Sobre \(\mathbb Z\) es
irreducible por el criterio de reducción módulo \(5\), ya que
\(3x^3-x-1\in\mathbb Z/(5)[x]\) tiene grado \(\leq 3\) pero no tiene
raíces. \End{proof}

\Begin{proposition}

Un polígono regular de \(p\) lados, \(p\in\mathbb Z\) primo, puede
construirse con regla y compás si y solo si \(p=2^n+1\).
\End{proposition}

\Begin{proof}

Esto equivale a la constructibilidad del ángulo de \(\frac{2\pi}{p}\)
radianes.

\(\Leftarrow\) Es un resultado de Gauss que no probaremos.

\(\Rightarrow\) El número complejo
\(\zeta=e^{\frac{2\pi i}{p}}=\cos\frac{2\pi}{p}+i\sin \frac{2\pi}{p}\)
es una raíz \(p\)-ésima de la unidad, es decir, una raíz del polinomio
\(x^p-1\). Este polinonimio factoriza como \[(x-1)(x^{p-1}+\cdots+x+1)\]
y como \(\zeta\neq 1\), \(\zeta\) es raíz del segundo, que se denomina
\(p\)-ésimo \textbf{polinomio ciclotómico} \[f(x)=x^{p-1}+\cdots+x+1.\]
Veamos que este polinomio es irreducible sobre \(\mathbb Q\). Para ello
hacemos el cambio de variable \(x=y+1\), que se corresponde con el
isomorfismo \(g\) que pasamos a definir.

Consideramos el único homomorfismo de anillos
\[g\colon \mathbb Q[x]\longrightarrow \mathbb Q[y]\] tal que
\(g_{|_{\mathbb{Q}}}\) es la inclusión
\(\mathbb{Q}\subset\mathbb{Q}[y]\) y \(g(x)=y+1\), que está bien
definido por el principio de sustitución. El homomorfismo \(g\) está
definido sobre un polinomio cualquiera \(h(x)\in\mathbb{Q}[x]\) como
\(g(h(x))=h(y+1)\). Análogamente, consideramos el único homomorfismo
\[g'\colon \mathbb Q[y]\longrightarrow \mathbb Q[x]\] tal que
\(g'_{|_{\mathbb{Q}}}\) es la inclusión
\(\mathbb{Q}\subset\mathbb{Q}[x]\) y \(g'(y)=x-1\). Sobre un polinomio
cualquiera \(h'(y)\in\mathbb{Q}[y]\), el homomorfismo \(g'\) está
definido como \(g'(h'(y))=h'(x-1)\). Es fácil comprobar que
\(g'\circ g=1_{\mathbb{Q}[x]}\) y \(g\circ g'=1_{\mathbb{Q}[y]}\), por
tanto \(g\) es un isomorfismo con inverso \(g^{-1}=g'\). En particular,
\(f(x)\) es irreducible en \(\mathbb{Q}[x]\) si y solo si
\(g(f(x))=f(y+1)\) es irreducible en \(\mathbb{Q}[y]\). Vamos a probar
esto último.

Como \(x^p-1=(x-1)f(x)\) entonces \[\begin{array}{rcl}
yf(y+1)&=&(y+1)^p-1\cr
&=& \sum_{n=1}^{p}\binom{p}{n}y^n.
\end{array}\] Aplicando la propiedad cancelativa en el dominio
\(\mathbb Q[y]\) obtenemos que \[\begin{array}{rcl}
p(y+1)&=&\sum_{n=1}^{p}\binom{p}{n}y^{n-1}.
\end{array}\] Este polinomio es irreducible por el criterio de
Eisenstein para el primo \(p\) ya que el coeficiente líder es \(1\), el
término independiente es \(p\), y \(p\) divide a \(\binom{p}{n}\) para
todo \(0{<}n{<}p\).

Por ser el polinomio ciclotómico \(f(x)\) irreducible y tener a
\(\zeta\) como raíz, deducimos que \(\zeta\) tiene grado \(p-1\) sobre
\(\mathbb Q\). Si \(\frac{2\pi}{p}\) fuera constructible, tendríamos un
cuerpo \(K\subset\mathbb R\) tal que
\(\cos\frac{2\pi}{p}, \sin\frac{2\pi}{p}\in K\) y \([K:\mathbb Q]=2^n\).
Como \(K\) está contenido en los reales, \([K[i]:K]=2\), luego
\([K[i]:\mathbb Q]=[K[i]:K][K:\mathbb Q]=2^{n+1}\). Además,
\(\zeta\in K[i]\) luego el grado de \(z\), que es \(p-1\), ha de ser una
potencia de \(2\). \End{proof}

\Begin{example}\textrm{\normalfont (Primos de Fermat)} Los enteros
primos \(p\in\mathbb Z\) tales que el polígono regular de \(p\) lados se
puede construir con regla y compás, es decir los que son de la forma
\(p=2^n+1\), se denominan \textbf{primos de Fermat}. Solo se conocen los
siguientes: 3, 5, 17, 257 y 65537. No se sabe siquiera si hay una
cantidad finita o infinita de primos de Fermat. Este problema fue
planteado por Eisenstein en 1844 y permanece abierto.

La siguiente imagen, obtenida de
\href{https://en.wikipedia.org/wiki/Constructible_polygon}{Wikipedia},
muestra la construcción paso a paso de un polígono regular de 17 lados
con regla y compás. En el artículo de Wikipedia enlazado se puede
encontrar otra construccón de este polígono regular, así como una
construcción completa del de 257 lados y el comienzo de la construcción
del de 65537 lados.

\begin{figure}
\centering
\includegraphics{static/images/HeptadecagonConstructionAni.gif}
\caption{Heptadecágono}
\end{figure}

\End{example}
