
En esta sección supondremos sin necesida de mencionarlo explícitamente
que todos los cuerpos que consideremos son subcuerpos de \(\mathbb C\).

\hypertarget{el-grupo-de-galois}{%
\subsection{El grupo de Galois}\label{el-grupo-de-galois}}

\Begin{definition}

Dada una extensión \(F\subset K\), su \textbf{grupo de Galois}
\(G(K/F)\) es el conjunto de los automorfismos de \(F\subset K\).
\End{definition}

\Begin{remark}

La estructura de grupo del grupo de Galois es la composición. El
elemento unidad es la identidad. El grupo de Galois de la extensión
trivial es el grupo trivial \(G(F/F)=\{\operatorname{id}_{F}\}\).
Recuerda que si la extensión \(F\subset K\) es finita cualquier
homomorfismo de extensiones \(f\colon K\rightarrow K\) es un elemento de
\(G(K/F)\), y si además \(F=\mathbb Q\) entonces cualquier homomorfismo
de anillos \(f\colon K\rightarrow K\) es un elemento de \(G(K/F)\).
Recuerda también que todo elemento de \(G(K/F)\) es además un
isomorfismo de \(F\)-espacios vectoriales, pero no todo isomorfismo de
\(F\)-espacios vectoriales \(f\colon K\rightarrow K\) está en \(G(K/F)\)
ya que podría no preservar el producto en \(K\), o incluso el \(1\).
Asimismo, recuerda que todo elemento de \(G(K/F)\) preserva raíces de
polinomios con coeficientes en \(F\). \End{remark}

\Begin{example}\textrm{\normalfont ($G(\mathbb C/\mathbb R)$)} Un
homomorfismo de \(\mathbb R\)-espacios vectoriales
\(f\colon \mathbb C\rightarrow\mathbb C\) está determinado por la imagen
de los elementos de una base, por ejemplo \(\{1,i\}\subset\mathbb C\).
Para que \(f\in G(\mathbb C/\mathbb R)\) ha de ser un homomorfismo de
anillos, así que debe satisfacer \(f(1)=1\). También ha de preservar
raíces en \(\mathbb C\) de polinomios en \(\mathbb R[x]\). Las raíces
complejas de \(x^2+1\) son \(\pm i\), así que \(f\) ha de cumplir
\(f(i)=\pm i\). Por tanto los dos posibles elementos de
\(G(\mathbb C/\mathbb R)\) son los homomorfismos de
\(\mathbb R\)-espacios vectoriales definidos por
\[\begin{array}{rcl}f(1)&=&1,\cr f(i)&=&i,\end{array}\] y por
\[\begin{array}{rcl}f(1)&=&1,\cr f(i)&=&-i.\end{array}\] Algunos de
estos dos homomorfismos de \(\mathbb R\)-espacios vectoriales podría no
estar en \(G(\mathbb C/\mathbb R)\) pues podría no preservar el
producto, pero ambos lo preservan porque claramente el primero es la
identidad \(\operatorname{id}_{\mathbb C}\) y el segundo es la
conjugación, que denotaremos \(c\). Así que
\(G(\mathbb C/\mathbb R)=\{\operatorname{id}_{\mathbb C},c\}\), que
es un grupo cíclico de orden \(2\) generado por la conjugación \(c\),
que satisface \(c\circ c=\operatorname{id}_{\mathbb C}\). \End{example}

\Begin{proposition}

Si \(F\subset K\) es una extensión de grado \([K:F]=2\) entonces
\(K=F[\sqrt{\delta}]\) para cierto \(\delta\in F\) y
\(G(K/F)=\{\operatorname{id}_K,c\}\) es un grupo cíclico de orden
\(2\) cuyo generador \(c\) denominamos \textbf{conjugación} y está
caracterizado por satisfacer \(c(\sqrt{\delta})=-\sqrt{\delta}\).
\End{proposition}

\Begin{proof}

Como la extensión no es trivial, ha de existir algún \(\alpha\in K\) tal
que \(\alpha\notin F\). El grado de este elemento ha de dividir a \(2\).
Como no puede ser \(1\) porque \(\alpha\notin F\), ha de ser \(2\). La
extensión \(F\subset F[\alpha]\) también tiene grado \(2\) y
\(F[\alpha]\subset K\) por tanto \(K=F[\alpha]\). Si
\(x^2+ax+b\in F[x]\) es el polinomio irreducible de \(\alpha\), entonces
\[\alpha=\frac{-a\pm\sqrt{\delta}}{2}\] donde \(\delta=a^2-4b\in F\) es
el \textbf{discriminante}. Deducimos por tanto que
\(\sqrt{\delta}\in K\), \(\sqrt{\delta}\notin F\), y
\(K=F[\sqrt{\delta}]\). Sabemos que \(\{1,\sqrt{\delta}\}\subset K\)
es una base como \(F\)-espacio vectorial. Como cualquier \(f\in G(K/F)\)
preserva el \(1\) y las raíces de \(x^2-\delta\), tenemos solo dos
posibilidades:
\[\begin{array}{rcl}f(1)&=&1,\cr f(\sqrt{\delta})&=&\sqrt{\delta},\end{array}\]
y
\[\begin{array}{rcl}f(1)&=&1,\cr f(\sqrt{\delta})&=&-\sqrt{\delta}.\end{array}\]
El primero es la identidad \(\operatorname{id}_{K}\), que es obviamente
un isomorfismo de extensiones. El segundo es el que denominamos
conjugación \(c\). Dejamos como ejercicio probar que la conjugación, que
a priori es solo un homomorfismo de \(F\)-espacios vectoriales, es de
hecho un homomorfismo de extensiones. Solo queda por demostrar que
preserva el producto. \End{proof}

\Begin{example}\textrm{\normalfont ($G(\mathbb Q[\sqrt[3]{2}]/\mathbb Q)$)}
Aquí \(\sqrt[3]{2}\) denota la raíz cúbica de \(2\) real por lo que
\(\mathbb Q[\sqrt[3]{2}]\subset\mathbb R\). El resto de raíces cúbicas
de \(2\) son puramente complejas. Cualquier
\(f \in G(\mathbb Q[\sqrt[3]{2}]/\mathbb Q)\) ha de preservar las raíces
de \(x^3-2\in\mathbb Q[x]\). La única raíz de este polinomio que está en
\(\mathbb Q[\sqrt[3]{2}]\) es \(\sqrt[3]{2}\), ya que las otras dos
están en \(\mathbb{C}\setminus\mathbb{R}\), por tanto
\(f(\sqrt[3]{2})=\sqrt[3]{2}\). Una base de \(\mathbb Q[\sqrt[3]{2}]\)
como \(\mathbb Q\)-espacio vectorial está formada por las primeras tres
potencias de \(\sqrt[3]{2}\), es decir,
\(\{1,\sqrt[3]{2},(\sqrt[3]{2})^2\}\). Como \(f\) ha de preservar la
unidad y los productos, \(f\) tiene que mandar cada uno de los elementos
de esta base a sí mismo, así que necesariamente
\(f=\operatorname{id}_{\mathbb Q[\sqrt[3]{2}]}\), por tanto en este caso
el grupo de Galois es el tivial,
\(G(\mathbb Q[\sqrt[3]{2}]/\mathbb Q)=\{\operatorname{id}_{\mathbb Q[\sqrt[3]{2}]}\}\)
a pesar de que la extensión \(\mathbb Q\subset \mathbb Q[\sqrt[3]{2}]\)
no es trivial, es de grado \(3\). \End{example}

\Begin{definition}

Dado un cuerpo \(F\) y un polinomio mónico no constante
\(p(x)\in F[x]\), el \textbf{cuerpo de descomposición} de \(p(x)\) es
\(F[\alpha_1,\dots,\alpha_n]\), donde \(\alpha_1,\dots,\alpha_n\)
son las raíces complejas de \(p(x)\). \End{definition}

\Begin{proposition}

Toda extensión \(F\subset K\) de grado \(2\) es un cuerpo de
descomposición. \End{proposition}

\Begin{proof}

Ya hemos visto que \(K=F[\sqrt{\delta}]\) para cierto \(\delta\in F\),
entonces \(K\) es el cuerpo de descomposición de \(x^2-\delta\) ya que
las raíces complejas de este polinomio son \(\pm\sqrt{\delta}\) y
\(F[\sqrt{\delta},-\sqrt{\delta}]=F[\sqrt{\delta}]\). \End{proof}

El \textbf{grupo simétrico} de \(n\) letras, es decir el \textbf{grupo
de permutaciones} de \(\{1,\dots,n\}\), se denotará \(S_n\).

\Begin{proposition}

Dada una extensión \(F\subset K\), si \(K\) es el cuerpo de
descomposición de un polinomio \(p(x)\in F[x]\) con \(n\) raíces
distintas en \(K\), entonces hay un único homomorfismo inyectivo
\[\varphi\colon G(K/F)\longrightarrow S_n\] tal que, si
\(\alpha_1,\dots,\alpha_n\in K\) son la raíces de \(p(x)\) en \(K\) y
\(f\in G(K/F)\), la permutación \(\varphi(f)=\sigma\) es la única que
satisface la siguiente ecuación para todo \(i=1,\dots,n\),
\[f(\alpha_i)=\alpha_{\sigma(i)}.\] \End{proposition}

\Begin{proof}

Como \(f\) preserva raíces de polinomios con coeficientes en \(F\),
\(f\) ha de mandar el conjunto \(\{\alpha_1,\dots,\alpha_n\}\)
dentro de sí mismo. Además ha de hacerlo de manera biyectiva por ser
\(f\) un automorfismo, por tanto existe una única permutación
\(\sigma\in S_n\) que satisface la ecuación del enunciado. Esto me
permite definir la aplicación \(\varphi\) de manera única.

Veamos que \(\varphi\) es un homomorfismo de grupos. Por un lado
\(\varphi(\operatorname{id}_{K})\) es la permutación identidad ya que
\(\operatorname{id}_{K}(\alpha_i)=\alpha_i\). Por otro lado, dados
\(f,g\in G(K/F)\), si denotamos \(\varphi(f)=\sigma\) y
\(\varphi(g)=\tau\) entonces \[\begin{array}{rcl}
(f\circ g)(\alpha_i)&=&f(g(\alpha_i))\cr
&=&f(\alpha_{\tau(i)})\cr
&=&\alpha_{\sigma(\tau(i))}\cr
&=&\alpha_{(\sigma\circ\tau)(i)}.
\end{array}\] Por tanto \[\begin{array}{rcl}
\varphi(f\circ g)&=&\sigma\circ\tau\cr
&=&\varphi(f)\circ \varphi(g).
\end{array}\]

Comprobemos por último que \(\varphi\) es inyectivo. Para ello debemos
probar que si \(f\in G(K/F)\) es tal que \(\varphi(f)\) es le
permutación identidad entonces \(f=\operatorname{id}_{K}\). Por
definición \(\varphi(f)\) es la permutación identidad si y solo si
\(f(\alpha_i)=\alpha_i\) para todo \(i=1,\dots,n\). Como
\(K=F[\alpha_1,\dots,\alpha_n]\), todo elemento \(\alpha\in K\) se
puede escribir como polinomio en \(\alpha_1,\dots,\alpha_n\) con
coeficientes en \(F\), es decir
\[\alpha=\sum_{m_i\geq 0} b_{m_1,\dots,m_n}\alpha_1^{m_1}\cdots\alpha_n^{m_n}\]
para ciertos \(b_{m_1,\dots,m_n}\in F\), casi todos nulos (y no
necesariamente únicos, pero esto es irrelevante). Entonces
\[\begin{array}{rcl}
f(\alpha)&=&f\left(\sum_{m_i\geq 0} b_{m_1,\dots,m_n}\alpha_1^{m_1}\cdots\alpha_n^{m_n}\right)\cr
&=&\sum_{m_i\geq 0} f(b_{m_1,\dots,m_n}\alpha_1^{m_1}\cdots\alpha_n^{m_n})\cr
&=&\sum_{m_i\geq 0} f(b_{m_1,\dots,m_n})f(\alpha_1)^{m_1}\cdots f(\alpha_n)^{m_n}\cr
&=&\sum_{m_i\geq 0} b_{m_1,\dots,m_n}\alpha_1^{m_1}\cdots\alpha_n^{m_n}\cr
&=&\alpha,
\end{array}\] así que \(f=\operatorname{id}_{K}\). En las ecuaciones
anteriores hemos usado que \(f\) es un homomorfismo de anillos que deja
fijo a \(F\) y a las raíces \(\alpha_1,\dots,\alpha_n\in K\) de
\(p(x)\). \End{proof}

Uno homomorfismo \(\varphi\) como el del enunciado se denomina
\textbf{representación} del grupo de Galois como grupo de permutaciones.

\Begin{proposition}

Dadas dos extensiones consecutivas \(F\subset L\subset K\), tenemos que
\(G(K/L)\subset G(K/F)\). \End{proposition}

\Begin{proof}

En efecto, si \(f\colon K\rightarrow K\) es un isomorfismo de anillos
que deja fijo a \(L\) entonces también deja fijo a \(F\) ya que
\(F\subset L\). \End{proof}

Los subgrupos del grupo de Galois nos permiten construir extensiones
intermedias.

\Begin{definition}

Dada una extensión \(F\subset K\) y un subgrupo \(H\subset G(K/F)\)
definimos el \textbf{cuerpo fijo} de \(H\) del siguiente modo:
\[K^{H}=\{\alpha\in K\;|\; f(\alpha)=\alpha\;\forall f\in H\}.\]
\End{definition}

\Begin{proposition}

Dada una extensión \(F\subset K\) y un subgrupo \(H\subset G(K/F)\), el
cuerpo fijo \(K^H\) es un subcuerpo de \(K\) que contiene a \(F\),
\[F\subset K^H\subset K.\] \End{proposition}

\Begin{proof}

El conjunto \(K^H\) está contenido en \(K\) por definición. Es más,
cualquier \(\alpha\in F\) satisface \(f(\alpha)=\alpha\) para todo
\(f\in G(K/F)\), en particular para todo \(f\in H\), por tanto
\(F\subset K^H\).

Veamos que \(K^H\subset K\) es un subanillo. Obviamente
\(0,1\in F\subset K^H\). Si \(\alpha,\beta\in K^H\) entonces, dado
\(f\in H\), como \(f\colon K\rightarrow K\) es un homomorfismo de
anillos, \[\begin{array}{rcl}
f(\alpha+\beta)&=&f(\alpha)+f(\beta)\cr
&=&\alpha+\beta,\cr
f(-\alpha)&=&-f(\alpha)\cr
&=&-\alpha,\cr
f(\alpha\beta)&=&f(\alpha)f(\beta)\cr
&=&\alpha\beta.
\end{array}\] Por tanto \(\alpha+\beta,-\alpha,\alpha\beta\in K^H\).
Además \(K^H\) es un cuerpo porque si \(\alpha\neq 0\) entonces
\[\begin{array}{rcl}
f(\alpha^{-1})&=&f(\alpha)^{-1}\cr
&=&\alpha^{-1}.
\end{array}\] \End{proof}

\Begin{proposition}

Dada una extensión \(F\subset K\) y un subgrupo \(H\subset G(K/F)\)
tenemos que \(H\subset G(K/K^H)\). \End{proposition}

\Begin{proof}

Es obvio porque, por definición de \(K^H\), todos los automorfismos de
la extensión que están en \(H\) dejan fijo a \(K^H\). \End{proof}

\hypertarget{funciones-simuxe9tricas}{%
\subsection{Funciones simétricas}\label{funciones-simuxe9tricas}}

\Begin{definition}

Dado un anillo \(R\), un polinomio
\(f=f(u_1,\dots,u_n)\in R[u_1,\dots,u_n]\) y un elemento
\(\sigma\in S_n\). El polinomio \(\sigma(f)\in R[u_1,\dots,u_n]\) es
\[
\sigma(f)=f(u_{\sigma(1)},\dots, u_{\sigma(n)}).
\] La \textbf{órbita} de \(f\) es el conjunto de polinomios
\[\{\sigma(f)\mid \sigma\in S_n\}.\] Decimos que \(f\) es
\textbf{simétrico} si \(f=\sigma(f)\) para todo \(\sigma\in S_n\).
\End{definition}

\Begin{example}\textrm{\normalfont (Una permutación aplicada a un polinomio)}
Si tomamos el polinomio
\(f=2u_1^2u_3^2-3u_2\in\mathbb{Z}[u_1,u_2,u_3]\) y le aplicamos la
permutacición \[\sigma=\left(
\begin{array}{ccc}
1&2&3\cr
2&3&1
\end{array}
\right)=(1\;2\;3)\in S_3\] obtenemos el polinomio
\[\sigma(f)=2u_2^2u_1^2-3u_3.\] Consideranto las \(3!=6\)
permutaciones de \(S_3\), puedes comprobar que la órbita de \(f\) es el
conjunto \[
\{2u_1^2u_3^2-3u_2, 2u_2^2u_3^2-3u_1, 2u_1^2u_2^2-3u_3\}.
\] \End{example}

\Begin{remark}

La órbita de un polinomio en \(n\) variables tiene como máximo
\(|S_n|=n!\) elementos. Es más, el número de elementos de la órbita
divide a \(n!\). La órbita de un polinomio es un conjunto unitario si y
solo si es simétrico. Los polinomios simétricos forman un subanillo de
\(R[u_1,\dots,u_n]\). La aplicación
\[\sigma\colon R[u_1,\dots,u_n]\rightarrow R[u_1,\dots,u_n]\]
definida arriba es, por el principio de sustitución, el único
homomorfismo de anillos tal que \(\sigma_{|_R}\) es la inlcusión
\(R\subset R[u_1,\dots,u_n]\) y \(\sigma(u_i)=u_{\sigma(i)}\). Dadas
\(\sigma,\tau\in S_n\) y \(f\in R[u_1,\dots,u_n]\),
\[\sigma(\tau(f))=(\sigma\tau)(f),\] por tanto el producto de
permutaciones se corresponde con la composición de los homomorfismos
inducidos. En particular, estos últimos son automorfismos ya que el
inverso de \(\sigma\) será el definido por la permutación inversa
\(\sigma^{-1}\). Aquí usamos que la permutación identidad induce la
identidad. \End{remark}

\Begin{definition}

Los \textbf{polinomios simétricos} o \textbf{funciones simétricas
elementales} en \(n\) variables \(s_i\in R[u_1,\dots,u_n]\) son: \[
\begin{array}{rcl}
s_1&=&\displaystyle \sum_{1\leq i\leq n}u_i=u_1+u_2+\cdots+u_n,\cr
s_2&=&\displaystyle \sum_{1\leq i < j\leq n}u_iu_j=u_1u_2+u_1u_3+\cdots+u_{n-1}u_n,\cr
s_3&=&\displaystyle \sum_{1\leq i < j < k\leq n}u_iu_ju_k=u_1u_2u_3+\cdots+u_{n-2}u_{n-1}u_n,\cr
&\vdots&\cr
s_n&=&u_1\cdots u_n.
\end{array}
\] Es decir, para cada \(1\leq j\leq n\), \[
s_j=\sum_{1\leq i_1<\cdots<i_j\leq n}u_{i_1}\cdots u_{i_j}.
\] \End{definition}

\Begin{remark}

Las funciones simétricas elementales en \(n\) variables son, salvo
signo, los coeficientes del polinomio \[
\begin{array}{rcl}
P(x)&=&(x-u_1)\cdots (x-u_n)\cr
&=&x^n-s_1x^{n-1}+s_2x^{n-2}-\cdots+(-1)^ns_n\cr
&=&\displaystyle \sum_{i=0}^n(-1)^is_ix^{n-i}.
\end{array}
\] En la última línea denotamos \(s_0=1\).

En particular, dado un polinomio mónico \(f\in F[x]\) degrado \(n\) \[
\begin{array}{rcl}
P(x)&=&x^n-a_1x^{n-1}+a_2x^{n-2}-\cdots+(-1)^na_n\cr
&=&\displaystyle \sum_{i=0}^n(-1)^ia_ix^{n-i}\cr
&=&(x-\alpha_1)\cdots (x-\alpha_n)
\end{array}
\] con \(a_0=1\), cuyas \(n\) raíces complejas denotamos \(\alpha_i\),
sus coeficientes se obtienen al aplicarle las funciones simétricas
elementales a estas raíces, \[a_i=s_i(\alpha_1,\dots,\alpha_n).\]
\End{remark}

\Begin{theorem}\textrm{\normalfont (de las funciones simétricas)} Dado
un polinomio simétrico \(g\in R[u_1,\dots,u_n]\), existe un único
polinomio \(G\in R[z_1,\dots,z_n]\) tal que \(g=G(s_1,\dots,s_n)\).
\End{theorem}

\Begin{proof}

Por doble inducción, primero en el número de variables y luego en el
grado.

Para una sola variable, el resultado es obviamente cierto par cualquier
grado ya que \(s_1=u_1\) y basta tomar \(G=g(z_1)\). También es obvio
para polinomios de grado \(0\). Supongamos que es cierto para polinomios
de hasta \(n-1\) variables.

Consideramos el polinomio \(g_0=g(u_1,\dots,u_{n-1},0)\). Por
hipótesis de inducción existe \(G_0\in R[z_1,\dots,z_{n-1}]\) tal que
\(g_0=G_0(s'_1,\dots,s'_{n-1})\), donde las
\(s_i'\in R[u_1,\dots,u_{n-1}]\) son las funciones simétricas
elementales en \(n-1\) variables. El polinomio
\[p(u_1,\dots,u_n)=g(u_1,\dots,u_n)-G_0(s_1,\dots,s_{n-1})\] es
simétrico pues los polinomios simétricos forman un subanillo de
\(R[u_1,\dots,u_n]\). Por construcción,
\[p(u_1,\dots,u_{n-1},0)=g_0-G_0(s'_1,\dots,s'_{n-1})=0\] así que
\(u_n|p\). Como \(p\) es simétrico, esto implica que \(u_i|p\) para
todo \(1\leq i\leq n\), así que \(s_n=u_1\cdots u_n|p\). Ha de
existir por tanto un polinomio \(h\in R[u_1,\dots,u_n]\) tal que
\(p=s_nh\). Al ser \(p\) y \(s_n\) simétricos, \(h\) es también
simétrico. Como \(h\) es de menor grado que \(g\), por hipótesis de
inducción existe \(H\in R[z_1,\dots,z_n]\) tal que
\(h=H(s_1,\dots,s_n)\). Al ser \[
\begin{array}{rcl}
g&=&p+G  _0(s_1,\dots,s_{n-1})\cr
&=&s_nH(s_1,\dots,s_n)+G_0(s_1,\dots,s_{n-1})
\end{array}\] podemos tomar \(G=z_nH+G_0\). \End{proof}

\Begin{definition}

El \textbf{discriminante} en \(n\) variables es el polinomio
\[D=\prod_{1\leq i<j\leq n}(u_i-u_j)^2=(u_1-u_2)^2\cdots(u_{n-1}-u_n)^2.\]
\End{definition}

\Begin{remark}

El discriminante es simétrico y, dados \(\alpha_1, \dots,\alpha_n\),
tenemos que \(D(\alpha_1, \dots,\alpha_n)\) si y solo si
\(\alpha_i=\alpha_j\) para ciertos \(i\neq j\). Denotaremos
\(\Delta\in R[z_1,\dots,z_n]\) al único polinomio tal que
\(D=\Delta(s_1,\dots,s_n)\). \End{remark}

\Begin{example}\textrm{\normalfont (Discriminantes en pocas variables)}
Para \(n=1\) el discriminante es \(D=1\). Si \(n=2\), entonces \[
\begin{array}{rcl}
D&=&(u_1-u_2)^2\cr
&=&(u_1+u_2)^2-4u_1u_2\cr
&=&s_1-4s_2.
\end{array}
\] Recuerda que el discriminante de un polinomio de grado \(2\)
\[x^2-a_1x+a_2=(x-\alpha_1)(x-\alpha_2)\] es
\[a_1^2-4a_2=\Delta(a_1,a_2)=D(\alpha_1,\alpha_2).\] \End{example}

\Begin{lemma}

Dado \(p_1\in R[u_1,\dots,u_n]\), si \(\{p_1,\dots,p_l\}\) es su
órbita y \(h\in R[w_1,\dots,w_l]\) es simétrico entonces
\(h(p_1,\dots,p_l)\in R[u_1,\dots,u_n]\) también es simétrico.
\End{lemma}

\Begin{proof}

Tomemos \(\tau\in S_n\). Como la órbita es
\[S=\{\sigma(p_1) \mid \sigma\in S_n\}\] y
\(\tau(\sigma(p_i))=(\tau\sigma)(p_1)\in S\), deducimos que
\(\tau(S)\subset S\). Es más, como
\(\tau\colon R[u_1,\dots,u_n]\rightarrow R[u_1,\dots,u_n]\) es un
automorfismo, \(\tau_{|_S}\) es una permutación de \(S\). Por tanto,
ha de existir \(\tau'\in S_l\) tal que \[\tau(p_i)=p_{\tau'(i)}\] para
todo \(i\). Entonces tenemos que \[
\begin{array}{rcl}
\tau(h(p_1,\dots,p_l))&=&h(\tau(p_1),\dots,\tau(p_l))\cr
&=&h(p_{\tau'(1)},\dots,p_{\tau'(l)})\cr
&=&\tau'(h)(p_1,\dots,p_l)\cr
&=&h(p_1,\dots,p_l)
\end{array}
\] por ser \(h\) simétrica. \End{proof}

\Begin{theorem}\textrm{\normalfont (de descomposición)} Si \(K\) es el
cuerpo de descomposición de \(f\in F[x]\) y \(g\in F[x]\) es mónico e
irreducible y posee una raíz en \(K\) entonces todas las raíces
complejas de \(g\) están en \(K\). \End{theorem}

\Begin{proof}

Sean \(\alpha_1,\dots,\alpha_n\) las raíces complejas de \(f\) y
\(\beta_1\) la raíz de \(g\) que está en \(K\). Como
\(\beta_1\in K=F[\alpha_1,\dots,\alpha_n]\), existe
\(p_1\in F[u_1,\dots,u_n]\) tal que
\(\beta_1=p_1(\alpha_1,\dots,\alpha_n)\). Sea
\(\{p_1,\dots,p_l\}\) la órbita de \(p_1\) y
\(\beta_i=p_i(\alpha_1,\dots,\alpha_n)\in K\), \(1\leq i\leq l\).

Nuestro objetivo ahora es probar que las raíces complejas de \(g\) están
entre los \(\beta_1,\dots,\beta_l\in K\). Para ello consideramos el
polinomio \[h(x)=(x-\beta_1)\cdots (x-\beta_l).\] Supongamos que hemos
probado que \(h\) tiene coeficientes en \(F\). Como \(g\) es el
polinomio irreducible de \(\beta_1\) sobre \(F\) y \(\beta_1\) también
es raíz de \(h\), deduciremos que \(g|h\) en \(F[x]\), así que las
raíces de \(g\) están entre las de \(h\), que es lo que nos habíamos
propuesto demostrar.

Para ver que \(h\) tiene coeficientes en \(F\), tomamos las funciones
simétricas elementales \(s'_1,\dots, s'_l\) en \(l\) nuevas variables
\(w_1,\dots,w_l\). Los coeficientes de \(h\) son los
\[s'_i(\beta_1,\dots,\beta_l)=s'_i(p_1(\alpha_1,\dots,\alpha_n),\dots,p_l(\alpha_1,\dots,\alpha_n)).\]
Los polinomios \(s'_i(p_1,\dots,p_l)\in F[u_1,\dots,u_n]\) son
simétricos en las \(n\) variables \(u_1,\dots,u_n\) por el lema
anterior. Por el teorema de las funciones simétricas, existen
\(G_1,\dots, G_l\in F[z_1,\dots,z_n]\) tales que
\(G_i(s_1,\dots,s_n)=s'_i(p_1,\dots,p_l)\). Aquí
\(s_i\in F[u_1,\dots,u_n]\) son las funciones simétricas en las \(n\)
variables \(u_1,\dots,u_n\). Así que los coeficientes de \(h\) son \[
G_i(s_1(\alpha_1,\dots,\alpha_n),\dots,s_n(\alpha_1,\dots,\alpha_n)).
\] Sabemos que \(s_i(\alpha_1,\dots,\alpha_n)\in F\) pues son los
coeficientes de \(f\). Como los \(G_i\) también tiene coeficientes en
\(F\), deducimos de la fórmula anterior que los coeficientes de \(h\)
están en \(F\). \End{proof}

\Begin{definition}

Dada una extensión \(F\subset K\) y un subgrupo \(H\subset G(K/F)\), la
\textbf{órbita} de \(\alpha\in K\) por \(H\) es
\[\{f(\alpha)\mid f\in H\}.\] \End{definition}

\Begin{theorem}

Dada una extensión \(F\subset K\) y un subgrupo \(H\subset G(K/F)\), si
\(\{\beta_1,\dots,\beta_l\}\) es la órbita de \(\beta_1\) por \(H\)
entonces el polinomio irreducible de \(\beta_1\) sobre \(K^H\) es
\[g(x)=(x-\beta_1)\cdots (x-\beta_l).\] En particular \(\beta_1\) es
algebraico y su grado es el número de elementos de su órbita.
\End{theorem}

\Begin{proof}

Cada \(f\in H\) induce una permutación de
\(\{\beta_1,\dots,\beta_l\}\). Los coeficientes de \(g\) son
funciones simétricas elementales evaluadas en los \(\beta_i\), por
tanto no varían al aplicar \(f\in H\). Esto demuestra que estos
coeficientes están en \(K^H\).

Sea \(h\in K^H[x]\) un polinomio que tenga \(\beta_1\) como raíz. Todo
elemento de \(f\in H\subset G(K/K^H)\) envía raíces de un polinomio con
coeficientes en \(K^H\) en otras raíces, por tanto toda la órbita
\(\{\beta_1,\dots,\beta_l\}\) está formada por raíces de \(h\). Esto
prueba que \(g|h\) en \(K[x]\) y por tanto también en \(K^H[x]\). Es lo
último que faltaba para demostrar el teorema. \End{proof}

\Begin{definition}

Una extensión \(F\subset K\) es \textbf{algebraica} si todo elemento de
\(K\) es algebraico sobre \(F\). \End{definition}

Hemos visto que las extensiones finitas son algebraicas. El recíproco no
es cierto en general, pero sí bajo ciertas hipótesis.

\Begin{lemma}

Si \(F\subset K\) es una extensión algebraica y el grado de los
elementos de \(K\) sobre \(F\) está uniformemente acotado entonces la
extensión es finita. \End{lemma}

\Begin{proof}

Vamos a probar que si no fuera finita entonces existirían elementos de
grado arbitrariamente grande. Para ello construimos una sucesión
estrictamente creciente de extensiones intermedias
\[F=F_0\subsetneq F_1\subsetneq F_2\subsetneq\cdots\subsetneq K\]
tales que \(F_{i-1}\subsetneq F_i\) es finita del siguiente modo.
Supuesto construido hasta \(F_{n-1}\), tomamos un elemento
\(\alpha_n\in K\setminus F_{n-1}\) y definimos
\(F_n=F_{n-1}[\alpha_n]\). Como \(\alpha_n\) es algebraico sobre
\(F\), también lo es sobre \(F_{n-1}\), así que
\(F_{n-1}\subsetneq F_n\) es finita, y en consecuencia
\(F\subsetneq F_n\) también, así que \(F_n\subsetneq K\). Por la
fórmula del grado para extensiones intermedias \([F_n:F]\geq 2^n\), así
que cualquier elemento primitivo de \(F\subsetneq F_n\) tiene grado
\(\geq 2^n\). \End{proof}

\Begin{theorem}\textrm{\normalfont (del cuerpo fijo)} Dada una extensión
\(F\subset K\) y un subgrupo finito \(H\subset G(K/F)\),
\(K^H\subset K\) es una extensión finita de grado \([K:K^H]=|H|\).
\End{theorem}

\Begin{proof}

Hemos visto en el teorema anterior que la extensión \(K^H\subset K\) es
algebraica. Es más, el grado de cualquier elemento es el número de
elementos de una órbita, por tanto \(\leq |H|\). El lema anterior
implica pues que \(K^H\subset K\) es finita. Sea \(\gamma\in K\) un
elemento primitivo, \(K=K^H[\gamma]\). Cualquier \(f\in H\) dejan fijo a
\(K^H\), así \(f,g\in H\) son iguales si y solo si
\(f(\gamma)=g(\gamma)\). Esto demuestra que la órbita de \(\gamma\)
tiene \(|H|\) elementos. Este es por tanto el grado de \(\gamma\) sobre
\(K^H\), que es igual a \([K,K^H]\) por ser un elemento primitivo.
\End{proof}

\Begin{corollary}

Si \(F\subset K\) es una extensión finita entonces \(G(K/F)\) es un
grupo finito y \(|G(K/F)|\) divide a \([K:F]\). \End{corollary}

\Begin{proof}

Como \(F\subset K^{G(K/F)} \subset K\) es una extensión intermedia,
\(|G(K/F)|=[K,K^{G(K/F)}]\) divide a \([K:F]\). \End{proof}

\hypertarget{extensiones-de-galois}{%
\subsection{Extensiones de Galois}\label{extensiones-de-galois}}

\Begin{definition}

Una extensión finita \(F\subset K\) es de \textbf{Galois} si
\(|G(K/F)|=[K:F]\). \End{definition}

\Begin{lemma}

Dada una extensión finita \(F\subset K\) y un subgrupo
\(H\subset G(K/F)\), la extensión \(K^H\subset K\) es de Galois y
\(H=G(K/K^H)\). \End{lemma}

\Begin{proof}

Sabemos que, en general, \(H\subset G(K/K^H)\) es un subgrupo, así que
\(|H|\leq |G(K/K^H)|\). También sabemos que \(|G(K/K^H)|\) divide a
\([K:K^H]=|H|\), así que tenemos también la otra desigualdad
\(|G(K/K^H)|\leq|H|\). Esto prueba que \(H=G(K/K^H)\), por tanto esta
extensión es de Galois. \End{proof}

\Begin{lemma}

Sea \(F\subset K=F[\gamma_1]\) una extensión finita, \(g\in F[x]\) es
el polinomio irreducible de \(\gamma_1\) y
\(\gamma_1,\dots,\gamma_r\in K\) las distintas raíces de \(g\) en este
cuerpo. Para cada \(1\leq i\leq n\) existe un único \(f_i\in G(K/F)\)
tal que \(f_i(\gamma_1)=\gamma_i\). Es más,
\(G(K/F)=\{f_1,\dots,f_r\}\). \End{lemma}

\Begin{proof}

Todos los \(\gamma_i\) poseen el mismo grado sobre \(F\) ya que tienen
el mismo polinomio irreducible \(g\), por tanto \(K=F[\gamma_i]\) para
todo \(i\). Sabemos que, para cada \(i\), hay un único isomorfismo
\[h_i\colon \frac{F[x]}{(g)}\cong K\] que deja fijo a \(K\) tal que
\(h(\bar{x})=\gamma_i\). Por tanto, \(f_i=h_ih_1^{-1}\in G(K/F)\) es
el único que satisface la propiedad del enunciado. Todo elemento
\(f\in G(K/F)\) está determinado por \(f(\gamma_1)\) y además preserva
raíces de \(g\in F[x]\), así que \(G(K/F)\) consta necesariamente de los
\(f_i\) anteriores. \End{proof}

\Begin{theorem}

Dada una extensión finita \(F\subset K\), los siguientes enunciados son
equivalentes:

\begin{enumerate}
\def\labelenumi{\arabic{enumi}.}
\tightlist
\item
  \(F\subset K\) es de Galois.
\item
  \(F=K^{G(K/F)}\).
\item
  \(K\) es el cuerpo de descomposición de un polinomio de \(F[x]\).
\end{enumerate}

\End{theorem}

\Begin{proof}

Veamos \(1.\Leftrightarrow 2.\) Por el teorema del cuerpo fijo,
\(|G(K/F)|=[K:K^{G(K/F)}]\). Como \(F\subset K^{G(K/F)}\subset K\),
\(|G(K/F)|=[K:F]\) si y solo si \(F=K^{G(K/F)}\).

Probemos ahora que \(1.\Leftrightarrow 3.\) Sea \(\gamma_1\in K\) un
elemento primitivo de \(F\subset K\), \(g\in F[x]\) su polinomio
irreducible y \(L\) el cuerpo de descomposición de \(g\). Sean
\(\gamma_1,\dots,\gamma_n\in \mathbb{C}\) las distintas raíces
complejas de \(g\), de las cuales \(\gamma_1,\dots,\gamma_r\in K\) y
el resto no están en \(K\). Denotemos \(n=[G:K]\). El grado de \(g\) es
\(n\). Como \(K=F[\gamma_1]\) y \(L=F[\gamma_1,\dots,\gamma_n]\),
\(F\subset K\subset L\). Usando el lema anterior vemos que
\(F\subset K\) es de Galois \(\Leftrightarrow\) \(r=n\)
\(\Leftrightarrow\) todas las raíces complejas de \(g\) están en \(K\)
\(\Leftrightarrow\) \(K\supset L\) \(\Leftrightarrow\) \(K=L\)
\(\Leftrightarrow\) \(K\) es un cuerpo de descomposición. En el último
paso hemos usado que \(g\) tiene una raíz en \(K\). \End{proof}

\Begin{corollary}

Toda extensión finita \(F\subset K\) es una extensión intermedia
\(F\subset K\subset L\) de una extensión de Galois \(F\subset L\).
\End{corollary}

\Begin{proof}

Basta tomar \(L\) como el cuerpo de descomposición de un elemento
primitivo de \(F\subset K\). \End{proof}

\Begin{corollary}

Si \(F\subset K\) es una extensión de Galois y \(F\subset L\subset K\)
es una extensión intermedia entonces \(L\subset K\) también es de
Galois. \End{corollary}

\Begin{proof}

Basta observar que si \(K\) es el cuerpo de descomposición de
\(g\in F[x]\) entonces también es el cuerpo de descomposición de del
mismo polinomio visto como polinomio con coeficientes en \(L\),
\(g\in L[x]\). \End{proof}

\Begin{theorem}\textrm{\normalfont (fundamental de la teoría de Galois)}
Dada una extensión de Galois \(F\subset K\), las siguientes aplicaciones
son biyectivas y mutuamente inversas:
\[\begin{array}{rcl}\left\{\text{ext. intermedias }F\subset L\subset K\right\}&\longleftrightarrow& \left\{\text{subgrupos }H\subset G(K/F)\right\},\cr L&\mapsto&G(K/L),\cr K^H&\leftarrow&H.\end{array}\]
\End{theorem}

\Begin{proof}

Dado un subgrupo \(H\subset G(K/F)\), ya hemos probado en un lema
anterior que \(H=G(K/K^H)\), así que la composición que empieza y acaba
en la derecha es la identidad. Dada ahora una extensión intermedia
\(F\subset L\subset K\), acabamos de probar que \(L\subset K\) es de
Galois, así que por el teorema anterior \(K^{G(K/L)}=L\). \End{proof}

\Begin{remark}

Observa que la correspondencia dada en el Teorema Fundamental da la
vuelta a las inclusiones. Es decir, dados dos subgrupos
\(H'\subset H\subset G(K/F)\) tenemos que \(K^{H'}\supset K^H\) y dadas
extensiones intermedias \(F\subset L\subset L'\subset K\) tenemos que
\(G(K/L)\supset G(K/L')\). El subgrupo trivial se corresponde con \(K\)
y el total con \(F\). \End{remark}

\Begin{corollary}

Toda extensión finita \(F\subset K\) posee una cantidad finita de
extensiones intermedias. \End{corollary}

\Begin{proof}

Cuando la extensión es de Galois el resultado es cierto porque el grupo
\(G(K/F)\), que es finito, tiene una cantidad finita de subgrupos, que
se corresponden con las extensiones intermedias. Si \(F\subset K\) no
fuera de Galois, basta tomar \(F\subset K\subset L\) con \(F\subset L\)
de Galois y observar que toda extensión intermedia de \(F\subset K\) lo
es también de \(F\subset L\). \End{proof}

\Begin{theorem}

Dada una extensión de Galois \(F\subset K\) y una extensión intermedia
\(F\subset L\subset K\), \(F\subset L\) es de Galois si y solo si el
subgrupo \(G(K/L)\subset G(K/F)\) es normal. En dicho caso
\[\frac{G(K/F)}{G(K/L)}\cong G(L/F).\] \End{theorem}

\Begin{proof}

Comenzaremos probando la equivalencia de la primera parte del enunciado.

Sea \(\gamma_1\in L\) un elemento primitivo, \(L=F[\gamma_1]\), con
polinomio irreducible \(g\in F[x]\). Sean
\(\gamma_1,\dots,\gamma_r\in K\) sus raíces complejas, que están en
\(K\) porque es un cuerpo de descomposición y \(\gamma_1\in K\).

\(\Rightarrow\) Por ser \(F\subset L\) de Galois, \(L\) es el cuerpo de
descomposición de \(g\), así que \(L=F[\gamma_1,\dots,\gamma_r]\).
Todo \(f\in G(K/L)\) preserva raíces de \(g\), por tanto se restringe
\(f_{|_{L}}\colon L\rightarrow L\) y esta restricción está determinada
por \(f(\gamma_1)\) que será algún \(\gamma_i\). En particular
\(f_{|_{L}}\) es la identidad si y solo si \(f(\gamma_1)=\gamma_1\).

Sea \(h\in G(K/L)\) un elemento cualquiera. Para ver que este grupo es
normal tenemos que probar que \(f^{-1}hf\in G(K/F)\) deja fijo a \(L\) y
por tanto \(f^{-1}hf\in G(K/L)\), es decir, que hay que probar que
\((f^{-1}hf)(\gamma_1)=1\). Esto es cierto porque \(h\) deja fijo a
\(L\), así que \[
\begin{array}{rcl}
(f^{-1}hf)(\gamma_1)&=&f^{-1}(h(f(\gamma_1)))\cr
&=&f^{-1}(h(\gamma_i))\cr
&=&f^{-1}(\gamma_i)\cr
&=&\gamma_1.
\end{array}
\]

\(\Leftarrow\) Si \(F\subset L\) no fuera de Galois no podría ser el
cuerpo de descomposición de \(g\), así que alguna raíz de \(g\) no
estaría en \(L\). Supongamos que \(\gamma_i\) es tal raíz. Como
\(L=K^{G(K/L)}\) y \(\gamma_i\notin L\), existe \(h\in G(K/L)\) tal qye
\(h(\gamma_i)\neq\gamma_i\). Es más, como \(F=K^{G(K/F)}\), las raíces
de \(g\) son la órbita de \(\gamma_1\) por \(G(K/F)\), así que existe
\(f\in G(K/F)\) tal que \(f(\gamma_1)=\gamma_i\). El elemento
\(f^{-1}hf\in G(K/F)\) no puede dejar fijo a \(\gamma_1\) ya que de lo
contrario
\(\gamma_i=f(\gamma_1)=hf(\gamma_1)=h(\gamma_i)\neq\gamma_i\). Esto
implica que \(f^{-1}hf\) no deja fijo a \(L\), luego
\(f^{-1}hf\in G(K/L)\).

Una vez establecida la equivalencia de la primera parte del enunciado,
demostraremos el isomorfismo de la segunda. Supongamos pues que
\(F\subset L\) es de Galois. Hemos visto que entonces todo
\(f\in G(K/F)\) se restringe a \(L\), es decir
\(f_{|_{L}}\in G(L/F)\). Esta restricción induce un homomorfismo de
grupos \[
\begin{array}{rcl}
G(K/F)&\longrightarrow&G(L/F),\cr
f&\mapsto&f_{|_{L}}.
\end{array}
\] Obviamente \(G(K/L)\) está contenido en el núcleo de este
homomorfismo ya que los elementos de \(G(K/L)\) se restringen a la
identidad sobre \(L\). Este homomorfismo es sobreyectivo porque
\(G(L/F)\) tiene \(r\) elementos, uno por cada raíz \(\gamma_i\) de
\(g\) determinado por \(\gamma_1\mapsto\gamma_i\), y además hemos
visto que en \(G(K/F)\) siempre hay elementos que satisfacen
\(\gamma_1\mapsto\gamma_i\). Por el primer teorema de isomorfía y el
teorema de Lagrange, el número de elementos del núcleo núcleo es \[
\frac{|G(K/F)|}{|G(L/F)|}=\frac{[K:F]}{[L:F]}=[K:L]=|G(K/L)|.
\] Por tanto \(G(K/L)\) es todo el núcleo y el isomorfismo del enunciado
es el definido por el homomorfismo de restricción y el primer teorema de
isomorfía, \[
\begin{array}{rcl}
\frac{G(K/F)}{G(K/L)}&\stackrel{\cong}\longrightarrow&G(L/F),\cr
[f]&\mapsto&f_{|_{L}}.
\end{array}
\] \End{proof}

\hypertarget{extensiones-ciclotuxf3micas}{%
\subsection{Extensiones
ciclotómicas}\label{extensiones-ciclotuxf3micas}}

Dado \(n\geq 1\), las \textbf{raíces \(n\)-ésimas de la unidad} son las
\(n\) raíces complejas diferentes del polinomio \[x^n-1,\] que son
\[e^{\frac{2\pi i t}{n}}, \qquad 0\leq t{<}n.\] El conjunto formado por
estos \(n\) números complejos es un grupo cíclico de orden \(n\) para la
multiplicación, generado por la \textbf{raíz \(n\)-ésima primitiva},
\[\zeta=\zeta_n=e^{\frac{2\pi i}{n}}.\] Si \(n=p\) es primo, cualquier
raíz distinta de \(1\) genera estre grupo.

\Begin{proposition}

Dado un entero primo \(p\geq 1\), la extensión
\(\mathbb Q\subset\mathbb Q[\zeta]\) es de Galois de grado \(p-1\) y su
grupo de Galois es cíclico. \End{proposition}

\Begin{proof}

El cuerpo de descomposición de \(x^p-1\) es
\[\mathbb Q[1,\zeta,\dots,\zeta^{p-1}]=\mathbb Q[\zeta].\] En efecto,
\(\supset\) es obvio y \(\subset\) es consecuencia de que como
\(\zeta\in\mathbb Q[\zeta]\) entonces todas las potencias
\(\zeta^t\in \mathbb Q[\zeta]\), \(0\leq t<n\), también. Esto demuestra
que \(\mathbb Q[\zeta]\) es de Galois.

Sabemos que \[x^p-1=(x-1)q(x)\] donde \[q(x)=x^{p-1}+\cdots+x+1\] es el
\(p\)-ésimo polinomio ciclotómico, que según vimos es irreducible. Como
\(\zeta\neq 1\), \(\zeta\) ha de ser raíz de \(q(x)\), así que el grado
de la extensión es \(p-1\).

Para ver que el grupo de Galois es cíclico, definimos un homomorfismo
\[\psi\colon G(\mathbb Q[\zeta]/\mathbb Q)\longrightarrow (\mathbb Z/(p))^\times\]
que llega al grupo \((\mathbb Z/(p))^\times\) de unidades del cuerpo
\(\mathbb Z/(p)\). Este último grupo sabemos que es cíclico de orden
\(p-1\). Todo \(f\in G(\mathbb Q[\zeta]/\mathbb Q)\) preserva raíces de
\(q(x)\), así que \[f(\zeta)=\zeta^i\] para cierto \(0{<}i{<}p\) único y
dependiente de \(f\). Definimos \[\psi(f)=\bar i.\] Acabamos de probar
que esta aplicación está bien definida. Veamos ahora que es un
homomorfismo. Dado \(g\in G(\mathbb Q[\zeta]/\mathbb Q)\), hay un único
\(0{<}j{<}p\) tal que \[g(\zeta)=\zeta^j\] y que define
\(\psi(g)=\bar j\). Entonces \[\begin{array}{rcl}
(f\circ g)(\zeta)&=&f(g(\zeta))\cr
&=&f(\zeta^j)\cr
&=&f(\zeta)^j\cr
&=&(\zeta^i)^j\cr
&=&\zeta^{ij}.
\end{array}\] Por tanto \[\begin{array}{rcl}
\psi(f\circ g)&=&\overline{ij}\cr
&=&\bar{i}\bar{j}\cr
&=&\psi(f)\psi(g).
\end{array}\] Esto demuestra que \(\psi\) es un homomorfismo.

Veamos que \(\psi\) es inyectivo. Si
\(f\in G(\mathbb Q[\zeta]/\mathbb Q)\) es tal que \[\psi(f)=\bar 1\] es
porque \[f(\zeta)=\zeta.\] Como \(f\) actúa como la identidad sobre los
racionales y sobre el generador de la extensión \(\mathbb{Q}[\zeta]\),
\(f\) ha de ser la identidad. Esto prueba que el núcleo de \(\psi\) es
trivial, así que es un homomorfismo inyectivo. \End{proof}

\hypertarget{extensiones-de-kummer}{%
\subsection{Extensiones de Kummer}\label{extensiones-de-kummer}}

Dado un cuerpo \(F\), nuestro objetivo es estudiar el cuerpo de
descomposición \(K\) del polinomio \[q(x)=x^p-a\in F[x]\] donde \(p\) es
primo y \(a\) no tiene raíces \(p\)-ésmas en \(F\). Si \(\alpha\) es una
raíz compleja de \(q(x)\), entonces el conjunto de todas sus raíces es
\[\alpha,\zeta_p\alpha,\dots,\zeta_p^{p-1}\alpha,\] donde \(\zeta_p\)
es la raíz \(p\)-ésima primitiva de la unidad, ya que todas son raíces
del polinomio \(q(x)\) anterior y son todas distintas, pues \(\zeta_p\)
tiene orden \(p\) para el producto. En particular si \(\zeta_p\in F\)
entonces \(K=F[\alpha]\).

\Begin{proposition}

Si \(\zeta_p\in F\) y \(q(x)=x^p-a\in F[x]\) no tiene raíces en \(F\)
entonces el cuerpo de descomposición \(K\) de \(q(x)\) tiene grado \(p\)
sobre \(F\). \End{proposition}

\Begin{proof}

Sea \(\alpha\) una raíz compleja de \(q(x)\). Hemos observado que
\(K=F[\alpha]\) y \(\alpha\) es una raíz de \(q(x)\), que es de grado
\(p\), por tanto \([K:F]\leq p\). Al ser \(F\subset K\) de Galois, para
probar la otra desigualdad bastará ver que \([K:F]=|G(K/F)|\geq p\).

Como \(\alpha\notin F= K^{G(K/F)}\), ha de existir algún \(f\in G(K/F)\)
tal que \(f(\alpha)\neq\alpha\). Como \(f\) preserva raíces de
polinomios en \(F[x]\), \(f(\alpha)=\zeta_p^i\alpha\) para cierto
\(0{<}i{<}p\). Usaremos esto para ver que las potencias \(f^j\) de \(f\)
son diferentes para todo \(0\leq j{<}p\), así que \(G(K/F)\) tendrá en
efecto al menos \(p\) elementos. Para ello basta comprobar que cada una
de estas potencias \(f^j\) manda \(\alpha\) a un elemento diferente.
Vamos a probar por inducción que
\[f^j(\alpha)=(\zeta_p^{i})^{j}\alpha.\] Todos estos elementos son
diferentes ya que al ser \(p\) primo todas las potencias de \(\zeta_p\)
distintas de \(1\), por ejempo \(\zeta_p^i\), tienen orden
multiplicativo \(p\), así que todos los \((\zeta_p^i)^{j}\) son
diferentes para \(0\leq j{<}p\). Para \(j=1\) la ecuación anterior es
obviamente cierta. Supongamos que es cierta para \(j-1\). Como
\(\zeta_p\in F\) entonces \(f(\zeta_p)=\zeta_p\) ya que. Por tanto,
\[\begin{array}{rcl}
f^j(\alpha)&=&f(f^{j-1}(\alpha))\cr
&=&f((\zeta_p^{i})^{j-1}\alpha)\cr
&=&(f(\zeta_p)^{i})^{j-1}f(\alpha)\cr
&=&(\zeta_p^{i})^{j-1}\zeta_p^i\alpha\cr
&=&(\zeta_p^{i})^{j}\alpha.
\end{array}\] \End{proof}

\Begin{remark}

A posteriori vemos que, en las condiciones de la proposición anterior,
\(x^p-a\) es un polinomio irreducible, pues cualquiera de sus raíces
complejas tiene grado \(p\). \End{remark}

Sorprendentemente el resultado a anterior tiene un recíproco.

\Begin{theorem}

Si \(p\) es un primo, \(F\) es un cuerpo tal que \(\zeta_p\in F\) y
\(F\subset K\) es una extensión de Galois de grado \([K:F]=p\) entonces
\(K=F[\alpha]\) para cierto \(\alpha\in K\) que es raíz de un polinomio
de la forma \(x^p-a\in F[x]\). \End{theorem}

\Begin{proof}

Al ser la extensión de Galois \(|G(K/F)|=[K:F]=p\), por tanto \(G(K/F)\)
es cíclico de orden \(p\), así que todo \(f\in G(K/F)\) distinto de la
identidad genera el grupo de Galois,
\[G(K/F)=\{f^0,f^1,\dots, f^{p-1}\}.\]

Ahora vamos a concentrarnos en el hecho de que \(f\) es un homomorfismo
de \(F\)-espacios vectoriales. Como \(f^p=\operatorname{id}_K\) tenemos
que cualquier autovalor \(\lambda\) de \(f\) satisface \(\lambda^p=1\),
es decir, sus autovalores son raíces \(p\)-ésimas de la unidad. Además
\(f\) es diagonalizable, pues niguna potencia de una caja de Jordan de
tamaño \(2\times 2\) o superior es la matriz identidad. Como \(f\) es
diagonalizable y distinto de la identidad, tendrá que tener algún
autovalor \(\lambda\neq 1\). Este autovalor ha de ser forzosamente de la
forma \(\lambda =\zeta_p^i\) para cierto \(0{<}i{<}p\).

Sea \(\alpha\in K\) un autovector asociado a \(\zeta_p^i\),
\[f(\alpha)=\zeta_p^i\alpha.\] Tenemos entonces que
\[\begin{array}{rcl}
f(\alpha^p)&=&f(\alpha)^p\cr
&=&(\zeta_p^i\alpha)^p\cr
&=&(\zeta_p^i)^p\alpha^p\cr
&=&\alpha^p.
\end{array}\] Se deduce por inducción que \(f^i(\alpha^p)=\alpha^p\)
para todo \(i\geq 1\), por tanto \(\alpha^p\in K^{G(K/F)}=F\). Esto
demuestra que \(\alpha\in K\) es raíz del polinomio
\(x^p-\alpha^p\in F[x]\). Además, como \(f(\alpha)\neq \alpha\) entonces
\(\alpha\notin F\) así que \(F\subsetneq F[\alpha]\subset K\) y como
\([K:F]=p\) es primo concluimos que \(K=F[\alpha]\). \End{proof}

Igual que antes, en las condiciones del enunciado de este teorema el
polinomio \(x^p-a\) es necesariamente irreducible.

Las extensiones del tipo que hemos estudiado en esta sección se
denominan \textbf{extensiones de Kummer}.

\hypertarget{solubilidad-por-radicales}{%
\subsection{Solubilidad por radicales}\label{solubilidad-por-radicales}}

\Begin{definition}

Decimos que \(\alpha\in \mathbb{C}\) es \textbf{soluble} sobre un cuerpo
\(F\) si existe una cadena de extensiones
\[F = F_0\subset F_1\subset F_2\subset\cdots\subset  F_n=K\] tal que
\(\alpha\in K\) y \(F_{i+1}=F_i[\sqrt[s_i]{r_i}]\) para ciertos
\(r_i\in F_i\) y \(s_i\geq 2\), \(0\leq i<n\). \End{definition}

Los números solubles sobre \(F\) son los que se obtienen a partir de
números de \(F\) realizando iteradamente sumas, restas, productos,
divisiones por números no nulos y raíces \(n\)-ésimas. Nuestro objetivo
es saber cuándo podemos hallar las raíces de un polinomio
\(p(x)\in F[x]\) de este modo a partir de sus coeficientes, es decir,
queremos saber cuándo las raíces de \(p(x)\) son solubles sobre \(F\).
Veremos cómo hacerlo usando el grupo de Galois del cuerpo de
descomposición de \(p(x)\).

\Begin{remark}

Como \(\sqrt[st]{r}=\sqrt[s]{\sqrt[t]{r}}\), no hay pérdida de
generalidad sin en la definición anterior suponemos que los \(s_i\) son
todos primos.

Añadiendo las raíces de manera sucesiva vemos que si
\(\alpha_1,\dots,\alpha_m\in\mathbb{C}\) son solubles entonces existe
una cadena de extensiones como la de la definición tal que
\(\alpha_1,\dots,\alpha_m\in K\). \End{remark}

\Begin{definition}

Un grupo \(G\) es \textbf{soluble} si existe una cadena de subgrupos
\[\{e\}=G_0 \subset G_1 \subset G_2 \subset \cdots\subset G_n=G\]
tal que \(G_i\subset G_{i+1}\) es un subgrupo normal con cociente
\(G_{i+1}/G_i\) abeliano para todo \(0\leq i<n\). \End{definition}

La solubilidad es una buena propiedad porque permite probar por
inducción que muchas propiedades de los grupos abelianos son también
ciertas para los grupo solubles.

\Begin{remark}

Los grupos abelianos son solubles. Los grupos simétricos \(S_2\),
\(S_3\) y \(S_4\) también, así como todos sus subgrupos. Sin embargo,
\(S_n\) no es soluble para ningún \(n\geq 5\), ni tampoco su subgrupo
alternado \(A_n\subset S_n\). La solubilidad se preserva por
isomorfismos. \End{remark}

\Begin{lemma}

Dado un grupo \(G\) y un subgrupo normal \(N\), \(G\) es soluble si y
solo si lo son \(N\) y \(G/N\). \End{lemma}

\Begin{proof}

Denotamos \(p\colon G\twoheadrightarrow G\subset N\) a la proyección
natural.

\(\Rightarrow\) Si
\[\{e\}=G_0 \subset G_1 \subset G_2 \subset \cdots\subset G_n=G\]
es una cadena en las condiciones de la definición entonces las cadenas
siguientes demuestran que \(N\) y \(G/N\) son solubles, \[
\begin{array}{c}
\{e\}=N\cap G_0 \subset N\cap G_1 \subset \cdots\subset N\cap G_n=N,\cr
\{e\}=p(G_0) \subset p(G_1) \subset \cdots\subset p(G_n)=G/N.
\end{array}
\] Aquí usamos que, gracias al primer teorema de isomorfía, \[
\frac{N\cap G_{i+1}}{N\cap G_{i}}\subset\frac{G_{i+1}}{G_i}
\cong\frac{p(G_{i+1})}{p(G_{i})}.
\]

\(\Leftarrow\) Si \(N\) y \(G/N\) son solubles gracias a las cadenas \[
\begin{array}{c}
\{e\}=N_0 \subset N_1 \subset N_2 \subset \cdots\subset N_m=N,\cr
\{e\}=K_0 \subset K_1 \subset K_2 \subset \cdots\subset K_n=G/N,
\end{array}
\] entonces \(G\) es soluble gracias a la cadena \[
\{e\}=N_0 \subset \cdots\subset N_m=p^{-1}(K_0)\subset \cdots\subset p^{-1}(K_m)=G.
\] Aquí usamos que, gracias al primer teorema de isomorfía, \[
\frac{p^{-1}(K_{i+1})}{p^{-1}(K_{i})}\cong \frac{K_{i+1}}{K_i}.
\] \End{proof}

\Begin{corollary}

Dos grupos \(G\) y \(H\) son solubles si y solo si \(G\times H\) es
soluble. \End{corollary}

\Begin{proof}

Basta usar el primer teorema de isomorfía para observar que
\(G\cong G\times \{e\}\subset G\times H\) es un subgrupo normal y
\((G\times H)/(G\times \{e\})\cong H\). \End{proof}

\Begin{proposition}

Un grupo finito \(G\) es soluble si y solo si existe una cadena de
subgrupos
\[\{e\}=G_0 \subset G_1 \subset G_2 \subset \cdots\subset G_n=G\]
tal que \(G_i\subset G_{i+1}\) es un subgrupo normal con cociente
\(G_{i+1}/G_i\) de orden primo, \(0\leq i<n\). \End{proposition}

\Begin{proof}

Antes que nada, observamos que la demostración del lema anterior también
sirve para probar que si \(G\) es un grupo y \(N\subset G\) es un
subgrupo normal, entonces \(G\) satisface la condición del enunciado de
esta proposición si y solo si \(N\) y \(G/N\) la cumplen. En particular,
dos grupos \(G\) y \(H\) la satisfacen si y solo si el producto
\(G\times H\) la cumple. Partiendo de esto, abordamos ahora la prueba de
esta proposición.

\(\Leftarrow\) Es obvio porque todo grupo de orden primo es cíclico y
por tanto abeliano.

\(\Rightarrow\) Si \(G=\mathbb{Z}/(p^n)\) basta tomar
\(G_i=(\bar{p}^{n-i})\), \(0\leq i<n\), ya que todos los subgrupos de
\(G\) son normales por ser abelianos y
\(\bar{p}^{n-i}\in\mathbb{Z}/(p^n)\) tiene orden \(p^i\), así que
\(|G_i|=p^i\) y por tanto \[
\begin{array}{rcl}
\left|\frac{G_{i+1}}{G_i}\right|&=&\frac{|G_{i+1}|}{|G_i|}\cr
&=&\frac{p^{i+1}}{p^i}\cr
&=&p.
\end{array}\] Si \(G\) es abeliano el resultado también es cierto, ya
que al ser finito sería un producto finito de grupos de la forma
\(\mathbb{Z}/(p^n)\), en virtud del segundo teorema de estructura.

En general, si \(G\) satisface la condición de solubilidad gracias a la
cadena \[
\{e\}=G_0 \subset G_1 \subset G_2 \subset \cdots\subset G_n=G,
\] vamos a probar por inducción que cada \(G_i\) satisface la condición
del enunciado. Obviamente \(G_0\) la satisface por ser trivial. Si
\(G_i\) la cumple, como \(G_i\subset G_{i+1}\) es normal y
\(G_{i+1}/G_i\) la satisface por ser abeliano, tenemos que
\(G_{i+1}\) también la cumple. \End{proof}

\Begin{lemma}

Dados dos polinomios \(f_1,f_2\in F[x]\), si \(L_1\) y \(L_2\) son
los cuerpos de descomposición de \(f_1\) y \(f_2\), respectivamente, y
\(K\) es el cuerpo de descomposición de \(f_1f_2\) entonces \(G(K/F)\)
es isomorfo a un subgrupo de \(G(L_1/F)\times G(L_2/F)\). \End{lemma}

\Begin{proof}

Tenemos que \(F\subset L_1,L_2\subset K\), ya que las raíces de un
producto de dos polinomios son la unión de las raíces de los factores.
Consideramos el homomorfismo
\[G(K/F)\longrightarrow\frac{G(K/F)}{G(K/L_1)}\times \frac{G(K/F)}{G(K/L_2)}\cong G(L_1/F)\times G(L_2/F)\]
definido en cada coordenada como la proyecció natural. El núcelo es
\(G(K/L_1)\cap G(K/L_2)\), es decir, los automorfismos de \(K\) que
dejan fijas a las raíces tanto de \(f_1\) como de \(f_2\). Un
automorfismo así deja fijas a las raíces de \(f_1f_2\) y por tanto a
su cuerpo de descomposición \(K\), así que tiene que ser la identidad.
Como el núcleo es trivial, el homomorfismo es inyectivo y, en virtur del
primer teorema de isomorfía, el dominio es isomorfo a un subgrupo del
codominio. \End{proof}

\Begin{lemma}

Si \(p_1,\dots,p_m\) son enteros primos dos a dos entonces el grupo de
Galois de la extensión de Galois
\(F\subset F[\zeta_{p_1},\dots,\zeta_{p_m}]\) es abeliano.
\End{lemma}

\Begin{proof}

Para \(m=1\) se prueba como en el caso \(F=\mathbb{Q}\). Por inducción,
si es cierto para \(m-1\) primos, nuestro grupo de Galois es isomorfo a
un subgrupo del producto de los de
\(F\subset F[\zeta_{p_1},\dots,\zeta_{p_{m-1}}]\) y
\(F\subset F[\zeta_{p_m}]\) en virtud del lema anterior. El producto
de grupos abelianos es abeliano y los subgrupos de los grupos abelianos
también. \End{proof}

\Begin{theorem}

Sea \(p(x)\in F[x]\) un polinomio con cuerpo de descomposición \(L\).
Las raíces complejas de \(p(x)\) son todas solubles sobre \(F\) si y
solo si \(G(L/F)\) es un grupo soluble. \End{theorem}

\Begin{proof}

\(\Leftarrow\) Denotamos \(G=G(L/F)\). Sea
\[\{e\}=G_0 \subset G_1 \subset G_2 \subset \cdots\subset G_n=G\]
una cadena de subgrupos como en la proposición anterior.

Supongamos primero que \(F\) tiene todas las raíces primitivas de la
unidad asociadas a los primos que aparecen como el orden de los
cocientes \(G_{i+1}/G_i\). En este caso basta considerar la cadena de
extensiones
\[L= L^{G_0} \supset L^{G_1} \supset \cdots\supset L^{G_n}\supset F[\zeta_{p_1},\dots, \zeta_{p_n}]\=F.\]
En efecto, el teorema sobre extensiones de Kummer garantiza que cada
\(L^{G_{i}}\supset L^{G_{i+1}}\) se obtiene añadiendo una raíz.

Si \(F\) no tuviera todas las raíces primitivas de la unidad
mencionadas, denotamos \(F'\) y \(L'\) a los cuerpos obtenidos al
añadírselas a \(F\) y a \(L\), respectivamente. Por construcción, la
extensión \(F\subset F'\) se puede interpolar como en la definición de
número soluble. El grupo \(G(L'/L)\) es abeliano por el lema anterior y
\[\frac{G(L'/F)}{G(L'/L)}\cong G(L/F),\] que es soluble, así que
\(G(L'/F)\) es soluble y su subgrupo normal \(G(L'/F')\) también. Como
\(F'\) posee todas las raíces primitivas de la unidad necesarias, el
párrafo anterior demuestra que \(F'\subset L'\) también se puede
interpolar como en la definición de número soluble. Por tanto
\(F\subset L'\) también, concatenando ambas interpolaciones.

\(\Rightarrow\) El argumento es muy parecido anterior. Lo dejamos como
ejercicio. \End{proof}
